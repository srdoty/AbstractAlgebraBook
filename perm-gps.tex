\documentclass[11pt]{article}
\usepackage[nohead,margin=1.50in]{geometry} %set margins
\usepackage{amsmath,amssymb,amsthm,pdiag} %AMS packages for math stuff
\usepackage{enumitem}
\setlist{topsep=1pt,itemsep=0pt,parsep=1pt,leftmargin=0.7cm}
\setenumerate[1]{label=(\alph*)}

\newenvironment{problems}
{
 \begin{enumerate}[topsep=1pt,itemsep=0pt,parsep=2pt,leftmargin=0.6cm,%
 label={\arabic*.}, ref=\arabic*] \small
}
{
 \end{enumerate}
}

%%% Define some theorem and example environments. The starred versions
%%% are un-numbered and the unstarred versions are numbered.
\newtheoremstyle{plain}
  {\topsep}   % ABOVESPACE
  {\topsep}   % BELOWSPACE
  {\slshape}  % BODYFONT
  {0pt}       % INDENT (empty value is the same as 0pt)
  {\bfseries} % HEADFONT
  {.}         % HEADPUNCT
  {5pt plus 1pt minus 1pt} % HEADSPACE
  {}          % CUSTOM-HEAD-SPEC

\swapnumbers
\newtheorem{thm}{Theorem}[section]
\newtheorem{lem}[thm]{Lemma}
\newtheorem{prop}[thm]{Proposition}
\newtheorem{cor}[thm]{Corollary}
\newtheorem*{thm*}{Theorem}
\newtheorem*{lem*}{Lemma}
\newtheorem*{prop*}{Proposition}
\newtheorem*{cor*}{Corollary}

\theoremstyle{definition}
\newtheorem{defn}[thm]{Definition}
\newtheorem{example}[thm]{Example}
\newtheorem{examples}[thm]{Examples}
\newtheorem{rmk}[thm]{Remark}
\newtheorem{conv}[thm]{Convention}
\newtheorem*{defn*}{Definition}
\newtheorem*{example*}{Example}
\newtheorem*{examples*}{Examples}
\newtheorem*{rmk*}{Remark}
\newtheorem*{conv*}{Convention}

%%% Define some convenient abbreviations for common mathematical
%%% notations.
\newcommand{\R}{\mathbb{R}} % use \R for the real numbers
\newcommand{\C}{\mathbb{C}} % use \C for the complex numbers
\newcommand{\Z}{\mathbb{Z}} % use \Z for the integers
\newcommand{\Q}{\mathbb{Q}} % use \Q for the rationals
\newcommand{\N}{\mathbb{N}} % use \N for the natural numbers
\newcommand{\compose}{\circ} % functional composition
\renewcommand{\implies}{\Rightarrow}
\renewcommand{\iff}{\Leftrightarrow}
\newcommand{\F}{{\mathbb F}}
\newcommand{\gen}[1]{\langle #1 \rangle}
\newcommand{\End}{\operatorname{End}}
\newcommand{\GL}{\mathrm{GL}}
\newcommand{\SL}{\mathrm{SL}}
\renewcommand{\O}{\mathrm{O}}
\newcommand{\SO}{\mathrm{SO}}
\newcommand{\U}{\mathrm{U}}
\newcommand{\SU}{\mathrm{SU}}
\newcommand{\g}{\mathfrak{g}}
\newcommand{\transpose}{\mathsf{T}}
\newcommand{\B}{\mathcal{B}}
\newcommand{\Rep}{\operatorname{Rep}}
\newcommand{\Mat}{\operatorname{Mat}}
\newcommand{\inner}[2]{\langle #1, #2 \rangle}
\newcommand{\sgn}{\operatorname{sgn}}
\newcommand{\n}{\underline{\mathbf{n}}}
\newcommand{\Sym}{\mathbb{S}}
\newcommand{\Alt}{\mathbb{A}}
\newenvironment{perm}[2]{\left(\begin{smallmatrix}#1 \\ #2}{\end{smallmatrix}\right)}


\allowdisplaybreaks
\parskip=2pt

%\title{Document Title}
%\author{author's name}

\begin{document}%\maketitle
\setcounter{section}{5}

\section{Permutations}\noindent
Lagrange and Galois studied permutations among the roots of
polynomials as a way of understanding solutions of polynomial
equations.  This eventually led to what is now called \emph{group
  theory}.

\begin{defn}\index{permutation}
A {\em permutation} of a set $X$ is a bijection $X \to X$.  Write
$\Sym_X$ for the set of all permutations of $X$; i.e.,
\[ 
\Sym_X = \{ \sigma: X \to X \mid \sigma \text{ is a bijection} \}.
\] 
In the special case $X = \n := \{1, 2, \dots, n \}$ we write $\Sym_n$
instead of $\Sym_X$.
\end{defn}

So $\Sym_n$\index{S@$\Sym_n$} is the set of all permutations of the
numbers from 1 to $n$. We can think of it as the set of all
permutations of any $n$ things, since we can always assign numbers
from 1 to $n$ to those things.

If $X$ is a finite set of $n$ elements, then the number of
permutations of $X$ is $n!$, the factorial of $n$. So the cardinality
$|\Sym_n| = n!$. If $X$ is an infinite set then $\Sym_X$ is also
infinite.



\begin{defn}\index{two-line~notation}
  If $\sigma \in \Sym_n$, then we can depict the permutation by
  writing $$\sigma =
  \begin{perm}{1&2&\cdots&n}{\sigma(1)&\sigma(2)&\cdots&\sigma(n)}
  \end{perm}.$$ This is called the 
  \emph{two-line notation} for a permutation.
\end{defn}

Read down the columns to see images of elements from the top row
sitting in the corresponding position in the bottom row. In other
words, if $\alpha$ maps $i$ to $j$ then we put $i$ over
$j$ in the $i$th column of the two-line notation. 

\begin{example}
$\alpha = \begin{perm}{1&2&3&4&5}{4&1&3&5&2} \end{perm}$
is the permutation which maps $1\to 4$, $2\to 1$, $3\to 3$, $4\to 5$,
and $5\to 2$.  Note that $3$ is a {\em fixed point} for $\alpha$. 
\end{example}

\begin{defn}[permutation diagrams]\index{permutation diagram}
Permutations can also be expressed by diagrams. A diagram is a
directed graph with $2n$ vertices and $n$ edges, with the vertices
arranged in two rows of $n$ each, such that each edge connects a
single vertex in the top row to a single vertex in the bottom row.
The vertices on the top and bottom rows are numbered 1 to $n$ in order
from left to right. Then the diagram depicts the permutation $\alpha$
that maps $i$ to $j$ if and only if there is an edge connecting vertex
$i$ in the top row with vertex $j$ in the bottom row.
\end{defn}

\begin{example}
We can express the permutation $\alpha$ in the previous example by the
diagram
\[ 
\alpha =\begin{perm}{1&2&3&4&5}{4&1&3&5&2} \end{perm} =
\begin{pdiag}{5}{1}
  \pdmap{1}{4}\pdmap{2}{1}\pdmap{4}{5}\pdmap{5}{2} \pdendmapfill
\end{pdiag}.
\]
Technically, the edges should all have arrows pointing down, but since
all the arrows point in the same direction, it is customary to omit
them.  You should be able to \emph{read} the diagram as a bijection by
reading the edges from top to bottom as defining images of each
numbered vertex.
\end{example}




\begin{defn}[multiplication of permutations]
Given two permutations, say $\alpha, \beta \in \Sym_n$, we can get a new
permutation $\alpha \compose \beta \in \Sym_n$ by the usual composition
of functions. \emph{Convention:} We usually simplify notation and
write the composite $\alpha \compose \beta$ as the ``product'' $\alpha
\beta$. \emph{With this convention, we have to read such products as
  composites.}
\end{defn}

Thus, $\alpha \compose \beta$ is the bijection defined by the rule
$(\alpha \compose \beta)(j) = \alpha(\beta(j))$ for all $j \in \n$. In
terms of the above convention, this reads as $(\alpha \beta)(j) =
\alpha(\beta(j))$.

We know that $\alpha \beta = \alpha \compose \beta$ is another
permutation because the composite of two bijections is always a
bijection.

Let me remind you that composition of functions is not always
commutative. That is, if $f, g$ are functions then it can happen that
$f \compose g \ne g \compose f$. Since permutations are functions,
this also applies to permutations.  So for permutations $\alpha,
\beta \in \Sym_n$ it can happen that $\alpha\beta \ne \beta\alpha$.



\begin{example}
If $\alpha = \begin{perm}{1&2&3&4&5}{2&4&1&5&3}\end{perm}$ and $\beta
= \begin{perm}{1&2&3&4&5}{3&1&4&5&2}\end{perm}$ are defined in terms
of the two-line notation then we have
\[
\alpha \beta  =  
\begin{perm}{1&2&3&4&5}{2&4&1&5&3}\end{perm} 
\begin{perm}{1&2&3&4&5}{3&1&4&5&2}\end{perm} =
\begin{perm}{1&2&3&4&5}{1&2&5&3&4}\end{perm}.
\] 
On the other hand, if we do the product the other way around we
obtain
\[
\beta \alpha = 
\begin{perm}{1&2&3&4&5}{3&1&4&5&2}\end{perm} 
\begin{perm}{1&2&3&4&5}{2&4&1&5&3}\end{perm} = 
\begin{perm}{1&2&3&4&5}{1&5&3&2&4}\end{perm}.
\]
You should check these results yourself to make sure that you
understand the example.  Notice that $\alpha \beta \ne \beta \alpha$
in this example.
\end{example}

In particular, you need to read $\alpha \beta = \alpha \compose \beta$
as $\alpha$ of $\beta$, which is the same as $\beta$ followed by
$\alpha$. In a composition $\alpha \compose \beta$, which is defined
by $(\alpha \compose \beta)(j) = \alpha(\beta(j))$, the \emph{second}
function is the \emph{first} to be applied to the
argument.\footnote{This confusing order reversal is a consequence of
  the fact that functions are normally written on the left of their
  argument. It can be avoided by deciding instead to write functions
  on the right of their argument.}

\begin{example}
Multiplication of permutations can also be calculated using
permutation diagrams. For instance,
\[
\begin{pdiag}{5}{2}
  \pdname{\alpha}\pdmap{1}{2}\pdmap{2}{4}\pdmap{3}{1}\pdmap{4}{5}\pdmap{5}{3}
  \pdendmapfill
  \pdname{\beta}\pdmap{1}{3}\pdmap{2}{1}\pdmap{3}{4}\pdmap{4}{5}\pdmap{5}{2}
  \pdendmapfill
\end{pdiag}
=
\begin{pdiag}{5}{1}
  \pdmap{1}{1}\pdmap{2}{5}\pdmap{3}{3}\pdmap{4}{2}\pdmap{5}{4}
  \pdendmapfill
\end{pdiag}
= \beta \alpha 
\]
computes $\beta \alpha$ in terms of the diagram, by reading the edges
all the way from top to bottom in the joined diagram displayed on the
left above. Again, because we are computing $\beta \alpha = \beta
\compose \alpha$, we are computing $\alpha$ followed by $\beta$, so we
must put the diagram of $\alpha$ above the diagram of $\beta$.
\end{example}





\begin{defn}[identity permutation]
  The simplest permutation in $\Sym_n$ is the {\em identity} permutation,
  which is just the identity mapping $id_X$ on the set $X=\n$.  We will
  often write $id$ for this permutation.
\end{defn}

\begin{example}
For instance, if $n=5$ we have $id
= \begin{perm}{1&2&3&4&5}{1&2&3&4&5}\end{perm}$. The diagram of this
permutation is 
\[ id = 
\begin{pdiag}{5}{1}
  \pdmap{1}{1}\pdmap{2}{2}\pdmap{3}{3}\pdmap{4}{4}\pdmap{5}{5}
  \pdendmapfill
\end{pdiag}.
\]
\end{example}


\begin{defn}[inverse permutation]\label{def:inverse-perm}
\index{inverse of a permutation}%
  Since a permutation $\alpha$ is a bijection, it is always an
  invertible mapping.  Thus $\alpha^{-1}$ exists. It is defined by the
  property $\alpha^{-1} \alpha = id = \alpha \alpha^{-1}$. 
\end{defn}

\begin{example}
If $\alpha =
\begin{perm}{1&2&3&4&5}{4&1&3&5&2}\end{perm}$
in the two-line notation then $\alpha^{-1} =
\begin{perm}{1&2&3&4&5}{2&5&3&1&4}\end{perm}$.
Note that $\alpha \alpha^{-1} = id = \alpha^{-1}\alpha$.  
\end{example}

In terms of diagrams, the diagram of $\alpha^{-1}$ is obtained by
turning the diagram of $\alpha$ upside down (and reversing the
direction of its arrows).

\begin{thm}[properties of permutation multiplication]
  Let $\alpha, \beta, \gamma \in \Sym_n$. Then:
  \begin{enumerate}
  \item multiplication is associative: $(\alpha\beta)\gamma = \alpha
    (\beta\gamma)$,
  \item the identity is neutral for multiplication: $\alpha\, id = id\,
    \alpha = \alpha$,
  \item inverses exist: $\alpha^{-1} \in \Sym_n$ exists such that
    $\alpha^{-1} \alpha = id = \alpha \alpha^{-1}$.
  \end{enumerate}
\end{thm}

\begin{proof}
(a) It is well known (and easy to check) that composition of functions
  is associative. Since permutations are functions, composition of
  permutations is associative.

(b) This is obvious. 

(c) We have already discussed this, in \ref{def:inverse-perm}.
\end{proof}

Another important property of permutation multiplication is: \emph{the
  inverse of a product is the product of the inverses taken in reverse
  order:} $(\alpha\beta)^{-1} = \beta^{-1}\alpha^{-1}$. The proof is
an exercise.


\begin{defn}[cycles and cycle notation]\index{cycle}
  A permutation in $\Sym_n$ which maps $i_1 \to i_2 \to \cdots \to
  i_{r-1} \to i_r \to i_1$ and which fixes all other numbers in the
  set $\n$ is called an $r$-cycle. In the \emph{cycle notation} it is
  written as $(i_1, i_2, \dots, i_r)$.
\end{defn}

\begin{example}
  The permutation $\alpha =
  \begin{perm}{1&2&3&4&5}{1&5&3&2&4}\end{perm}$
  is a $3$-cycle in $\Sym_5$. In the {\em cycle notation} we write this
  permutation as $\alpha = (2,5,4)$.
\end{example}

Note that the cycle notation is ambiguous unless $n$ is
specified. Also, there is more than one way to write a cycle in the
cycle notation, since for instance $(2,5,4) = (5,4,2) = (4,2,5)$ are
all the same 3-cycle! Despite these deficiencies, the cycle notation
is extremely useful. 

\begin{rmk}\index{identity permutation}
  The identity permutation $id$ in $\Sym_n$ is (by standard
  convention) often written as the 1-cycle $(1)$. From now on we will
  usually write $(1)$ for the identity permutation.
\end{rmk}


The inverse of a cycle is also a cycle of the same length: it is the
cycle obtained by writing the numbers of the original cycle in reverse
order. For instance, the inverse of $(2,5,4)$ is the cycle $(4,5,2)$. 

\begin{defn}\index{transposition}
A $2$-cycle is also known as a {\em transposition} or \emph{swap}. It
simply interchanges two numbers and fixes all others. Every
transposition is its own inverse.
\end{defn} 

\begin{defn}
Two given cycles are said to be {\em disjoint} if they have no numbers
in common. 
\end{defn} 

\begin{example}
The cycles $(2,5,4)$ and $(3,7,1,9)$ are disjoint, while $(2,5,4)$ and
$(3,5)$ are not.
\end{example}



\begin{thm}[disjoint cycle factorization] \label{B:disj}
\index{disjoint cycle factorization}
  Any permutation can be written as a product of disjoint
  cycles. Moreover, disjoint cycles commute with one another, so the
  product of disjoint cycles can be taken in any order.
\end{thm}

\begin{proof}
If all numbers are fixed by the permutation, then it is identity, and
can be expressed as a product of disjoint 1-cycles. Otherwise, let
$i_1$ be the first number which is not fixed. It then maps to another
number, say $i_2$, and so on. Because a permutation is a bijection on
a \emph{finite} set, eventually we must reach a number $i_r$ which is
mapped back to $i_1$, so we obtain an $r$-cycle $(i_1,i_2,\dots,
i_r)$. Now continue the argument with the next number which is not
fixed by the permutation, and which has not already appeared in some
cycle. This process must terminate after finitely many steps since we
are permuting finitely many elements. This proves the first claim. The
second claim is obvious.
\end{proof}


\begin{example}
The above proof is constructive, and highly computational, in that it
provides a \emph{procedure} for computing such a product of disjoint
cycles for any given permutation. Here is an illustrative example. Let
$\alpha = \begin{perm} {1&2&3&4&5&6&7&8&9}
  {2&5&3&1&8&9&4&7&6}\end{perm}$.  Then $\alpha$ sends $1\to 2 \to 5
\to 8 \to 7 \to 4 \to 1$, which gives the 6-cycle $(1,2,5,8,7,4)$. The
next number that doesn't appear in this cycle is 3, which is a fixed
point. The next after that is 6, and $\alpha$ sends $6 \to 9 \to 6$,
which gives a 2-cycle $(6,9)$. At this point every number not fixed by
$\alpha$ appears in a cycle, so we are finished.  The permutation
$\alpha$ has the cycle factorization $\alpha = (1,2,5,8,7,4)(6,9)
=(6,9)(1,2,5,8,7,4)$. Some people might include the 1-cycle (3) in
order to emphasize the fact that 3 is fixed, writing $\alpha =
(1,2,5,8,7,4)(6,9)(3)$. This is fine, too.
\end{example}

It is a fundamental fact that every permutation is expressible as a
product of transpositions (not necessarily disjoint). One way to prove
this uses the following observation.

\begin{lem}\label{B:trans} 
  Any $r$-cycle can be written as a product of (not necessarily
  disjoint) transpositions.
\end{lem}

\begin{proof} 
One verifies the identity $(i_1, i_2, \dots, i_r) = (i_1,i_2)(i_2,i_3)
\cdots (i_{r-1},i_r)$ by direct calculation.
\end{proof}

\begin{examples}
1. $(3,4,5) = (3,4)(4,5)$.

2. $(3,6,4,2) = (3,6)(6,4)(4,2)$.  
\end{examples}

It should also be noted that there is \emph{always} more than one way
to express a given permutation as a product of transpositions. If
$\sigma$ is any permutation and $\tau$ any transposition then $\sigma
= \sigma\tau^2$ (because $\tau^2 = id$). Thus, if $\sigma$ is factored
as a product of transpositions, then by tacking on two additional
factors of $\tau$ you have found another such factorization of
$\sigma$.

Furthermore, there are \emph{other} ways of factoring into
transpositions that do not arise from the formula in the lemma. For
instance, check that $(3,4,5) = (4,5)(5,3)$.

\begin{thm} \label{B:transpositionthm} 
Any permutation can be written as a product of (not necessarily
disjoint) transpositions.
\end{thm}

\begin{proof}
Combine \ref{B:trans} with \ref{B:disj}.
\end{proof}

As already noted, there are always many different ways (in fact,
infinitely many) to factor a permutation as a product of
transpositions.



\section*{Exercises}

\begin{problems}
\item Consider the permutations $\alpha = 
  \begin{perm}
    {1&2&3&4&5&6}{3&1&5&6&2&4}
  \end{perm}$ and $\beta =
  \begin{perm}
    {1&2&3&4&5&6}{6&3&4&1&5&2}
  \end{perm}$.
\begin{enumerate}
\item Compute the products $\alpha\beta$ and $\beta\alpha$.

\item What permutation is $\alpha^{-1}$? 

\item Compute $\alpha^2$ and $\alpha^3$.

\item What is the smallest positive power of $\alpha$ which equals
  identity? (I.e., compute the order of $\alpha$.)  
\end{enumerate}

\item Compute the order $|\beta|$ of $\beta = \begin{perm}
    {1&2&3&4&5&6&7&8&9&10&11&12&13&14}{3&1&5&6&2&4&13&11&9&7&10&8&14&12}
  \end{perm}$. Justify your answer. 


\item 
\begin{enumerate}
\item Write out all 3-cycles in $\Sym_4$. How many are there?
\item How many 3-cycles are there in $\Sym_n$?
\item For any $r$, how many $r$-cycles are there in $\Sym_n$?
\end{enumerate}


\item 
\begin{enumerate}
\item Write the following permutation as a product of disjoint
  cycles:
  \[ \alpha =
  \begin{perm}
    {1&2&3&4&5&6&7&8&9}{6&1&7&5&4&2&8&9&3} 
  \end{perm}.
  \]

\item Now write $\alpha$ as a product of transpositions. 
\end{enumerate}

\item Show that if $\alpha = \alpha_1 \alpha_2 \cdots \alpha_k$ is a
  product of disjoint cycles, then $\alpha^t = \alpha_1^t \alpha_2^t
  \cdots \alpha_k^t$. Show by an example that this may fail for
  products of non-disjoint cycles.


\item (Maps on the right).\index{maps on the right} Suppose we decide
  to write functions on the \emph{right} of their argument instead of
  on the left.  (This is called \emph{postfix} notation in computer
  science.) This means that we write $(x)f$ instead of $f(x)$. One
  usually reads $(x)f$ as \emph{the image of $x$ under $f$}. In this
  notation, the definition of $f \compose g$ becomes $(x)(f \compose
  g) = ((x)f)g$. In other words, this fixes the annoying order
  reversal that we see when maps are written on the left.  Show that
  if we adopt this convention then:
  \begin{enumerate}
  \item $(1,2,3) = (3,2)(2,1)$.
  \item $(i_1, i_2, \dots, i_r) = (i_r, i_{r-1}) \cdots (i_3,i_2)(i_2,i_1)$.
  \item The product $\alpha \beta$ can be computed by placing the
    diagram of $\alpha$ above the diagram of $\beta$ and then reading
    the edges from the top of $\alpha$ to the bottom of $\beta$.
  \end{enumerate}

\end{problems}


\newpage\section{Permutation Groups}\noindent
We now come to our first examples of groups, the permutation
groups. These were first studied by Lagrange in the late 1700s.  Here
are the two key definitions.


\begin{defn}
  Let $G$ be a nonempty subset of $\Sym_n$. 

  (a) We say that $G$ is \emph{closed under products} if $\alpha, \beta
  \in G$ implies $\alpha \beta \in G$.

  (b) We say that $G$ is \emph{closed under inverses} if $\alpha \in
  G$ implies $\alpha^{-1} \in G$.
\end{defn}



\begin{defn}\index{permutation~group}
  A nonempty subset $G \subset \Sym_n$ is a \emph{group of
    permutations}, or \emph{permutation group} for short, if the set
  $G$ is closed under products and inverses.
\end{defn}

Notice that {\em any permutation group must contain the identity
  permutation}, since it contains some element $\alpha$ (because it is
nonempty) and contains $\alpha^{-1}$ (because of closure under
inverses) and thus must contain $\alpha \alpha^{-1} = id = (1)$ (by
closure under products).  Hence, if a given subset of $\Sym_n$ does
not include the identity permutation then it cannot be a permutation
group.

\begin{examples} 
1. The set $\Sym_n$ itself is a permutation group, called the
\emph{symmetric group}\index{symmetric group} on $n$ letters. 

2. The \emph{trivial group} is the group $\{ (1) \}$ consisting of
just the identity permutation. This group is not interesting; thus the
name.

3. The group $K = \{ (1), (1,2), (3,4), (1,2)(3,4) \}$ is another
example of a permutation group.
\end{examples}



\begin{defn}[order of a group]\index{order!of~a~group}
The {\em order} of a permutation group is the number of elements in
the group. In other words, the order of a group is its cardinality as
a set. If $G$ is a group, then we always write $|G|$ for its order.
\end{defn}


Note that permutation groups always have finite order since they are,
by definition, subsets of some $\Sym_n$, and $\Sym_n$ is itself a finite
set. We will see examples of groups of infinite order later.


\begin{defn}[powers of a permutation]
Let $\alpha \in \Sym_n$ be a permutation and let $m\in \Z$ be a positive
integer. We define:
\begin{align*} 
\alpha^m &= \alpha \alpha \cdots \alpha \quad (\text{$m$ factors})\\
\alpha^0 &= (1) = id \\
\alpha^{-m} &= \alpha^{-1} \alpha^{-1}\cdots \alpha^{-1} \quad(\text{$m$ factors})
\end{align*}
\end{defn}

Notice that $(\alpha^m)^{-1} = \alpha^{-m}$. Furthermore, $\alpha^r
\alpha^s = \alpha^{r+s}$ and $(\alpha^r)^s = \alpha^{rs}$ for all $r,s
\in \Z$. So the usual laws of exponents (but only for integer
exponents) are applicable to powers of permutations.


\begin{lem}
  Let $\alpha \in \Sym_n$. Then there must be some positive integer
  $p$ such that $\alpha^p = id$.
\end{lem}


\begin{proof}
Since the set $\Sym_n$ is closed under products, the set $S =
\{\alpha, \alpha^2, \alpha^3, \dots \}$ of all positive integer powers
of $\alpha$ is contained in $\Sym_n$. So the set $S$ must be a
\emph{finite} set. This implies that there must be repetition in the
powers of $\alpha$. In other words, there exist distinct positive
integers $r,s$ such that $\alpha^r = \alpha^s$. We may assume that
$r>s$ (otherwise we can just interchange them). Then it follows that
\[
  \alpha^{r-s} = \alpha^r \alpha^{-s} = \alpha^s \alpha^{-s} =
  \alpha^0 = id.
\]
Since $r-s > 0$, this completes the proof of the lemma.
\end{proof}

\begin{defn}[order of a permutation]\index{order!of~an~element}
Let $\alpha \in \Sym_n$.  The {\em order} of $\alpha$ (written as
$\text{order}(\alpha)$ or $|\alpha|$) is the \underline{smallest}
positive integer $m$ such that $\alpha^m = id$.
\end{defn}

Note that only verifying the property $\alpha^m = id$ is \emph{not}
enough to prove that the order of $\alpha$ is $m$. You also need to
show that the positive exponent $m$ is minimal with respect to that
property. 

\begin{example}
  The order of the identity $id$ is the integer $1$: $|id| = 1$. 
\end{example}

\begin{example}
  If $\alpha = \begin{perm}{1&2&3&4&5}{4&1&3&5&2}\end{perm}$ then
  $|\alpha| = 4$, as you should verify by computing the successive
  powers $\alpha^2$, $\alpha^3$, and $\alpha^4$. Since $\alpha^1 \ne
  id$, $\alpha^2 \ne id$, $\alpha^3 \ne id$, but $\alpha^4 = id$, it
  follows that $4$ is the smallest positive exponent of $\alpha$
  giving the identity.
\end{example}

\begin{prop}
  Let $\alpha$ be a permutation. If $|\alpha|=m$ and $\alpha^k = id$
  for some positive integer $k$ then $k$ must be a multiple of $m$.
\end{prop}

\begin{proof}
Divide $k$ by $m$ to get an integer quotient $q$ and remainder $r$, so
that
\[
  k = qm+r \quad\text{and}\quad 0 \le r < m.
\]
Then $id = \alpha^k = \alpha^{qm+r} = (\alpha^m)^q \alpha^r = id\,
\alpha^r = \alpha^r$, so $\alpha^r = id$. If $r>0$ then we contradict
the fact that $|\alpha|=m$, so it follows that $r=0$ and thus $k = qm$
as required.
\end{proof}

Since any permutation can be written as a product of disjoint cycles,
the following result enables us to easily compute the order of any
permutation. Note that if $\alpha = \alpha_1 \alpha_2 \cdots \alpha_k$
is a product of disjoint cycles, then $\alpha^t = \alpha_1^t \alpha_2^t
\cdots \alpha_k^t$, because disjoint cycles commute.

\begin{prop} \label{B:order}
The order of any $r$-cycle is $r$.  The order of a product $\alpha_1
\alpha_2 \cdots \alpha_n$ of {\em disjoint} cycles is the least common
multiple (lcm) of their individual orders.
\end{prop}

\begin{proof}
Let $\alpha = (i_1, i_2, \dots, i_r)$ be any $r$-cycle. Then for any
$j < r$, the map $\alpha^j$ sends $i_1$ to $i_j$ and hence cannot be
equal to the identity $(1)$. On the other hand, it is easy to check
that $\alpha^r = (1)$. This proves the first claim.  The proof of the
second claim is left to you as an exercise.
\end{proof}

\begin{example}
  If $\alpha = (6,9,5)(2,7,3,10)(1,11)$ then $|\alpha| =
  \mathrm{lcm}(3,4,2) = 12$. This follows from the preceding result.
\end{example}


\begin{defn}[cyclic groups]\index{cyclic group}
  Let $\alpha\in \Sym_n$. The smallest permutation group 
  containing $\alpha$ is called the \emph{cyclic group} generated by
  $\alpha$. This group is often written as $\gen{\alpha}$.
\end{defn}

\begin{prop}
  Let $\alpha\in \Sym_n$, and let $G = \gen{\alpha}$ be the cyclic
  group generated by $\alpha$. Then $G = \{id, \alpha, \alpha^2,
  \dots, \alpha^{k-1} \}$, where $k = |\alpha|$.
\end{prop}

Note that the order of the cyclic group $G$ generated by $\alpha$ is
the same as the order of $\alpha$ itself: if $G = \gen{\alpha}$ is
cyclic then $|G| = |\alpha|$. This is a characteristic property of
cyclic groups. The notation $C_k$ is often used for a cyclic group of
order $k$.


\begin{prop}\label{prop:inverse}
Let $\alpha \in \Sym_n$. If $|\alpha| = m$ then $\alpha^{-1} =
\alpha^{m-1}$.
\end{prop}

\begin{proof}
$\alpha^{m-1} = \alpha^m \alpha^{-1} = (1) \alpha^{-1} = \alpha^{-1}$. 
\end{proof}

This implies the following useful fact.

\begin{thm}[closure under products suffices]
  Suppose that $G$ is a nonempty subset of $\Sym_n$. Then $G$ is a
  permutation group if and only if the set $G$ is closed under
  products.
\end{thm}

\begin{proof}
Suppose that $\alpha \in G$. Since $G$ is closed under products, it is
clear that $G$ must contain the subgroup $\gen{\alpha}$ generated by
$\alpha$. So by Proposition \ref{prop:inverse}, $\alpha^{-1} =
\alpha^{m-1} \in G$, where $m = |\alpha|$.
\end{proof}


\section*{Exercises}

\begin{problems}

\item Does there exist a permutation group of order $r$ for any given
  positive integer $r$? Justify your answer.

\item Compute the cyclic group $\gen{\alpha}$ generated by $\alpha$
  for $\alpha = (1,2,3,4)$ in $\Sym_4$. What is the order of the group?

\item Let $\alpha = (i_1, i_2, \dots, i_r)$ be an $r$-cycle. Write
  $\alpha^2$ as a product of disjoint cycles. [You may have to
  distinguish the cases where $r$ is odd or even.]

\item Prove the second claim in Proposition \ref{B:order}. 

\item Let $\alpha = (1,2)(3,4)$ and $\beta = (1,2,3,4)$ in
  $\Sym_4$. By definition, the group $G=\gen{\alpha, \beta}$
  \emph{generated} by $\alpha, \beta$ is the smallest permutation
  group containing both $\alpha, \beta$. Find and list all the
  elements of $G$. What is $|G|$?

\item Let $\alpha = (1,2)$ and $\beta = (1,2,3)$ in $\Sym_3$. Let
  $G=\gen{\alpha, \beta}$ be the group generated by $\alpha,
  \beta$. Show that $G = \Sym_3$. 


\item \label{ex-perm:adjacent} 
 \begin{enumerate}
  \item Show that $(1,3) = (2,3)(1,2)(2,3)$. 
  \item Show that $(1,4) = (3,4)(2,3)(1,2)(2,3)(3,4)$.
  \item Prove that for $j > 1$ we have $(1,j) =$
  $$(j-1,j)(j-2,j-1)\cdots(1,2)\cdots(j-2,j-1)(j-1,j).$$
  \item Prove that for $i < j$ we have $(i,j) =$
  $$(j-1,j)(j-2,j-1)\cdots(i,i+1)\cdots(j-2,j-1)(j-1,j).$$
 \end{enumerate}
 \noindent Part (d) shows that it is possible to write any
  transposition as a product of {\em adjacent} ones; i.e., ones of the
  form $(k,k+1)$.


\item \label{ex-perm:adjgen} Prove that $\Sym_n$ is generated by the
  set $\{ (1,2), (2,3), \dots, (n-1,n) \}$ of adjacent transpositions.
  [Hint: By Theorem \ref{B:transpositionthm} it is enough to show that
    any transposition is expressible as a product of the ones in the
    given set. Now use the result of Problem \ref{ex-perm:adjacent}.]

\item  \label{ex-perm:twogen} 
\begin{enumerate}
\item Show that if $\alpha = (1,2)$, $\beta = (1,2,\dots,n)$ then for
  any $1 < i <n$ we have $(i,i+1) = \beta^{i-1} \alpha
  (\beta^{i-1})^{-1} = \beta^{i-1} \alpha\beta^{n-i+1}$. 

\item  Prove that $\Sym_n$ is generated by the set $S = \{ (1,2),
(1,2,3,\dots,n) \}$. [Hint: Use part (a) and the result of the
  preceding exercise.]
\end{enumerate}



\end{problems}



\newpage\section{The sign of a permutation}\noindent
We have shown that any permutation can be factored as a product of
transpositions, but in infinitely many ways. It is time to investigate
this in greater detail.

\begin{thm}[the identity is even]
  Every factorization of the identity $id$ as a product of
  transpositions must use an even number of transpositions.
\end{thm}

This might seem obvious, but a careful proof is difficult. The proof
can be done by induction on the number of transpositions. We omit the
proof, and refer to the excellent article\footnote{Keith Conrad,
  \emph{The sign of a permutation},\\
  \texttt{http://www.math.uconn.edu/~kconrad/blurbs/grouptheory/sign.pdf}.}
by Keith Conrad (see the bibliography at the end).

\begin{cor}
  Let $\alpha \in \Sym_n$. Suppose that $\alpha = \sigma_1 \cdots
  \sigma_r$ and $\alpha = \tau_1 \cdots \tau_s$ are two ways of
  expressing $\alpha$ as a product of transpositions. Then the
  difference $r-s$ must be an even number.
\end{cor}

\begin{proof}
We have $id = \alpha \alpha^{-1} = \sigma_1 \cdots \sigma_r \tau_s
\cdots \tau_1$. By the previous theorem, $r+s$ is even. This implies
that $r-s$ must be even. (If $r-s$ is odd then $2r = (r+s) + (r-s)$
would be odd, which is absurd.)
\end{proof}

The corollary implies that if we find one way to factor $\alpha \in
\Sym_n$ as a product of an odd number of transpositions, then all ways
of expressing $\alpha$ as a product of transpositions uses an odd
number of them. The same holds if we replace ``odd'' by ``even.'' This
means that the following definition makes sense.


\begin{defn}\index{even permutation}\index{odd permutation}
  Let $\alpha \in \Sym_n$. We say that $\alpha$ is an \emph{odd
    permutation} if there is some way to express $\alpha$ as a product
  of an odd number of transpositions.  We say that $\alpha$ is an
  \emph{even permutation} if there is some way to express $\alpha$ as
  a product of an even number of transpositions. The \emph{sign} of
  $\alpha$ is\index{sign of a permutation} defined to be
  \[
    \sgn(\alpha) = 
    \begin{cases}
      -1 & \text{ if $\alpha$ is odd}\\
      1 & \text{ if $\alpha$ is even}.
    \end{cases}
  \]
\end{defn}


\begin{examples}
1. The sign of $id$ is $1$. (The identity is even.)  

2. The sign of any transposition is $-1$. (A transposition is odd.)

3. The sign of $(1,2,3) = (1,2)(2,3)$ is $1$. The $3$-cycle $(1,2,3)$
is even.  

4. The sign of any $r$-cycle is $(-1)^{r-1}$. The proof is an
exercise.
\end{examples}

\begin{thm}[sign is multiplicative]\label{thm:sign-is-mult}
  For any $\alpha, \beta \in \Sym_n$ we have $\sgn(\alpha \beta) =
  \sgn(\alpha) \sgn(\beta)$.
\end{thm}

The proof is an easy exercise. 

\begin{defn}\index{A@$\Alt_n$}\index{alternating group}
  For any $n \ge 2$, the \emph{alternating group} $\Alt_n$ is the set
  of all even permutations in $\Sym_n$.
\end{defn}


It remains to verify that the definition makes sense. To do so, we
must show that the set $\Alt_n$ of even permutations is closed under
products. This is an exercise. Note that $\Alt_1$ is the empty set.


\begin{examples}
  1. $\Alt_2 = \{ (1) \}$. So $|\Alt_2| = 1$.

  2. $\Alt_3 = \{ (1), (1,2,3), (3,2,1) \}$. This is the same as the
  cyclic group generated by $(1,2,3)$. So $|\Alt_3| = 3$. 

  3. $\Alt_4 = \{ (1), (1,2,3), (1,2,4), (1,3,4), (2,3,4), (3,2,1),
  (4,2,1), (4,3,1), \\(4,3,2),  (1,2)(3,4), (1,3)(2,4), (1,4)(2,3)\}$. So
  $|\Alt_4| = 12$.

  3. In general, for any $n \ge 2$, we have $|\Alt_n| = n!/2$. 
\end{examples}

\begin{prop}
  $|\Alt_n| = n!/2$ for any $n \ge 2$. 
\end{prop}

\begin{proof}
Let $\beta$ be any transposition, say $\beta = (1,2)$ for instance.
Holding $\beta$ fixed, it is easy to check that the mapping $f: \Alt_n
\to \Sym_n$ defined by $f(\alpha) = \alpha\beta$ is a bijection
between the disjoint sets $\Alt_n$ and $\Sym_n - \Alt_n$. Hence
$|\Alt_n| = |\Sym_n-\Alt_n|$. If we write $k = |\Alt_n|$ then the fact
that $\Sym_n = \Alt_n \cup (\Sym_n - \Alt_n)$ along with the
disjointness of the two sets implies that $|\Sym_n| = |\Alt_n| +
|\Sym_n - \Alt_n|$. In other words, $n! = k+k$, so $k = n!/2$, as
required.
\end{proof}


\begin{rmk}
  It can be shown that $\sgn(\alpha) = (-1)^{I(\alpha)}$, where
  $I(\alpha)$ is the number of \emph{inversions}\index{inversions} in
  $\alpha$. By definition, an inversion occurs when $i < j$ but
  $\alpha(i) > \alpha(j)$. You can count the number of inversions in
  $\alpha$ by drawing the diagram of $\alpha$ and counting the number
  of edge crossings.
\end{rmk}


\begin{rmk} 
One important application of permutations is the following closed
formula for the determinant\index{determinant} of an $n \times n$
matrix $A = (a_{ij})$:
\[
\det A = \sum_{\alpha \in \Sym_n} \sgn(\alpha)\, a_{1,\alpha(1)}
a_{2,\alpha(2)} \cdots a_{n,\alpha(n)}. 
\]
This formula says that to compute the determinant one must form a
product of entries chosen one from each row and column, and then take
the signed sum of all such products. 
\end{rmk}



\section*{Exercises}



\begin{problems}

\item Show the inverse of a permutation must have the same sign as the
  permutation.

\item Prove that for any $n \ge 2$ the set $\Alt_n$ of all even
  permutations is a group.

\item Let $S$ be the set of all odd permutations in some fixed
  $\Sym_n$. Is $S$ a permutation group? Why or why not?

\item Prove that for $n \ge 2$ the order of the alternating group
  $\Alt_n$ is $n!/2$.  [Hint: Establish a bijection $f$ from $\Alt_n$
  onto the set $B_n$ of all odd permutations in $\Sym_n$, by choosing
  any transposition $\tau$ and setting $f(\alpha) = \tau\alpha$. Show
  this is a bijection, and conclude that $|\Alt_n| = |B_n|$.]

\item Prove Theorem \ref{thm:sign-is-mult}.




\item Find a set of generators for $\Alt_n$, for $n \ge 3$.
\end{problems}

\end{document}

