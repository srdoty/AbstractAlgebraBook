\documentclass[11pt]{article}
\usepackage[nohead,margin=1.50in]{geometry} %set margins
\usepackage{amsmath,amssymb,amsthm,pdiag} %AMS packages for math stuff
\usepackage{multicol} % for use in HW section
\usepackage{enumitem}
  \setlist{topsep=1pt,itemsep=0pt,parsep=1pt}
  \setenumerate[1]{label=(\alph*)}
\usepackage[all]{xy}


\newenvironment{problems}
{
 \begin{enumerate}[topsep=1pt,itemsep=0pt,parsep=2pt,leftmargin=0.6cm,%
 label={\arabic*.}, ref=\arabic*] \small
}
{
 \end{enumerate}
}

%%% Define some theorem and example environments. The starred versions
%%% are un-numbered and the unstarred versions are numbered.
\newtheoremstyle{plain}
  {\topsep}   % ABOVESPACE
  {\topsep}   % BELOWSPACE
  {\slshape}  % BODYFONT
  {0pt}       % INDENT (empty value is the same as 0pt)
  {\bfseries} % HEADFONT
  {.}         % HEADPUNCT
  {5pt plus 1pt minus 1pt} % HEADSPACE
  {}          % CUSTOM-HEAD-SPEC

\swapnumbers
\newtheorem{thm}{Theorem}[section]
\newtheorem{lem}[thm]{Lemma}
\newtheorem{prop}[thm]{Proposition}
\newtheorem{cor}[thm]{Corollary}
\newtheorem*{thm*}{Theorem}
\newtheorem*{lem*}{Lemma}
\newtheorem*{prop*}{Proposition}
\newtheorem*{cor*}{Corollary}

\theoremstyle{definition}
\newtheorem{defn}[thm]{Definition}
\newtheorem{example}[thm]{Example}
\newtheorem{examples}[thm]{Examples}
\newtheorem{rmk}[thm]{Remark}
\newtheorem{rmks}[thm]{Remarks}
\newtheorem{conv}[thm]{Convention}
\newtheorem*{defn*}{Definition}
\newtheorem*{example*}{Example}
\newtheorem*{examples*}{Examples}
\newtheorem*{rmk*}{Remark}
\newtheorem{rmks*}{Remarks}
\newtheorem*{conv*}{Convention}


%%% Define some convenient abbreviations for common mathematical
%%% notations.
\newcommand{\R}{\mathbb{R}} % use \R for the real numbers
\newcommand{\C}{\mathbb{C}} % use \C for the complex numbers
\newcommand{\Z}{\mathbb{Z}} % use \Z for the integers
\newcommand{\Q}{\mathbb{Q}} % use \Q for the rationals
\newcommand{\N}{\mathbb{N}} % use \N for the natural numbers
\newcommand{\F}{{\mathbb F}}
\newcommand{\compose}{\circ} % functional composition
\newcommand{\gen}[1]{\langle #1 \rangle}
\newcommand{\End}{\operatorname{End}}
\newcommand{\GL}{\mathrm{GL}}
\newcommand{\SL}{\mathrm{SL}}
\newcommand{\PSL}{\mathrm{PSL}}
\renewcommand{\O}{\mathrm{O}}
\newcommand{\SO}{\mathrm{SO}}
\newcommand{\U}{\mathrm{U}}
\newcommand{\SU}{\mathrm{SU}}
\newcommand{\g}{\mathfrak{g}}
\newcommand{\transpose}{\mathsf{T}}
\newcommand{\B}{\mathcal{B}}
\newcommand{\Rep}{\operatorname{Rep}}
\newcommand{\Mat}{\operatorname{Mat}}
\newcommand{\inner}[2]{\langle #1, #2 \rangle}
\newcommand{\sgn}{\operatorname{sgn}}
\newcommand{\n}{\underline{\mathbf{n}}}
\newcommand{\Sym}{\mathbb{S}}
\newcommand{\Alt}{\mathbb{A}}
\newcommand{\D}{\mathbb{D}}
\newenvironment{perm}[2]{\left(\begin{smallmatrix}#1 \\ #2}{\end{smallmatrix}\right)}
\newcommand{\lcm}{\operatorname{lcm}}
\newcommand{\res}{\operatorname{res}}
\newcommand{\normal}{\triangleleft\,}%better than \lhd
\newcommand{\morenormal}{\triangleright}
\newcommand{\im}{\operatorname{im}}

\allowdisplaybreaks
\parskip=2pt

%\title{Document Title}
%\author{author's name}

\begin{document}%\maketitle


%\appendix
\setcounter{section}{18}
%\renewcommand{\thesection}{\Alph{section}}
\section{Cyclic groups}\noindent
Cyclic groups are the simplest groups to understand, and they appear
as subgroups of any group. We collect their main properties here in one
place, for ease of reference.

Recall that a group is called \emph{cyclic}\index{cyclic~group} if it
is generated by a single element. In multiplicative notation, if $x$
is a generator, then the cyclic group generated by $x$ is the set
\[
  \gen{x} = \{ x^k : k \in \Z \}
\]
of all integer powers of $x$. This could be a cyclic subgroup in some
larger group, or an abstract cyclic group. There are two cases to be
analyzed: either the generator $x$ has infinite order, or not.

If the generator $x$ has infinite order (i.e., $x^k \ne 1$ for all
positive integers $k$) then all the integer powers of $x$ must be
distinct, because if $x^j = x^k$ for $j \ne k$ then
$x^{j-k} = x^{k-j} = 1$, contradicting the assumption that $x$ has
infinite order.  Thus the group $\gen{x}$ is infinite. In that case,
we claim that it is isomorphic to the additive group $\Z$ of
integers. An isomorphism is defined by the rule $f(k) = x^k$. This is
a bijection, with inverse $g$ defined by $g(x^k) = k$; you can easily
check that $f(g(x^k)) = x^k$ and $g(f(k))=k$ for all $k$.  Since $f$
goes from an additive group to a multiplicative one, we have to check
that $f(j+k) = f(j)f(k)$, which is true since $x^{j+k} = x^j x^k$.
Since $f$ is a bijection and $f(j+k) = f(j)f(k)$ for all $j,k \in \Z$,
it follows that $f$ is an isomorphism, as claimed.

The remaining possibility is that $x$ has finite order, say $x$ has
order $n$ for some positive integer $n$. Then the set of powers
\[
  \gen{x} = \{ x^k : k \in \Z \} = \{ x^k : k = 0, 1, \dots, n-1 \}
\]
collapses to a \emph{finite} set since $x^n = 1$. It is customary to
denote this finite group by $C_n$. People often write
$C_n = \gen{x: x^n = 1}$ to indicate that $C_n$ is generated by an
element $x$ satisfying the relation $x^n = 1$. The relation $x^n = 1$
implies that $x^j = x^k$ in $C_n$ if and only if
$j \equiv k \pmod{n}$.  Thus, powers of $x$ are multiplied in the
group $C_n$ by adding their exponents modulo $n$. That is, we have
\[
   x^a x^b = x^c \text{ in } C_n \text{ where } c = \res_n(a+b).
\]
Recall that $[a] + [b] = [a+b]$ in the additive group
$\Z_n$. Equivalently, $[a] + [b] = [c]$ in $\Z_n$ where
$c = \res_n(a+b)$ is the residue modulo $n$ of $a+b$. So the bijection
$f: \Z_n \to C_n$ defined by the rule $f([k]) = x^k$ is an
isomorphism, because $f([j]+[k]) = f([j]) f([k])$ for all
$k = 0, 1, \dots, n-1$.

To summarize, we have proved the following important result. 

\begin{thm}\index{classification!of~cyclic~groups}
  Any infinite cyclic group is isomorphic to the additive group $\Z$
  of integers. Any finite cyclic group is isomorphic to the additive
  group $\Z_n$ of integers modulo $n$, for some positive integer $n$.
\end{thm}

Since isomorphism is transitive, this means also that all infinite
cyclic groups are isomorphic, and all finite cyclic groups of the same
order are isomorphic. So we now understand all cyclic groups, up to
isomorphism. 

The above theorem gives important information about all groups,
because if $G$ is any group and $x \in G$ then $H = \gen{x}$ is a
cyclic subgroup of $G$, hence is isomorphic to $\Z$ or to some $\Z_n$.
Furthermore, any result proved about cyclic groups applies equally
well to the cyclic subgroups found in any group.

\begin{thm}\label{thm:order-x^k}\index{order!of~a~power}
  Let $x$ be an element of a group $G$. If $x$ has order $n$ then
  $x^k$ has order $n/g$, where $g = \gcd(n,k)$\index{gcd}. If $x$ has
  infinite order then $x^k$ has infinite order.
\end{thm}

\begin{proof}
Suppose $|x|=n$. Because $\gen{x}$ is isomorphic to the additive group
$\Z_n$, with $x^k$ corresponding to $[k]$, it suffices to show that
the order of $[k]$ in $\Z_n$ is $n/g$. By definition of order, the
order of $[k]$ is the least positive integer $m$ such that $m[k] =
[0]$; i.e., the least positive integer $m$ such that $[mk] = [0]$. If
$k>0$ then $mk$ must be the least common multiple of $n,k$: $mk =
\lcm(n,k)$. Hence $m = \lcm(n,k)/k$. Now a theorem from basic number
theory says that $nk = \gcd(n,k) \lcm(n,k)$, so $m = \lcm(n,k)/k =
n/\gcd(n,k)$, and the proof is finished in case $k>0$. If $k=0$ then
the result is trivial since $g=n$ and the identity has order 1. If
$k<0$ then we can use the fact that the order of $x$ is the same as
the order of $x^{-1}$.  This implies that the order of $x^k$ is the
same as the order of $x^{-k}$. Since $\gcd(n,k) = \gcd(n,-k)$, the
stated formula works in the negative case as well.

Finally, if $x$ has infinite order then so does $x^k$, because
assuming that $x^k$ has finite order leads immediately to a
contradiction.
\end{proof}

Recall that two integers $k,n$ are said to be \emph{relatively prime}
if $\gcd(n,k) = 1$.  The Euler phi
function\index{Euler's~phi-function} $\varphi(n)$ is defined to be the
number of integers $k$ in the range $1 \le k \le n-1$ such that $k$ is
relatively prime to $n$. It is easily proved that:
\begin{enumerate}[label=(\roman*)]
\item If $m,n$ are relatively prime then $\varphi(mn) =
  \varphi(m)\varphi(n)$.
\item If $p$ is prime then $\varphi(p^t) = p^t - p^{t-1}$.
\end{enumerate}
These two properties can be used to calculate $\varphi(n)$ whenever we
can find the prime factorization of $n$.

The next result follows easily from the previous theorem.

\begin{cor}\label{cor:orders}
  \begin{enumerate}
  \item The order of any element of $C_n$ divides $n$. 
  \item If $x \in C_n$ has order $n$ then
    $\gen{x^j} = \gen{x^k} \iff \gcd(n,j) = \gcd(n,k)$ and
    $|x^j| = |x^k| \iff \gcd(n,j) = \gcd(n,k)$.
  \item If $x \in C_n$ has order $n$ then $x^k$ generates $C_n$ if and
    only if $\gcd(n,k) = 1$. So $C_n$ has $\varphi(n)$ distinct
    generators.
  \item $[k]$ in $\Z_n$ generates $\Z_n$ if and only if $\gcd(n,k) =
    1$.  So the additive group $\Z_n$ has $\varphi(n)$ distinct
    generators.
  \end{enumerate}
\end{cor}

\begin{proof}
  This is left to you as an exercise.
\end{proof}

We also get information about the multiplicative groups $\Z_n^\times$
of units, whenever they are cyclic. 

\begin{cor}
  If the multiplicative group $\Z_n^\times$ is cyclic, then it is
  isomorphic to the additive group $\Z_{\varphi(n)}$ and it has
  $\varphi(\varphi(n))$ generators.
\end{cor}

\begin{proof}
We already proved that $\Z_n^\times = \{ [k] : \gcd(n,k) = 1 \}$, so
$|\Z_n^\times| = \varphi(n)$. If it is cyclic then it must be
isomorphic to $\Z_{\varphi(n)}$ by the previous theorem. Part (c) of
the preceding corollary says that there are $\varphi(\varphi(n))$
generators.
\end{proof}

Of course, this result raises the question: for which values of $n$ is
the multiplicative group $\Z_n^\times$ cyclic? Note that $\Z_n^\times$
is cyclic if and only if an element of order $\varphi(n)$ exists in
the group. Such elements are called primitive roots.

\begin{defn}\index{primitive~root~modulo~$n$}
  A congruence class $[a] \in \Z_n^\times$ is a \emph{primitive root}
  in $\Z_n^\times$ if it has order $\varphi(n)$; i.e., if it generates
  the group. Furthermore, an integer $a$ is called a \emph{primitive
    root modulo} $n$ if its residue class $[a]$ is a primitive root in
  $\Z_n^\times$.
\end{defn}

Wikipedia has a nice article on primitive roots in modular
arithmetic, for those who wish to know more.  The answer to our
question is provided by the following theorem from classical number
theory.

\begin{thm}[primitive roots theorem]\index{primitive~roots~theorem}
  There is a primitive root in the multiplicative group $\Z_n^\times$
  if and only if $n = 2$, $4$, $p^t$ or $2p^t$ where $p$ is an odd
  prime.
\end{thm}

We leave the proof, which is elementary but somewhat time-consuming,
to the number theory textbooks. Primitive roots are used in
cryptography, so the theorem has practical applications.


We now return to the study of cyclic groups in general. 

\begin{thm}
  Every subgroup of an infinite cyclic group is infinite cyclic. Every
  subgroup of a finite cyclic group is finite cyclic.    
\end{thm}

\begin{proof}
It suffices to prove the first statement for the additive cyclic group
$\Z$, since any infinite cyclic group is isomorphic to $\Z$. Let $G$
be any subgroup of $\Z$. If $G$ is the trivial subgroup $\{0\}$ then we
are done, as $\{0\} = \gen{0}$ is cyclic. Otherwise, $G$ must have at
least one positive element, so by the well-ordering principle of
natural numbers, the set of positive elements of $G$ has a least
member, say $k$. Then we claim that $G = \gen{k}$.  Clearly $G \supset
\gen{k}$ by closure, so it suffices to prove the reverse
inclusion. Let $m \in G$. By the division algorithm, there are unique
integers $q,r$ such that $m = qk+r$ and $0 \le r < k$.  Then $r = m -
qk \in G$ by closure, since $m,k \in G$. Since $k$ is the \emph{least}
positive integer in $G$, it follows that $r=0$. Hence $m=qk$ and thus
$m \in \gen{k}$.  This proves the reverse inclusion that establishes
the equality $G = \gen{k}$, which implies that $G = k\Z$ is the
subgroup consisting of all multiples of $k$. This is infinite cyclic.

It suffices to prove the second claim for the additive group $\Z_n$,
since any cyclic group of order $n$ is isomorphic to $\Z_n$. We can
use exactly the same argument as above to see that if $G$ is any
subgroup of $\Z_n$ then $G$ is either the trivial subgroup or $G =
\gen{[k]]}$, where $k$ is the least positive element of $G$, where we
  represent elements of $\Z_n$ by their residues $0,1, 2, \dots, n-1$.
\end{proof}


\begin{cor}
  Let $G$ be a finite cyclic group of order $n$. Then the order of any
  subgroup must divide $n$. Furthermore, $G$ has precisely one
  subgroup of order $k$ for every divisor $k$ of $n$.
\end{cor}

\begin{proof}
  If $H$ is a subgroup of $G$ then $H$ is cyclic by the previous
  theorem. Thus $H$ is generated by some power $x^k$ where $x$ is a
  generator of $G$. We proved in Theorem \ref{thm:order-x^k} that the
  order of $x^k$ is $n/g$ where $g = \gcd(n,k)$, so $|x^k|$ is a
  divisor of $n$. Since the order of $x^k$ is the same as the order of
  the subgroup it generates, the order of $H$ is a divisor of $n$. 

  If $k \mid n$ then $\gen{x^{n/k}}$ is a subgroup of order $k$, since
  $|x^{n/k}| = k$. Furthermore, this is the only subgroup of order $k$.
\end{proof}


\begin{cor}
  For any positive divisor $d$ of $n$, the number of elements of order
  $d$ in $C_n$ or $\Z_n$ is $\varphi(d)$.
\end{cor}

\begin{proof}
  Since $C_n \cong Z_n$, it suffices to prove this for $C_n$. By the
  previous corollary, there is precisely one subgroup of order $d$,
  say $\gen{y} \cong C_d$ for some $y \in C_n$, where $y$ has order
  $d$. Since this is the unique subgroup of order $d$, it must contain
  every element of $C_n$ of order $d$.  By Corollary
  \ref{cor:orders}(c), there are precisely $\phi(d)$ generators of
  $C_d$, and they are the elements of order $d$ in $C_n$.
\end{proof}

\begin{example}
  The number of elements of order $20$ in the cyclic group $\Z_{900}$
  is $\varphi(20) = \varphi(4 \cdot 5) = (4-2)(5-1) = 8$. The unique
  subgroup of order $20$ is the subgroup
  $\gen{[900/20]} = \gen{[45]}$.
\end{example}






\section*{Exercises}
\begin{problems}

\item Compute the following:
  \begin{enumerate}
  \item The number of generators of $\Z_{20}$, $\Z_{100}$, and
    $\Z_{1000}$.
  \item The number of generators of $C_{20}$, $C_{100}$, and
    $C_{1000}$.
  \item The order of $[235]$ in $\Z_{1000}$ and the order of $x^{235}$
    in the abstract cyclic group $C_{1000} = \gen{x: x^{1000} = 1}$.
  \item The number of elements of $\Z_{1000}$ or $C_{1000}$ of order
    $40$.
  \end{enumerate}

\item Compute the following:
  \begin{enumerate}
  \item The order of the multiplicative group $\Z_{250}^\times$.
  \item The number of generators of the multiplicative group
    $\Z_{250}^\times$.
  \item The number of elements of $\Z_{250}^\times$ of order $25$. 
  \end{enumerate}


\item Prove Corollary \ref{cor:orders}.

\item Compute the following:
  \begin{enumerate}
  \item The order of the multiplicative group $\F_{499}^\times$.
  \item The number of generators of the multiplicative group
    $\F_{499}^\times$.
  \item The number of elements of $\F_{499}^\times$ of order $41$. 
  \end{enumerate}

\item The fact that the multiplicative group $\F_p^\times$ of units in
  a finite field of $p$ elements (where $p$ is a prime) is always
  cyclic is of great importance in public-key cryptography. But
  actually finding a generator is sometimes difficult. Try to find a
  generator of the group $\F_{499}^\times$. You may wish to use a
  computer to aid your search.


\item The Elgamal cryptosystem\index{Elgamal~encryption} works as
  follows, in order to setup secure communication from Bob (or anyone)
  to Alice.\footnote{Cryptographic tradition demands that the two
    parties are named Alice and Bob. Also by tradition, the evil
    attacker trying to decrypt the secret messages is named Eve.}
  \begin{enumerate}
  \item First Alice chooses a very large prime $p$ and finds a
    generator $[g]$ of the cyclic group $\F_p^\times$. She chooses an
    integer $1 \le k \le p-1$ at random and computes $[h] = [g]^k$ in
    the multiplicative group $\F_p^\times$. She publishes the data
    $K_A = (p,g,h)$ as her \emph{public-key} and keeps the
    \emph{private-key} $k$ secret.\footnote{This is a one-way system,
      in the sense that it can be used only by Bob (or anyone) to send
      secret messages to Alice. If Bob wishes to receive secret
      messages, then he must setup his own public-key $K_B$ by
      following the same steps as Alice. After Bob publishes his
      public-key, Alice (or anyone) can use it to send secret messages
      to Bob.}

  \item To encrypt a secret message $[x] \in \F_{p}^\times$ to send to
    Alice, Bob chooses a random integer $1 \le m \le p-1$ and computes
    $c_1 = [g]^m$ and $c_2 = [x] \cdot [h]^m$ in the group
    $\F_p^\times$. The encrypted message that he sends to Alice is the
    pair $(c_1, c_2)$.

  \item When Alice receives the encrypted ciphertext message $(c_1,
    c_2)$, she computes the product $(c_1^k)^{-1} \cdot c_2$ in the
    group $\F_p^\times$, using her secret key $k$. Note that Alice
    knows about the extended Euclidean algorithm, so computing a
    modular inverse is no problem for her.
  \end{enumerate}
  Prove that this cryptosystem works; that is, prove that
  $(c_1^k)^{-1} \cdot c_2 = [x]$ in the multiplicative group
  $\F_p^\times$.


\item In order for an evil attacker Eve to break an Elgamal
  cryptosystem, she needs to solve the \emph{discrete logarithm
    problem}, which is the problem of finding an exponent $k$ such
  that $g^k = h$ in a cyclic group $C_n$, where $g$ is a generator of
  the group. Write $k = \log_g h$ to mean that $g^k = h$ in
  $C_n$. Note that the value $k = \log_g h$ is only defined modulo
  $n$. Put on your evil attacker hat, and find the following discrete
  logarithms:

  \begin{enumerate}
  \item $\log_x x^{27}$ in $C_{10} = \gen{x: x^{10} = 1}$.
  \item $\log_{x^9} x^{271}$ in $C_{50} = \gen{x: x^{50} = 1}$. 
  \item $\log_{[2]} [9]$ in $\Z_{11}^\times$. Use trial and error.
  \item $\log_{10} 37$ in $\Z_{47}^\times$. You may want to seek help
    from a computer.
  \end{enumerate}

  \textbf{Remark.} It is truly remarkable that cyclic groups can be
  used to construct a cryptosystem sufficiently secure that it is used
  worldwide for secure internet transmissions. It is believed that
  solving the discrete logarithm problem in the cyclic group
  $\F_p^\times$ is so hard a problem that even a supercomputer would
  take billions of years to finish, assuming that the prime $p$ is
  sufficiently\footnote{On the order of 1500 decimal digits for
    current technology.}  large. Unfortunately, this belief remains
  unproven.

\end{problems}

\renewcommand{\thesection}{\arabic{section}}
\end{document}
