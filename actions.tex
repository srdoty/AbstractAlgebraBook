\documentclass[11pt,oneside]{article}
\usepackage[nohead,margin=1.50in]{geometry} %set margins
\usepackage{amsmath,amssymb,amsthm,pdiag} %AMS packages for math stuff
\usepackage{multicol} % for use in HW section
\usepackage{enumitem}
  \setlist{topsep=1pt,itemsep=0pt,parsep=1pt}
  \setenumerate[1]{label=(\alph*)}

\newenvironment{problems}
{
 \begin{enumerate}[topsep=1pt,itemsep=0pt,parsep=2pt,leftmargin=0.6cm,%
 label={\arabic*.}, ref=\arabic*] \small
}
{
 \end{enumerate}
}

%%% Define some theorem and example environments. The starred versions
%%% are un-numbered and the unstarred versions are numbered.
\newtheoremstyle{plain}
  {\topsep}   % ABOVESPACE
  {\topsep}   % BELOWSPACE
  {\slshape}  % BODYFONT
  {0pt}       % INDENT (empty value is the same as 0pt)
  {\bfseries} % HEADFONT
  {.}         % HEADPUNCT
  {5pt plus 1pt minus 1pt} % HEADSPACE
  {}          % CUSTOM-HEAD-SPEC

\swapnumbers
\newtheorem{thm}{Theorem}[section]
\newtheorem{lem}[thm]{Lemma}
\newtheorem{prop}[thm]{Proposition}
\newtheorem{cor}[thm]{Corollary}
\newtheorem*{thm*}{Theorem}
\newtheorem*{lem*}{Lemma}
\newtheorem*{prop*}{Proposition}
\newtheorem*{cor*}{Corollary}

\theoremstyle{definition}
\newtheorem{defn}[thm]{Definition}
\newtheorem{example}[thm]{Example}
\newtheorem{examples}[thm]{Examples}
\newtheorem{rmk}[thm]{Remark}
\newtheorem{rmks}[thm]{Remarks}
\newtheorem{conv}[thm]{Convention}
\newtheorem*{defn*}{Definition}
\newtheorem*{example*}{Example}
\newtheorem*{examples*}{Examples}
\newtheorem*{rmk*}{Remark}
\newtheorem{rmks*}{Remarks}
\newtheorem*{conv*}{Convention}


%%% Define some convenient abbreviations for common mathematical
%%% notations.
\newcommand{\R}{\mathbb{R}} % use \R for the real numbers
\newcommand{\C}{\mathbb{C}} % use \C for the complex numbers
\newcommand{\Z}{\mathbb{Z}} % use \Z for the integers
\newcommand{\Q}{\mathbb{Q}} % use \Q for the rationals
\newcommand{\N}{\mathbb{N}} % use \N for the natural numbers
\newcommand{\F}{{\mathbb F}}
\newcommand{\compose}{\circ} % functional composition
\newcommand{\gen}[1]{\langle #1 \rangle}
\newcommand{\End}{\operatorname{End}}
\newcommand{\GL}{\mathrm{GL}}
\newcommand{\SL}{\mathrm{SL}}
\renewcommand{\O}{\mathrm{O}}
\newcommand{\SO}{\mathrm{SO}}
\newcommand{\U}{\mathrm{U}}
\newcommand{\SU}{\mathrm{SU}}
\newcommand{\g}{\mathfrak{g}}
\newcommand{\transpose}{\mathsf{T}}
\newcommand{\B}{\mathcal{B}}
\newcommand{\Rep}{\operatorname{Rep}}
\newcommand{\Mat}{\operatorname{Mat}}
\newcommand{\inner}[2]{\langle #1, #2 \rangle}
\newcommand{\sgn}{\operatorname{sgn}}
\newcommand{\n}{\underline{\mathbf{n}}}
\newcommand{\Sym}{\mathbb{S}}
\newcommand{\Alt}{\mathbb{A}}
\newcommand{\D}{\mathbb{D}}
\newenvironment{perm}[2]{\left(\begin{smallmatrix}#1 \\ #2}{\end{smallmatrix}\right)}
\newcommand{\lcm}{\operatorname{lcm}}
\newcommand{\res}{\operatorname{res}}
\newcommand{\im}{\operatorname{im}}

\allowdisplaybreaks
\parskip=2pt

%\title{Document Title}
%\author{author's name}

\begin{document}%\maketitle
\setcounter{section}{24}

\section{Group actions}\noindent
Group actions are useful for a variety of purposes. Not only do they
give us new information about groups themselves, but there are also
important applications to physics, chemistry, geometry and
combinatorics. 


\begin{defn}\index{group~action}\index{action~of~a~group}
Let $G$ be a group and $X$ a set. We say that $G$ \emph{acts on} the
set $X$ (on the left) if there is a mapping $G \times X \to X$,
written as $(g,x) \mapsto g\cdot x$, satisfying the properties
\[
  (gh)\cdot x = g\cdot(h\cdot x); \qquad  1\cdot x = x 
\]
for all $g,h \in G$ and all $x \in X$.  When $G$ acts on $X$, one
equivalently says that $X$ is a $G$-set.\index{G@$G$-set}
\end{defn}

As you might guess, if $G$ acts (on the left) on $X$ then people often
abbreviate $g\cdot x$ to $gx$.  It is also useful to consider right
actions. A \emph{right action} of a group $G$ on a set $X$ is a
mapping $X \times G \to X$, written as $(x,g) \mapsto x\cdot g$,
satisfying the properties
\[
  x\cdot (gh) = (x \cdot g)\cdot h); \qquad  x \cdot 1 = x 
\]
for all $g,h \in G$ and all $x \in X$. For the sake of definiteness,
all the actions that we will consider below are left actions, but all
the results proved for left actions have counterparts for right
actions.


\begin{examples}
  1. The dihedral group\index{dihedral~group} $\D_n$ acts on the set
  of vertices of a regular $n$-gon. It also acts on the set of edges
  of the regular $n$-gon.

2. Any matrix group $G$ of $n \times n$ matrices over a field $F$ acts
on the vector space $V = F^n$ by matrix multiplication; more
precisely, if $A \in G$ is a matrix in the group $G$ and $x$ is a
(column) vector in $F^n$ then the matrix product $Ax \in F^n$ gives
the action of $A$ on $x$.

3. In particular, the orthogonal group $\O(2)$ of $2 \times 2$
orthogonal matrices acts on the vector space $V=\R^2$ of points in the
Euclidean plane. Given a point $x \in \R^2$ and a matrix $A \in
\O(2)$, the action of $A$ on $x$ is given by $Ax \in \R^2$.
\end{examples}

Usually we think of a given $G$-set $X$ as a set of ``points'' so that
we can use geometric language. The action of $G$ moves points in $X$
to other points in $X$. Starting from a given point and observing
where it moves (as $G$ acts) determines the orbit of the point.

\begin{defn}[Orbit and stabilizer]
\index{orbit}\index{stabilizer}%
If $G$ is a group acting on a set $X$, the {\em orbit} of a point $x
\in X$ is the set $O_x = \{ gx \mid g\in G \}$.  Note that $O_x
\subset X$.  The {\em stabilizer} of the point $x$ is the subgroup
$G_x = \{g\in G \mid gx =x \}$.
\end{defn}

\begin{lem}\label{lem:stab-is-a-subgp}
  For any $G$-set $X$, the stabilizer $G_x$ of a point $x \in X$ is
  always a subgroup of $G$.
\end{lem}

The proof is an easy exercise for the reader. Note that we do
\emph{not} claim that $G_x$ is a normal subgroup. In fact, it is not
always normal.



\begin{prop} \label{eqrel}  
Let $G$ be a group acting on a set $X$. Consider the relation $\sim$
on $X$ defined by $x \sim y$ if and only if there exists some $g \in
G$ such that $y=gx$. Then $\sim$ is an equivalence relation on
$X$. The equivalence classes for $\sim$ are precisely the orbits. Thus
the action of $G$ partitions $X$ into a disjoint union of its distinct
orbits.
\end{prop}


Again, the proof is an easy exercise.


The following result could be called the \emph{fundamental theorem} of
group actions.\index{fundamental~theorem!of~group~actions}

\begin{thm}[The orbit-stabilizer theorem]
\index{orbit-stabilizer~theorem}\label{osr}% 
 Let $G$ be a group acting on a set $X$. Let $x \in X$. Then $|O_x| =
 [G:G_x]$. In words: the cardinality of the orbit of $x$ equals the
 index of the stabilizer of $x$.
\end{thm}

\begin{proof}
We need to find a bijection between $O_x$ and $G/G_x =$ the set of left
cosets of $G_x$ in $G$. Given a point $g\cdot x \in O_x$, where $g
\in G$, send it to the left coset $gG_x$.  This rule defines a map $f:
O_x \to G/G_x$.

The map $f$ is well-defined since if $g\cdot x = h\cdot x$ for $g,h
\in G$ then by left multiplication by $h^{-1}$ we get $(h^{-1}g)\cdot
x = x$, so $h^{-1}g \in G_x$ and thus $gG_x=hG_x$.  The map $f$ is
injective since if $gG_x = hG_x$ then $h^{-1}g \in G_x$ so
$(h^{-1}g)\cdot x = x$, so $g\cdot x = h\cdot x$.  The map $f$ is
surjective because any left coset $gG_x$ is the image of the point
$g\cdot x$ in the orbit of $x$.

Thus, $f$ is a bijective function from $O_x$ to the set $G/G_x$ of
left cosets of $G_x$. The index $[G:G_x]$ is by definition the number
of left cosets, so the proof is complete.
\end{proof}



Although the theorem holds generally, for any group (finite or
infinite) acting on any set (finite or infinite), it is most useful
when $G$ is finite, in which case $[G:G_x]=|G|/|G_x|$ by Lagrange's
theorem. This gives a nice corollary of the orbit-stabilizer theorem.


\begin{cor}\label{countG}
  Suppose that $G$ is a finite group acting on a set $X$. Then for any
  point $x \in X$ we have $|G| = |G_x|\cdot |O_x|$.
\end{cor}

\begin{proof}
Lagrange's theorem says that $[G:G_x]=|G|/|G_x|$. The orbit-stabilizer
theorem says that $[G:G_x] = |O_x|$, so $|O_x| = |G|/|G_x|$ and the
result follows.
\end{proof}



In case the $G$-set $X$ is finite we have the following result, which
can be used to count the number of points in the set $X$.


\begin{cor} \label{orbit-union}
  Let $X$ be a finite $G$-set. Let $O_{x_1}, \dots, O_{x_k}$ be the
  distinct orbits. Then $|X| = \sum_{i=1}^k \; [G:G_{x_i}]$.
\end{cor}

\begin{proof}
Since the action of $G$ partitions $X$ into disjoint orbits, we can
write $X$ as the union of its distinct orbits:
\[
 X = \textstyle \bigcup_{i=1}^k \; O_{x_i} = O_{x_1} \sqcup O_{x_2} \sqcup
 \cdots \sqcup O_{x_n}.
\]
Since the orbits listed are non-overlapping (i.e., pairwise disjoint)
we have
\[
 |X| = \sum_{i=1}^k |O_{x_i}| = \sum_{i=1}^k [G:G_{x_i}]
\]
where the last equality on the line above is justified by Theorem
\ref{osr}. This completes the proof.
\end{proof}



\begin{rmk}\index{orbit~representatives}
The set $\{x_1, \dots, x_k\}$ in the preceding
result is called a \emph{complete set of orbit representatives}. To
obtain such a set, one chooses exactly one element from each distinct
orbit.
\end{rmk}

Here is one more piece of terminology that is often used in the theory
of group actions.


\begin{defn}\index{transitive~action}
Let $X$ be a $G$-set. We say that the action of $G$ is {\em
  transitive}, or that $G$ acts \emph{transitively}, if there is only
one orbit. This means that we can get from any point $x \in X$ to any
other point $y \in X$ by acting by a suitable group element.
\end{defn}


\begin{example}
  The action of $\D_n$ on the set of vertices (or the set of edges) of
  a regular $n$-gon is transitive.
\end{example}


\section*{Exercises}

\begin{problems}

\item Let $X$ be a $G$-set. Prove Lemma \ref{lem:stab-is-a-subgp},
  that the stabilizer $G_x$ of any point $x\in X$ is a subgroup of
  $G$.

\item Prove Proposition \ref{eqrel}.

\item Let $\D_4$ act on the set of vertices of a square by its natural
  action.
  \begin{enumerate}
  \item Is the action transitive? 
  \item Compute the stabilizer of a vertex.
  \end{enumerate}

\item Let $\Sym_4$ act on the set $\{1,2,3,4\}$ by $\alpha \cdot j =
  \alpha(j)$, for each $j \in \{1,2,3,4\}$.
  \begin{enumerate}
  \item Is the action transitive?
  \item Compute the stabilizer of $4$. What group is it isomorphic to?
  \end{enumerate}

\item Let $\Sym_n$ act on the set $\n = \{1, \dots, n\}$ by $\alpha
  \cdot j = \alpha(j)$, for each $j \in \n$.
  \begin{enumerate}
  \item Is the action transitive?
  \item Compute the stabilizer of $n$. What group is it isomorphic to?
  \end{enumerate}

\item Is the action of $\GL_2(\R)$ on $\R^2$ given by $A\cdot X = AX$
  (ordinary matrix multiplication) transitive? Justify your answer.

\item Let $X = \{1,2,3\}$. Consider the action of the alternating
  group $\Alt_3 = \{ (1), (1,2,3), (3,2,1) \}$ on $X$ defined by
  $\alpha \cdot j = \alpha(j)$. 
  \begin{enumerate}
  \item How many orbits are there?
  \item Describe the orbits completely.
  \item Compute the stabilizer of $3$. 
  \end{enumerate}



\item Let $X$ be the set of line segments connecting any two vertices
  of the square, i.e., the edges and the diagonals. The group $\D_4$
  acts on the set $X$ in a natural way: if $L$ is a line segment
  connecting vertices $i$ and $j$ and $g \in \D_4$ then $g\cdot L$ is
  the line segment connecting vertices $g\cdot i$ and $g \cdot j$. 
  \begin{enumerate}
  \item Describe the orbits completely. (You may wish to number the
    vertices of the square.) How many orbits are there?
  \item Compute the stabilizer of one of the diagonals.
  \item Compute the stabilizer of one of the edges.
  \end{enumerate}

\end{problems}



\newpage
\section{Applications of group actions}\noindent
We now consider a few examples of the theory of group actions. As we
will see, group actions are useful to count the size of various sets;
thus group actions are a useful tool in
\index{combinatorics}combinatorics.\footnote{Combinatorics, roughly
  speaking, is the science of counting.}

\begin{example}\index{rotation~group}
The rotation group $\SO_2(\R)$ acts on the set $\R^2$ by ordinary
matrix multiplication: $A\cdot x = Ax$ for any $A \in \SO_2(\R)$ and
any $x \in \R^2$ (considered as a column vector).  The orbit of any
point $(a,b) \in \R^2$ is a circle of radius $\sqrt{a^2+b^2}$.
\end{example}


\begin{example}\index{dihedral~group}
Consider the symmetry group $G=\D_n$ of a regular $n$-gon in the
plane. Let $v$ be a given vertex of the $n$-gon. Clearly $\D_n$ has
$n$ different rotations, and the rotation subgroup $\gen{r}$ generated
by the basic rotation $r$ acts transitively on the set $X$ of all $n$
vertices.  Of all the rotations, only the identity fixes $v$. There is
just one reflection that fixes $v$, namely the reflection across the
(unique) line of symmetry of the figure passing through the vertex
$v$. This proves that the stabilizer $G_v$ of the point $v$ has order
2, since $G_v$ consists of just the identity and the indicated
reflection. Hence, by Corollary \ref{countG} we conclude that $|\D_n|
= |O_v|\cdot |G_v| = n\cdot 2 = 2n$.  Group actions provide a tool to
prove rigorously that $|\D_n|=2n$, something we found difficult
earlier because we lacked the appropriate theory and terminology.
\end{example}


\begin{example} 
Let $G$ be the group of proper symmetries of a cube. (Recall that
proper symmetries are rotations.)  We are going to use Corollary
\ref{countG} to count the number of elements of $G$. Clearly $G$ acts
on the set of 8 vertices of the cube. The action is transitive since
you can get from any chosen vertex to any other, by an appropriate
sequence of rotations. Let us fix a chosen vertex, call it $v$. Then
$|O_v|=8$.  Moreover, the only rotations fixing $v$ are the three
rotations whose axis lies along the diagonal line segment connecting
$v$ to its opposite vertex. (It helps to hold an actual cube in your
hands to see this.) So $|G_v|=3$. Thus by Corollary \ref{countG} we
conclude that $|G| = 8\cdot 3 = 24$. So the octahedral group has $24$
elements. (Recall that the symmetry group of the cube and the
octahedron are the same group, since the cube and octahedron are dual
polyhedra.)

By similar methods you should be able to count the number of proper
symmetries of a tetrahedron (12) and the number of proper symmetries
of a dodecahedron or its dual, an icosahedron (60). Without the aid of
the orbit =stabilizer theorem, counting the size of these groups would
be rather daunting
\end{example}



\begin{example} 
Consider the group $\Sym_n$ acting on the set $\n = \{1, \dots, n \}$
in the natural way: $\alpha \cdot j = \alpha(j)$.  This action is
transitive because one can get from any $i$ to any $j$ by a suitable
permutation. So if we fix a chosen number $i$ then $O_i = \{1,\dots,
n\}$. The number of permutations that fix $i$ is $|\Sym_{n-1}|$, since
we are free to permute the other $n-1$ positions arbitrarily. Thus by
Corollary \ref{countG} we conclude that $|\Sym_n|=n
|\Sym_{n-1}|$. Since $|\Sym_1|=1$, we conclude by a simple induction
on $n$ that $|\Sym_n| = n!$ for any positive integer $n$. This gives a
new proof (via the orbit-stabilizer theorem) that $|\Sym_n| = n!$.
\end{example}



\begin{defn} (Permutation Representation)
\index{permutation~representation}\label{permrep}%
Let $X$ be a $G$-set. Given $g \in G$, consider the map $\varphi_g: X
\to X$ given by the rule $x \mapsto g\cdot x$. In other words,
$\varphi_g(x) = g\cdot x$.  Then the map $\varphi_g$ is a bijection
(exercise), so $\varphi_g$ is a permutation of the set $X$.  In other
words, $\varphi_g \in \Sym_X$, the group of permutations of $X$.  The
rule $\varphi(g) = \varphi_g$ for each $g\in G$ defines a function
$\varphi$ from $G$ to $\Sym_X$.  This map $\varphi \colon G \to
\Sym_X$ is called the \emph{permutation representation} of the given
$G$-set $X$.
\end{defn}

\begin{lem}\label{lem:perm-rep-is-a-homo}
  The permutation representation $\varphi \colon G \to \Sym_X$ is a
  group homomorphism.
\end{lem}

\begin{proof}
We need to check that $\varphi(gh) = \varphi(g) \varphi(h)$. In other
words, we must check that $\varphi_{gh} = \varphi_g \circ
\varphi_h$. We can verify this equality of functions by verifying that
the functions on the two sides of the equality act the same on every
possible input. So consider any $x \in X$. Then by definition, we have
\[
  \varphi_{gh}(x) = (gh) \cdot x, \qquad (\varphi_g \circ
  \varphi_h)(x) = \varphi_g(\varphi_h(x)) = \varphi_g(h \cdot x) = g
  \cdot(h \cdot x).
\]
These are the same by the definition of group actions.
\end{proof}

The permutation representation provides a way to model abstract group
elements $g \in G$ by permutations $\varphi_g$ of $X$, in such a way
that the group multiplication is reproduced in the permutations.

Every group action gives rise to a permutation representation in this
way. We now apply this observation to prove the following important
result.


\begin{thm}[Cayley's theorem]\index{Cayley's~theorem} 
Every group is isomorphic to a group of permutations on some set $X$.
In particular, every finite group $G$ is isomorphic to a subgroup of
$\Sym_n$ where $n = |G|$.
\end{thm}

\begin{proof}
Consider the action of $G$ on $G$ itself by left multiplication:
$g\cdot x = gx$ for any $g,x \in G$. Here we are taking the set $X$ to
be $G$ itself.  We use the permutation representation of Definition
\ref{permrep}.  By Lemma \ref{lem:perm-rep-is-a-homo}, we know that
$\varphi: G \to \Sym_G$ is a group homomorphism. The kernel of
$\varphi$ is trivial, since $\varphi(g) = id_G$ implies that
$\varphi_g = id_G$, which in turn implies that $gx=x$ for all $x \in
G$. This forces $g=1$.

Thus we have an injective homomorphism $\varphi: G \to \Sym_G$. This
homomorphism induces an isomorphism of $G$ onto its image, which is a
subgroup of $\Sym_G$. In other words, we have produced an isomorphism
from $G$ to a group of permutations, as was desired.

To get the last statement, in case $G$ is finite, just number the
elements of $G$ from 1 to $n$. Then permutations of $G$ can be
regarded as permutations of the set $\mathbf{n} = \{1, \dots, n\}$;
i.e., $\Sym_G \cong \Sym_n$ where $n =|G|$.
\end{proof}




In the proof of Cayley's theorem above, we considered the action of
$G$ on itself by left multiplication.  Another way that a group $G$
can act on itself is by conjugation.  Analysis of this action leads to
important structural information about the group.


\begin{defn}\index{conjugation} \label{conj-elt}
Given $g,x \in G$ define $g\cdot x = gxg^{-1}$. This action of $G$ on
itself is called \emph{conjugation}. The element $gxg^{-1}$ is called
a \emph{conjugate} of $x$.
\end{defn}



The orbits for the conjugation action are called {\em conjugacy
  classes}\index{conjugacy~classes}, and $G$ is the disjoint union is
its distinct conjugacy classes. If two elements of $G$ lie in the same
conjugacy class, they are said to be \emph{conjugate} in $G$.  We
write $C(x)$ for the conjugacy class of $x$; i.e.,
$C(x) = \{ gxg^{-1} \colon g \in G\}$.


\begin{defn}\index{centralizer}
Given an element $x \in G$, its stabilizer is the subgroup $G_x$ given
by
\[
  \{ g\in G \mid gxg^{-1} = x \} = \{ g\in G \mid gx=xg \}.
\]
This subgroup is known as the {\em centralizer} of $x$, denoted by
$Z_G(x)$.
\end{defn}


In words, the centralizer of $x$ is the set of all $g \in G$ which
commute with the element $x$.  In this context, the orbit-stabilizer
theorem (Theorem \ref{osr}) says that
\[
   |C(x)| = [G:Z_G(x)]
\] 
for any $x \in G$. (Because, $C(x)=O_x$ and $Z_G(x)=G_x$ in the
earlier notation.)

Notice that the center $Z(G)$ of the group $G$ is contained in every
centralizer: $Z(G) \subset Z_G(x)$, for any $x\in G$. In fact, $Z(G)$
is equal to the intersection of all the centralizers:
$Z(G)=\bigcap_{x\in G} Z_G(x)$.


\begin{defn}\index{center}
For any subset $S$ of $G$, the centralizer of $S$ by $Z_G(S) =
\bigcap_{x\in S} Z_G(x)$. Then $Z_G(S) = \{ g\in G \mid gx=xg,
\forall\ x\in S\}$. In words, the centralizer of a subset $S$ is the
set of all elements of $G$ commuting with all elements of $S$.
\end{defn}

In this notation, the center $Z(G)$ is equal to the centralizer of
$G$; i.e., $Z(G) = Z_G(G)$. This is clear from the definitions.

Let me point out that if $z$ is an element of the center $Z(G)$ then
the conjugacy class of $z$ (the orbit) is just the singleton set
$\{z\}$ (since $z$ commutes with all elements, so $gzg^{-1} = z$ for
any $g$) and $Z_G(z) = G$.


These notions lead us to the following important theorem, which gives
new information about finite groups.

\begin{thm}[The class equation]\index{class~equation}\label{classeqn}
Let $G$ be a finite group, $Z(G)$ the center of $G$. Let $x_1, \dots,
x_t$ be a complete set of representatives for the conjugacy classes
that are disjoint from $Z(G)$. Then
\[
   |G| = |Z(G)| + \sum_{j=1}^t [G:Z_G(x_j)].
\]
\end{thm}


\begin{proof}
This is a restatement of Corollary \ref{orbit-union}. Label the
elements of $Z(G)$ by $z_1, z_2, \dots, z_s$. Then $z_1, \dots, z_s,
x_1, \dots, x_t$ is a complete set of representatives of the conjugacy
classes, so by Corollary \ref{orbit-union} we have
\[
  |G| = \sum_{i=1}^s [G:Z_G(z_i)] + \sum_{j=1}^t [G:Z_G(x_j)].
\]
But each $[G:Z_G(z_i)]$ in the first sum is equal to 1 since $Z_G(z_i)
= G$ by the remarks above, so the first sum is equal to $s$, the
number of elements in $Z(G)$, and the result is proved.
\end{proof}



\begin{defn}\index{p@$p$-group}
Let $p$ be a prime. A $p$-{\em group} is a group in which every
element is of prime power order $p^r$, for some positive integer $r$.
\end{defn}

Note that any group of order $p^r$ for a positive integer $r$ must be
a $p$-group, by Lagrange's theorem.  The study of $p$-groups is quite
important for understanding finite groups. It turns out that
$p$-groups are the most difficult class of finite groups to
understand, partly because there are so many of them.  For instance,
it is known that there are 267 different groups of order $64 = 2^6$,
up to isomorphism. We will have more to say about $p$-groups later.


Here are two immediate applications of the class equation, each of
which is a result about $p$-groups.



\begin{cor} Let $p$ be a prime number. 
\begin{enumerate}
\item Any group of order $p^r$ ($r \ge 1$) must have at least $p$
  elements in its center.

\item All groups of order $p^2$ are abelian. 
\end{enumerate}
\end{cor}


\begin{proof}
(a) By the class equation, $p^r = |Z(G)|+\sum [G:Z_G(x_j)]$. But each
  $Z_G(x_j)$ is a {\em proper} subgroup of $G$ since otherwise $x_j$
  would be in the center. So, by Lagrange's Theorem, the order of
  $Z_G(x_j)$ must be a proper divisor of $p^r$. But that means the
  order is of the form $p^m$ for some $m<r$ and so each $[G:Z_G(x_j)]$
  is divisible by $p$.  It follows that $|Z(G)|$ must be divisible by
  $p$. Part (a) is proved.

(b) If we can show that $Z(G) = G$ then we are done, because $Z(G)$ is
  obviously abelian. So suppose not. Then there exists some $x \in G$
  with $z \notin Z(G)$. Now $Z_G(x)$ is a subgroup of $G$ containing
  $Z(G)$. By part (a), $Z(G)$ must have at least $p$ elements, so the
  same is true of $Z_G(x)$. By Lagrange's Theorem, it follows that
  $|Z_G(x)|$ is either $p$ or $p^2$. But $|Z_G(x)|$ cannot be equal to
  $p^2$ or else $x$ would belong to the center, contrary to our
  assumption on $x$. So $|Z_G(x)|$ must equal $p$.  But this implies
  that $Z_G(x)=Z(G)$. But $x\in Z_G(x)$ ($x$ commutes with itself) so
  $x \in Z(G)$. This is a contradiction.  The contradiction forces
  $Z(G)=G$, so $G$ is abelian. Part (b) is proved.
\end{proof}


Next we consider the conjugation action on subsets of $G$.  Recall
that in set theory the power set of a set $S$ is the collection
$\mathcal{P}(S)$ of all subsets of $S$.  Any group $G$ acts on its
power set $X = \mathcal{P}(G)$ by conjugation, as follows.

\begin{defn}
Given any subset $S$ of $G$, and any element $g \in G$, the conjugate
$gSg^{-1} = \{gsg^{-1} \colon g \in G\}$ of $S$ by $g$ is another
subset of $G$. Thus the rule $g\cdot S = gSg^{-1}$ for any $g \in G$,
$S \in \mathcal{P}(G)$ defines an action of $G$ on the set
$\mathcal{P}(G)$.
\end{defn}

The orbit of a given subset $S$ for this action is the set of all
conjugates $gSg^{-1}$ of $S$; this set is sometimes denoted
$C(S)$. The power set $\mathcal{P}(G)$ of $G$ is the disjoint union of
these conjugacy classes. The stabilizer of the subset $S$ is 
\[
   N_G(S) = \{ g \in G \mid gSg^{-1} = S \} = \{ g\in G \mid gS = Sg \},
\]
which is called the {\em normalizer}\index{normalizer} of $S$ in
$G$. Note that the set equality $gS=Sg$ does \emph{not} mean that
$gs=sg$ for each $s \in S$; rather it means that for every $s\in S$
there is some $s'\in S$ such that $gs = s'g$.  In this context, the
orbit-stabilizer theorem (Theorem \ref{osr}) says that
\[
  |C(S)| = [G:N_G(S)]
\]
for any subset $S$ of $G$. 

 
Note that if $H$ is a given subgroup of $G$ then $H \triangleleft G$
if and only if $N_G(H) = G$. This follows readily from the definitions. 

\section*{Exercises}

\begin{problems}

\item Compute the conjugacy classes of $\Sym_3$. How many elements are
  in the centralizer $Z_{\Sym_3}(\alpha)$ if $\alpha$ is a 3-cycle?

\item Compute the conjugacy classes of $\Sym_4$. How many elements are
  in the centralizer $Z_{\Sym_4}(\alpha)$ if $\alpha$ is a 4-cycle?

\item Let $G$ be a group. Prove that:
\par(a) The center $Z(G)$ is a subgroup of any centralizer $Z_G(x)$,
for any $x\in G$. 
\par(b) $Z(G) = \bigcap_{x\in G} Z_G(x)$. 

\item Prove that for any subset $S$ of a group $G$, $Z_G(S)$ must be a
  subgroup of $N_G(S)$. 

\item Prove that if $H < G$ then $H \triangleleft G$ if and only if
  $N_G(H) = G$.

\item (a) Prove that the map $\varphi_g$ of \ref{permrep} is a bijection. 
\par (b) Verify that $\varphi$ (see \ref{permrep}) is a homomorphism.

\item Use a permutation representation of the dihedral group $\D_3$ to
  find a permutation group isomorphic to $\D_3$. Write out a list of
  the elements of the permutation group, and explain how the
  isomorphism is defined.

\item Use a permutation representation of the cyclic group $\Z_n$ to
  find a permutation group isomorphic to $\Z_n$. Explain how the
  isomorphism is defined.

\item If $G$ is a group of order $p^r$ where $r$ is a positive integer
  and $p$ is a prime, then show that every subgroup of $G$ has order
  $p^k$ for some integer $k \le r$.

\item If $G$ is a group of order $p^r$ where $r$ is a positive integer
  and $p$ is a prime, then show that $|Z(G)| = p^k$ where $k \ge 1$.

\end{problems}



\newpage
\section{Burnside's Lemma}\noindent
Another important application of group actions is the result commonly
known as Burnside's lemma. This result was proved originally by Georg
Frobenius in 1887, so it is not due to Burnside. Burnside included it
in his popular 1897 book \emph{On the Theory of Groups of Finite
  Order}. By an accident of history, the result has become known as
Burnside's lemma.


\begin{thm}[Burnside's lemma]\index{Burnside's lemma}
  Let $G$ be a finite group acting on a finite set $X$. Then the
  number $N$ of orbits is given by 
  \[
    N = \frac{1}{|G|} \sum_{g \in G} |X^g|
  \]
  where $X^g = \{ x \in X \mid g \cdot x = x\}$ is the set of fixed
  points of $g$.
\end{thm}


\begin{proof}
First we prove the result in case $G$ acts transitively. Then $N=1$
and we need to show that $|G| = \sum_{g \in G} |X^g|$. Let 
\[ T = \{(g,x) \in G \times X \mid g \cdot x = x\}. \]
Fix some $x \in X$. The pair $(g,x) \in T$ if and only if $g \in G_x$,
so the number of such pairs is $|G_x|$. Since the action is
transitive, $G_x$ is conjugate to $G_y$ for any $y \in X$, so $|G_y| =
|G_x|$ for all $y \in X$. So if we sum over $y$ we get
\[
  |T| = \sum_{y \in X} |G_y| = \sum_{y \in X} |G_x| = |X| \cdot |G_x| = |G|
\]
where the final equality is by the orbit-stabilizer theorem. On the
other hand, fix some $g \in G$ and count $|T|$ another way. The pair
$(g,x) \in T$ for some $x \in X$ if and only if $g \cdot x = x$, so
the set of $x \in X$ with this property is just $X^g$. Summing over
all $g \in G$ we get
\[
  |T| = \sum_{g \in G} |X^g|.
\]
By transitivity of equality, the right hand side of each of the last
two displayed equations are equal, which proves the theorem in the
transitive case.

Now we consider the general case. Observe that $G$ acts transitively
on each of its $N$ orbits. Thus the formula we just proved applies to each
orbit. Also the total number of fixed points of a group element is the
sum of the number of fixed points in each orbit. Hence
\[
  N |G| = \sum_{g \in G} |X^g|.
\]
This proves the result after dividing both sides by $|G|$.
\end{proof}

This simple counting formula has many useful applications. We give
just one example here to whet the reader's appetite.

\begin{example}
Suppose that we wish to count the number of ways to color the vertices
of a regular pentagon by red and green. There are two ways to color
each vertex, and five vertices, so the simplest answer to our question
is that there are $32 = 2^5$ colorings.

But we are usually interested in more sophisticated counting
questions. Suppose that we are making a necklace with five red and
green beads. We do not wish to distinguish between patterns which are
the same under a symmetry of the pentagon, because such patterns
produce the same necklace. 

To count these, let $X$ be the set of all 32 patterns. The symmetry
group $\D_5$ of the pentagon acts on $X$, and we want to know how many
orbits there are. The Burnside lemma will tell us the answer, as soon
as we have computed the number of fixed points of each symmetry. 

Case 1. $X^1 = X$. Obviously each element of $X$ is fixed by $1 \in
\D_5$. So $|X^1| = 32$. 

Case 2. $|X^r|=2$. The only patterns fixed by a rotation $r$ are the
ones in which all colors are the same, either all red or all green.
This holds true for all three non-trivial rotations.

Case 3. $|X^{d}|=8$. Recall that a basic reflection $d$ fixes one
vertex and interchanges the other four vertices in opposite pairs.
The only patterns fixed by the basic reflection $d$ are therefore those
that have the same color on opposite vertices. So there are two
colorings for the fixed vertex, and two each for the opposite pairs,
for a total of $2^3=8$ colorings left fixed by $d$. The same is true
of each of the other four reflections.

Now we apply the Burnside formula. The identity element of $\D_5$
produces 32 fixed points, each of the four non-trivial rotations
produces 2 fixed points, and finally each of the five reflections
gives 8 fixed points, so the number of orbits is
\[
  N = \frac{1}{10}(32 + 4 \cdot 2 + 5 \cdot 8) = 8.
\]
This solves our problem. There are exactly 8 distinct colorings by two
colors of a necklace with five beads.
\end{example}


Chemists use Burnside's lemma to count chemical compounds. For
example, a benzene molecule can be modeled by six carbon atoms in a
regular hexagon in a plane. One of three radicals can be attached to
each carbon atom to form a benzene molecule. Counting the number of
possible benzene molecules is thus an exercise in Burnside's lemma.

There is a generalization of Burnside's lemma known as the \emph{Polya
  enumeration theorem}. Applications of group theory to counting
problems abound, and we can only hint at the possibilities here.


\section*{Exercises}
\begin{problems}

\item Determine the number of necklaces with 6 beads of two possible
  colors.

\item Determine the number of necklaces with 4 beads of three possible
  colors.

\item Determine the number of necklaces with 5 beads of three possible
  colors.

\item A benzene molecule can be modeled by six carbon atoms in a
  regular hexagon in a plane. One of three radicals can be attached to
  each carbon atom to form a benzene molecule. Count the number of
  possible benzene molecules using Burnside's lemma.

\item In how many ways can the faces of a cube be colored by three
  colors up to rotational symmetry? [Hint: The answer should be 57.]

\end{problems}



\end{document}


