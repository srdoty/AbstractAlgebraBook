\documentclass[11pt,oneside]{article}
\usepackage[nohead,margin=1.50in]{geometry} %set margins
\usepackage{amsmath,amssymb,amsthm,pdiag} %AMS packages for math stuff
\usepackage{multicol} % for use in HW section
\usepackage{enumitem}
  \setlist{topsep=1pt,itemsep=0pt,parsep=1pt}
  \setenumerate[1]{label=(\alph*)}

\newenvironment{problems}
{
 \begin{enumerate}[topsep=1pt,itemsep=0pt,parsep=2pt,leftmargin=0.6cm,%
 label={\arabic*.}, ref=\arabic*] \small
}
{
 \end{enumerate}
}

%%% Define some theorem and example environments. The starred versions
%%% are un-numbered and the unstarred versions are numbered.
\newtheoremstyle{plain}
  {\topsep}   % ABOVESPACE
  {\topsep}   % BELOWSPACE
  {\slshape}  % BODYFONT
  {0pt}       % INDENT (empty value is the same as 0pt)
  {\bfseries} % HEADFONT
  {.}         % HEADPUNCT
  {5pt plus 1pt minus 1pt} % HEADSPACE
  {}          % CUSTOM-HEAD-SPEC

\swapnumbers
\newtheorem{thm}{Theorem}[section]
\newtheorem{lem}[thm]{Lemma}
\newtheorem{prop}[thm]{Proposition}
\newtheorem{cor}[thm]{Corollary}
\newtheorem*{thm*}{Theorem}
\newtheorem*{lem*}{Lemma}
\newtheorem*{prop*}{Proposition}
\newtheorem*{cor*}{Corollary}

\theoremstyle{definition}
\newtheorem{defn}[thm]{Definition}
\newtheorem{example}[thm]{Example}
\newtheorem{examples}[thm]{Examples}
\newtheorem{rmk}[thm]{Remark}
\newtheorem{rmks}[thm]{Remarks}
\newtheorem{conv}[thm]{Convention}
\newtheorem*{defn*}{Definition}
\newtheorem*{example*}{Example}
\newtheorem*{examples*}{Examples}
\newtheorem*{rmk*}{Remark}
\newtheorem{rmks*}{Remarks}
\newtheorem*{conv*}{Convention}


%%% Define some convenient abbreviations for common mathematical
%%% notations.
\newcommand{\R}{\mathbb{R}} % use \R for the real numbers
\newcommand{\C}{\mathbb{C}} % use \C for the complex numbers
\newcommand{\Z}{\mathbb{Z}} % use \Z for the integers
\newcommand{\Q}{\mathbb{Q}} % use \Q for the rationals
\newcommand{\N}{\mathbb{N}} % use \N for the natural numbers
\newcommand{\F}{{\mathbb F}}
\newcommand{\compose}{\circ} % functional composition
\newcommand{\gen}[1]{\langle #1 \rangle}
\newcommand{\End}{\operatorname{End}}
\newcommand{\GL}{\mathrm{GL}}
\newcommand{\SL}{\mathrm{SL}}
\renewcommand{\O}{\mathrm{O}}
\newcommand{\SO}{\mathrm{SO}}
\newcommand{\U}{\mathrm{U}}
\newcommand{\SU}{\mathrm{SU}}
\newcommand{\g}{\mathfrak{g}}
\newcommand{\transpose}{\mathsf{T}}
\newcommand{\B}{\mathcal{B}}
\newcommand{\Rep}{\operatorname{Rep}}
\newcommand{\Mat}{\operatorname{Mat}}
\newcommand{\inner}[2]{\langle #1, #2 \rangle}
\newcommand{\sgn}{\operatorname{sgn}}
\newcommand{\n}{\underline{\mathbf{n}}}
\newcommand{\Sym}{\mathbb{S}}
\newcommand{\Alt}{\mathbb{A}}
\newcommand{\D}{\mathbb{D}}
\newenvironment{perm}[2]{\left(\begin{smallmatrix}#1 \\ #2}{\end{smallmatrix}\right)}
\newcommand{\lcm}{\operatorname{lcm}}
\newcommand{\res}{\operatorname{res}}
\newcommand{\im}{\operatorname{im}}

\allowdisplaybreaks
\parskip=2pt

%\title{Document Title}
%\author{author's name}

\begin{document}%\maketitle
\setcounter{section}{23}

\section{Direct products of groups}\noindent
Recall the Cartesian product of two sets in set theory, which is used
to construct the euclidean plane $\R^2 = \R \times \R$ as the set of
all ordered pairs of real numbers.  For any sets $A$, $B$ we can
similarly form the Cartesian product set 
\[
  A \times B = \{ (x,y) \mid x \in A, y \in B\}.  
\]
Given two groups one can make their Cartesian product into a
group in a very natural way. This makes it possible to construct many
new examples of groups by taking products of ones we already know. 

As usual, we use multiplicative notation in this section, leaving it
to the reader to make the necessary notational adjustments in other
cases.

\begin{defn}\index{direct~product}\index{product~of~groups}\label{prod} 
  The \emph{direct product} of given groups $G$, $H$ is the group $G
  \times H = \{ (x,y) \mid x\in G, y\in H \}$ of ordered pairs, with
  binary operation
  \[
    (x,y) \cdot (u,v) = (xu,yv), \text{ for all $x,u\in G$, $y,v\in
    H$.}
  \]
  The identity element of $G \times H$ is the pair $(1_G, 1_{H})$.
  The inverse of a pair $(x,y)$ is the pair $(x^{-1},y^{-1})$; that
  is: $(x,y)^{-1} = (x^{-1},y^{-1})$.
\end{defn}

It must be checked that this actually works; that is, we must verify
that the above multiplication rule on $G \times H$ makes $G \times
H$ a group. This is left for you as an exercise.

If $G$, $H$ have finite order then we have $|G \times H| =
|G|\cdot|H|$; i.e., the order of the direct product $G \times H$ is
the product of the orders of $G$ and $H$. If one of the groups $G, H$
is infinite then so is their direct product.


\begin{examples} 
1. The group $\Z_2 \times \Z_2$ is a group of order 4.  It is not hard
to see that it is isomorphic with the Klein 4-group.

2. If $m,n$ are relatively prime then $\Z_m \times \Z_n$ is isomorphic
to $\Z_{mn}$.  In particular, if $p$, $q$ are distinct primes then
$\Z_p \times \Z_q \cong \Z_{pq}$.
\end{examples}



Recall that if $H$, $K$ are subsets of a group $G$ then $HK = \{ hk: h
\in H, k \in K \}$. 



\begin{thm}[Internal direct products]
Let $G$ be a group and suppose that $H,K$ are normal subgroups of
$G$. If $H K = G$ and $H \cap K = \{1\}$ then $G \cong H \times K$ and
every element of $G$ is uniquely expressible as a product of an
element of $H$ by an element of $K$.
\end{thm}

\begin{proof}
We define a map $f$ from $H \times K$ to $G$ by the rule $f((x,y)) =
xy$, for any $x\in H$, $y \in K$.  Since $HK=G$ the map $f$ is
surjective.  

Next we claim that {\em elements of $H$ must commute with elements of
  $K$}. For any $x \in H$, $y \in K$ consider the product
$(xyx^{-1})y^{-1} = x(yx^{-1} y^{-1})$. Since $K$ is normal, the left
hand side is an element of $K$. Since $H$ is normal, the right hand
side is an element of $H$. Thus the product under consideration lies
in the intersection $H \cap K$, so $xyx^{-1}y^{-1} = 1$, so $xy =
yx$. This proves the claim.

We can now verify that $f$ is actually a homomorphism:
$f(\,(x,y)(x',y')\,) = f( (xx',yy')) = xx'yy' = xyx'y' = f((x,y))
f((x',y'))$.  Finally, we check that $f$ is injective. Let $(x,y)$
($x\in H$, $y\in K$) be an element of the kernel. Then $xy = 1$, so $y
= x^{-1}$ must belong to $H$, thus to $H \cap K$, and thus $y =
1$. Thus $x=1$ as well, and $(x,y) = (1,1)$. This shows that the
kernel of $f$ must be the trivial subgroup of $H \times K$, so $f$ is
injective, as desired.  We have shown that $f$ is a bijective
homomorphism. Thus $f$ is an isomorphism from $H \times K$ to $G$, so
$H \times K \cong G$.

It remains to show that every element $g \in G$ is uniquely
expressible in the form $hk$, for some $h \in H$, $k \in K$. That $g$
has such an expression is clear from the hypothesis $G=HK$, so we only
need to prove uniqueness. Suppose that $g = h_1k_1 = h_2k_2$ where
$h_1, h_2 \in H$, $k_1, k_2 \in K$. Then by left multiplying by
$h_2^{-1}$ and right multiplying by $k_1^{-1}$ we obtain $h_2^{-1} h_1
= k_2 k_1^{-1}$. In this equation, the left hand side is an element of
$H$ while the right hand side is an element of $K$. So both sides of
the equation belong to $H \cap K$. But $H \cap K = \{1\}$, so
$h_2^{-1} h_1 = 1$ and $k_2 k_1^{-1} = 1$; i.e., $h_1 = h_2$ and $k_1
= k_2$. This proves the uniqueness statement.
\end{proof}

The point of the previous theorem is to ``factor'' the group $G$ as a
direct product of two of its subgroups. This is a group-theoretic
analogue of factoring integers. 

\begin{defn}
  Whenever a group $G$ is isomorphic to the direct product of normal
  subgroups $H, K$ then we say it is the \emph{internal direct
    product} of those subgroups, and we write $G = H \times K$.
\end{defn}

According to the theorem, we can factor $G$ as the internal direct
product of normal subgroups $H,K$ if and only if $HK = G$ and $H \cap
K = \{1\}$.

\begin{examples}
1. If $m,n$ are relatively prime then $\Z_{mn}$ has a unique subgroup
$H$ of order $m$, and a unique subgroup $K$ of order $n$. In fact, $H
= \gen{[n]}$ and $K = \gen{[m]}$ as you can easily check. Then $H+K =
\Z_n$ and $H \cap K = \{[0]\}$, so $\Z_{mn}$ is the internal direct
product of $H, K$. Of course $H \cong \Z_m$ and $K \cong \Z_n$ since
all subgroups of a cyclic group are cyclic. Note that we write $ H+K$
instead of $HK$ here because the groups are additive groups.

2. The group $\Sym_3$ has subgroups $H = \gen{(1,2)}$, $K =
\gen{(1,2,3)}$ of order 2 and 3, respectively. It is clear that $HK =
\Sym_3$ and $H \cap K = \{(1)\}$. Furthermore, $K$ is a normal
subgroup of $\Sym_3$ since it is a subgroup of index 2.  Alas, the
subgroup $H$ is \emph{not} normal. So $\Sym_3$ is not equal to the
direct product of these two subgroups. (In fact, it is impossible to
express $\Sym_3$ as the internal direct product of any two of its
subgroups.)
\end{examples}

The discussion extends to products of more than two groups. 

\begin{defn}\label{def:gen-prod} 
  The \emph{direct product} of given groups $G_1, G_2, \dots, G_n$ is
  the group $G_1 \times G_2 \times \cdots \times G_n = \{ (x_1, x_2,
  \dots, x_n) \mid x_k \in G_k \text{ for all } k = 1, \dots, n \}$,
  with binary operation
  \[
    (x_1, x_2, \dots, x_n) \cdot (y_1, y_2, \dots, y_n) = (x_1y_1,
  x_2y_2, \dots, x_ny_n).
  \]
  The identity element is the tuple $(1, \dots, 1)$.  The inverse of
  $(x_1, \dots, x_n)$ is the tuple $(x_1^{-1}, \dots, x_n^{-1})$ of
  inverses; i.e.\ $(x_1, \dots, x_n)^{-1} = (x_1^{-1}, \dots,
  x_n^{-1})$.
\end{defn}

Again, it must be verified that this really is a group. The
verification is no more difficult than for the case of products of two
groups, and thus left to the reader. 


If $H_1, H_2, \dots, H_n$ are subsets of a group $G$ then we extend
the product notation in the obvious way: $H_1H_2 \cdots H_n = \{
h_1h_2 \cdots h_n: h_k \in H_k, \text{ for all } k = 1, 2, \dots, n
\}$.



\begin{thm}[Internal direct products]
Let $G$ be a group and suppose that $H_1, H_2, \dots, H_n$ are normal
subgroups of a group $G$. If $H_1H_2 \cdots H_n = G$ and
\[
  H_k \cap (H_1 \cdots H_{k-1} H_{k+1} \cdots H_n) = \{1\}, \text{ for
    all $k = 1, \dots, n$ }
\] 
then $G \cong H_1 \times H_2 \times \cdots \times H_n$ and every
element of $G$ is uniquely expressible as a product of the form $h_1
h_2 \cdots h_n$, where $h_k \in H_k$ for all $k$.
\end{thm}

The proof is omitted.  

\begin{defn}
  Whenever normal subgroups $H_1, H_2, \dots, H_n$ of a group $G$ can
  be found such that $G \cong H_1 \times H_2 \times \cdots \times H_n$
  then we say that $G$ is the \emph{internal direct product} of the
  subgroups, and write $G = H_1 \times H_2 \times \cdots \times H_n$.
\end{defn}

For this to hold, it is necessary that the subgroups are normal
subgroups and that they satisfy the conditions of the theorem.

Since internal and external direct products are isomorphic, people
usually do not bother to make a distinction between them.

For finite subgroups of a group, the following simple counting
proposition can be useful. 

\begin{prop}\label{prop:counting-HK}
  Let $H,K$ be finite subgroups of a group $G$ such that $H \cap K =
  \{1\}$.  Then $|HK| = |H| \cdot |K|$.
\end{prop}

\begin{proof}
By definition, $HK = \{xy \mid x \in H, y \in K\}$. The number of
elements is $|H| \cdot |K|$ precisely when all the listed products are
distinct, so that's what needs to be shown. So suppose that $x_1y_1 =
x_2 y_2$ where $x_1,x_2 \in H$, $y_1, y_2 \in K$. Left multiply the
equation $x_1y_1 = x_2 y_2$ by $x_1^{-1}$ and right multiply by
$y_2^{-1}$ to get $y_1y_2^{-1} = x_1^{-1}x_2$. Since $H,K$ are
subgroups of $G$, the left hand side $y_1y_2^{-1} \in K$ and the right
hand side $x_1^{-1}x_2 \in H$. Since they are equal, both elements are
in $H \cap K$. Since $H \cap K = \{1\}$, it follows that $y_1y_2^{-1}
= 1$ and $x_1^{-1}x_2 =1$; i.e., $y_1 = y_2$ and $x_1 = x_2$.
\end{proof}




\section*{Exercises}
\begin{problems}
\item Verify that if $G,H$ are groups then $G \times H$ is a group.


\item If $G, H$ are abelian groups, show that their direct product $G
  \times H$ is abelian.

\item Show that $G \times \{1\} \cong G$.

\item Is the dihedral group $\D_n$ ever isomorphic to $\Z_n \times
  \Z_2$? Prove your answer.

\item Show that if $m,n$ are relatively prime then $C_{mn} \cong C_m
  \times C_n$. (Here, $C_n$ means the cyclic group of order $n$.)

\item Show that if $m,n$ are relatively prime then $\Z_{mn} \cong \Z_m
  \times \Z_n$.

\item Prove that $G \times H \cong H \times G$.

\item Prove that if $G \times H \cong G \times K$ then $H \cong
  K$. (This can be called a \emph{cancellation property} for direct
  products.)


\item The symmetric group $\Sym_3$ has composition factors isomorphic
  to the cyclic groups $C_2, C_3$ of order 2 and 3. Show that $\Sym_3
  \not\cong C_2 \times C_3$. (This shows that not all groups are
  isomorphic to the product of their composition factors.)

\item If $|G| = mn$ where $m,n$ are relatively prime and $G$ has
  normal subgroups $H,K$ of order $m,n$ respectively, then show that
  $G = H \times K$.

\item Show that if $H,K$ are normal subgroups of a group $G$, $HK =
  G$, and every $g \in G$ is \emph{uniquely} expressible in the form
  $g = hk$ for some $h \in H$, $k \in K$ then $H \cap K = \{1\}$ and
  hence $G = H \times K$.

\item If $H_1, H_2, \dots, H_n$ are normal subgroups of a
  group $G$, $H_1H_2 \cdots H_n = G$, and every $g \in G$ is
  \emph{uniquely} expressible as a product of the form $g =h_1 h_2
  \cdots h_n$, where $h_k \in H_k$ for all $k$, then show that
  \[
  H_k \cap (H_1 \cdots H_{k-1} H_{k+1} \cdots H_n) = \{1\}, \text{ for
    all $k = 1, \dots, n$ }
  \] 
  and thus $G = H_1 \times H_2 \times \cdots \times H_n$.

\item Show that if $G = H_1 \times H_2 \times \cdots \times H_n$ is
  the internal direct product of normal subgroups $H_1, H_2, \dots,
  H_n$ then for any $i \ne j$ we have:
  \begin{enumerate}
  \item $H_i \cap H_j = \{1\}$.
  \item $ab = ba$ for all $a \in H_i$, $b \in H_j$.  [Hint: Argue that
    $aba^{-1}b^{-1}$ is in both $H_i$ and $H_j$.]
  \end{enumerate}



\item 
  \begin{enumerate}
  \item Show that the map $x \mapsto (x,1)$ is an injective
    homomorphism from $G$ into $G \times H$.
  \item Show that the image of this homomorphism is a normal subgroup 
    of $G \times H$ isomorphic to $G$.
  \end{enumerate}


\item\index{projection} Show that the rule $(x,y) \mapsto x$ defines a
  surjective homomorphism $p_1$ mapping $G \times H$ onto
  $G$. Similarly, the rule $(x,y) \mapsto y$ defines a surjective
  homomorphism $p_2$ mapping $G\times H$ onto $H$. These maps are
  called \emph{projections}. Describe their kernels and images.

\item If $G = H \times K$ is the direct product of two normal
  subgroups $H,K$ then prove that $G/H \cong K$ and $G/K \cong H$.

\item\index{diagonal~subgroup} Let $G$ be a group and consider the
  direct product $G \times G$.  Show that the set
  $S = \{ (x,x) \mid x\in G \}$ is a subgroup of $G \times G$. Then
  show it is isomorphic to $G$. (It is known as the \emph{diagonal}
  subgroup of $G \times G$.)



\end{problems}

\end{document}
