\documentclass[11pt,oneside]{article}
\usepackage[nohead,margin=1.50in]{geometry} %set margins
\usepackage{amsmath,amssymb,amsthm,pdiag} %AMS packages for math stuff
\usepackage{multicol} % for use in HW section
\usepackage{enumitem}
  \setlist{topsep=1pt,itemsep=0pt,parsep=1pt}
  \setenumerate[1]{label=(\alph*)}

\newenvironment{problems}
{
 \begin{enumerate}[topsep=1pt,itemsep=0pt,parsep=2pt,leftmargin=0.6cm,%
 label={\arabic*.}, ref=\arabic*] \small
}
{
 \end{enumerate}
}

%%% Define some theorem and example environments. The starred versions
%%% are un-numbered and the unstarred versions are numbered.
\newtheoremstyle{plain}
  {\topsep}   % ABOVESPACE
  {\topsep}   % BELOWSPACE
  {\slshape}  % BODYFONT
  {0pt}       % INDENT (empty value is the same as 0pt)
  {\bfseries} % HEADFONT
  {.}         % HEADPUNCT
  {5pt plus 1pt minus 1pt} % HEADSPACE
  {}          % CUSTOM-HEAD-SPEC

\swapnumbers
\newtheorem{thm}{Theorem}[section]
\newtheorem{lem}[thm]{Lemma}
\newtheorem{prop}[thm]{Proposition}
\newtheorem{cor}[thm]{Corollary}
\newtheorem*{thm*}{Theorem}
\newtheorem*{lem*}{Lemma}
\newtheorem*{prop*}{Proposition}
\newtheorem*{cor*}{Corollary}

\theoremstyle{definition}
\newtheorem{defn}[thm]{Definition}
\newtheorem{example}[thm]{Example}
\newtheorem{examples}[thm]{Examples}
\newtheorem{rmk}[thm]{Remark}
\newtheorem{rmks}[thm]{Remarks}
\newtheorem{conv}[thm]{Convention}
\newtheorem*{defn*}{Definition}
\newtheorem*{example*}{Example}
\newtheorem*{examples*}{Examples}
\newtheorem*{rmk*}{Remark}
\newtheorem{rmks*}{Remarks}
\newtheorem*{conv*}{Convention}


%%% Define some convenient abbreviations for common mathematical
%%% notations.
\newcommand{\R}{\mathbb{R}} % use \R for the real numbers
\newcommand{\C}{\mathbb{C}} % use \C for the complex numbers
\newcommand{\Z}{\mathbb{Z}} % use \Z for the integers
\newcommand{\Q}{\mathbb{Q}} % use \Q for the rationals
\newcommand{\N}{\mathbb{N}} % use \N for the natural numbers
\newcommand{\F}{{\mathbb F}}
\newcommand{\compose}{\circ} % functional composition
\newcommand{\gen}[1]{\langle #1 \rangle}
\newcommand{\End}{\operatorname{End}}
\newcommand{\GL}{\mathrm{GL}}
\newcommand{\SL}{\mathrm{SL}}
\newcommand{\PSL}{\mathrm{PSL}}
\renewcommand{\O}{\mathrm{O}}
\newcommand{\Orth}{\mathrm{O}}
\newcommand{\SO}{\mathrm{SO}}
\newcommand{\U}{\mathrm{U}}
\newcommand{\SU}{\mathrm{SU}}
\newcommand{\g}{\mathfrak{g}}
\newcommand{\transpose}{\mathsf{T}}
\newcommand{\B}{\mathcal{B}}
\newcommand{\Rep}{\operatorname{Rep}}
\newcommand{\Mat}{\operatorname{Mat}}
\newcommand{\inner}[2]{\langle #1, #2 \rangle}
\newcommand{\sgn}{\operatorname{sgn}}
\newcommand{\n}{\underline{\mathbf{n}}}
\newcommand{\Sym}{\mathbb{S}}
\newcommand{\Alt}{\mathbb{A}}
\newcommand{\D}{\mathbb{D}}
\newenvironment{perm}[2]{\left(\begin{smallmatrix}#1 \\ #2}{\end{smallmatrix}\right)}
\newcommand{\lcm}{\operatorname{lcm}}
\newcommand{\res}{\operatorname{res}}
\newcommand{\im}{\operatorname{im}}
\newcommand{\normal}{\triangleleft\,}%better than \lhd
\newcommand{\morenormal}{\triangleright}

\allowdisplaybreaks
\parskip=2pt

%\title{Document Title}
%\author{author's name}

\begin{document}%\maketitle
\setcounter{section}{22}


\section{Simple groups}\noindent
The notion of a simple group was introduced by Galois. Around 1981, a
complete classification of the finite simple groups was obtained. This
was the culmination of many decades of effort of many researchers. The
classification theorem is regarded as one of the most important
achievements of twentieth century mathematics; see Appendix C for a
brief history of its proof.

Some of the theorems in this section will be stated without
proof. They are given here for the sake of general knowledge of the
subject, and will not be used in the sequel.


\begin{defn}[Galois]\index{simple~group}
A group $G$ is called {\em simple} if it has no normal subgroups other
than $\{1\}$ and $G$ itself.
\end{defn}

This means that $G$ is simple if and only if the only quotient groups
of $G$ are isomorphic to $\{1\}$ and $G$.

Notice that the definition of a simple group is analogous to the
definition of a prime number in the integers. A prime integer is one
which has no factors other than the trivial ones, and a simple group
is one which has no factor groups other than the trivial ones.


\begin{examples}  
  1. Every cyclic group\index{cyclic~group} of prime order is a
simple group. This is an easy consequence of Lagrange's theorem.

  2. Every alternating group
  $\Alt_n$ is simple, except for $\Alt_4$. This was discovered by
  Galois, and the proof is not easy.
\end{examples}

The following result, known as the \emph{correspondence theorem}, is
another basic theorem about homomorphisms and quotient groups.


\begin{thm}[correspondence theorem]\index{correspondence~theorem} 
\label{CT}% 
Let $K$ be any normal subgroup of a given group $G$ and let $\pi: G
\to G/K$ be the canonical homomorphism.  Then the mapping $H \mapsto
\pi(H)=H/K$ defines a bijective correspondence between the subgroups
of $G$ containing $K$ and the subgroups of $G/K$.  Furthermore, in
this correspondence $H \normal G$ if and only if $H/K \normal G/K$.
\end{thm}


\begin{proof}
This is an exercise in using the isomorphism theorems.
\end{proof}

As an application of the correspondence theorem, we describe a way of
``factoring'' a given finite group into a list of simple groups,
analogous to the way in which we factor a given positive integer
into its prime factors.  

Suppose that $G$ is a given finite group.  Let $N_1$ be a proper
normal subgroup of $G$ which is as large as possible.  Then by the
correspondence theorem $G/N_1$ has no nontrivial proper normal
subgroup, so $G/N_1$ is simple. Next, choose a proper normal subgroup
$N_2$ of $N_1$ that is as large as possible. Then as before, the
quotient $N_1/N_2$ must be simple. Continue in this way as long as
possible. Eventually you will arrive at a subgroup $N_{k-1}$ in which
the largest proper normal subgroup is the trivial subgroup $\{1\}$,
and the process terminates with $N_k = \{1\}$. We know that the
process must terminate since $G$ is finite.  

A series of normal subgroups such as the one we just described is
called a {\em composition series} of $G$. It describes a way in which
$G$ can be factored as a series of simple group quotients. This
explains the interest in simple groups: in some sense every finite
group can be described by a series of simple groups.

These observations lead to the following definition.


\begin{defn}\index{composition~series}
Let $G$ be a group. A {\em composition series} of $G$ is a sequence of
normal subgroups
\[
G = N_0 \morenormal N_1 \morenormal N_2 \morenormal \cdots \morenormal
N_{k-1} \morenormal N_k = \{1\}
\] 
where $N_j/N_{j+1}$ is a simple group for each $j$. The various
simple quotient groups $N_j/N_{j+1}$ are called the {\em composition
  factors} of $G$.
\end{defn}

The following important theorem says that the set of composition
factors of a finite group are uniquely determined by the group, apart
from the order in which they are produced. This is a fundamental
property of every finite group.


\begin{thm}[Jordan--H\"older]\index{Jordan--H\"older~theorem}
Any finite group has a composition series. Moreover, its composition
factors are unique, except for order and isomorphism.  In other words,
if
\[
G = N_0 \morenormal N_1 \morenormal N_2 \morenormal \cdots \morenormal
N_{m-1} \morenormal N_m = \{1\}
\]
and
\[
G = K_0 \morenormal K_1 \morenormal K_2 \morenormal \cdots \morenormal
K_{n-1} \morenormal K_n = \{1\}
\]
are any two composition series for $G$, then $m=n$ and there is a
permutation $\alpha \in \Sym_n$ such that $N_i/N_{i+1} \cong
K_{\alpha(i)}/K_{\alpha(i+1)}$ for each $i$.
\end{thm}

\begin{proof} The existence of the composition series was proved in the
remarks preceding the theorem. The proof of the uniqueness statement
is omitted, but can be easily found by consulting the literature.
\end{proof}

Recall that the fundamental theorem of arithmetic says that every
positive integer can be written as a product of primes, and the prime
factors are unique apart from their order. The Jordan--H\"{o}lder
theorem is somewhat analogous, it says that every finite group has a
composition series in which the composition factors are simple groups,
and the composition factors are unique apart from their order. In this
analogy, the simple groups are analogous to prime numbers.

The following definition is based on the fundamental work of Galois on
solutions of polynomial equations.

\begin{defn}\index{solvable~group}
A group $G$ is called {\em solvable} if it has a series of normal
subgroups
\[
G = N_0 \morenormal N_1 \morenormal N_2 \morenormal \cdots \morenormal
N_{k-1} \morenormal N_k = \{1\}
\] 
such that the quotient group $N_j/N_{j+1}$ is cyclic of prime order,
for each $j$.
\end{defn}

In other words, the composition factors of a solvable group are all
cyclic groups of prime order. Later we will prove that \emph{any
  finite abelian group is solvable}.


The reason for the terminology `solvable' is explained by the
following result, known as the \emph{fundamental theorem of Galois
  theory}. This is the famous result that connects group theory to the
ancient problem of solving polynomial equations. The proof is beyond
the scope of this course, and will not be discussed here.



\begin{thm}[Galois]\index{fundamental~theorem!of~Galois~theory}
Let $p(x)$ be an irreducible polynomial with rational coefficients,
and let $G = \mathrm{Gal}(p)$ be its Galois group. Then the complex
roots of $p(x)$ are expressible in terms of radicals if and only if
$G$ is a solvable group.
\end{thm}




The following famous result is usually known as the \emph{odd order
  theorem}. It was first proved in 1963 by Walter Feit and John
Thompson.

\begin{thm}[the Feit--Thompson odd order theorem]
\index{odd~order~theorem}\index{Feit--Thompson~theorem}% 
  Every finite group of odd order is solvable.
\end{thm}

This was a landmark result in group theory.  The published proof of
this theorem appeared in {\em Pacific Journal of Mathematics},
vol.\ 13, pp.\ 775--1029. Yes, that is correct: the proof occupies 255
printed pages! Perhaps we can be forgiven for not giving the proof here.




\section*{Exercises}
\begin{problems}

\item Use Lagrange's theorem to prove that every cyclic group of prime
  order is a simple group.

\item Compute the composition factors of $\Sym_2$ and $\Sym_3$. Are
  they solvable groups?

\item Show that $\Alt_3$ is simple. 

\item Compute the composition factors of $\Sym_4$. Is $\Sym_4$ a
  solvable group? Justify your answer.

\item Assuming that $\Alt_5$ is a simple group (this was proved by
  Galois) show that $\Sym_5$ is not a solvable group. 

\item Let $p$ be a prime. Show that the dihedral group $\D_p$ of order
  $2p$ is a solvable group, by computing its composition factors.

\item Find a composition series of $\D_4$, and compute its composition
  factors. Is $\D_4$ a solvable group? 


\item \label{ex:quintic} Galois proved that the alternating group
  $\Alt_5$ is simple. Galois also showed that the symmetry group of
  the general quintic equation (degree 5 polynomial with arbitrary
  variable coefficients) is $\Sym_5$. Assuming these facts, prove that
  the roots of a general quintic\index{quintic~equation} cannot be
  expressed in terms of radicals.

\item Prove the correspondence theorem (Theorem \ref{CT}).

\item Show that if $G$ is a group of prime power order $p^r$ for a
  prime $p$ and $r \ge 1$ then the composition factors of $G$ are all
  isomorphic to $\Z_p$.

\end{problems}


\newpage
%\appendix
\setcounter{section}{1}
\renewcommand{\thesection}{\Alph{section}}
\section{Appendix: Brief history of simple groups}\noindent
The notion of a simple group was introduced by Galois in 1831.  He
knew that the cyclic groups of prime order, and the alternating groups
$\Alt_n$ for $n\ge 5$, were examples of simple groups. Jordan found
some more simple groups (they were matrix groups) in 1870, and
L.E.\ Dickson in Chicago found some more in the years between 1892 and
1905. Around 1905 Miller and Cole showed that five groups described by
E.\ Mathieu in 1861 were in fact simple groups; these five, which did
not seem to fit in with other known examples, came to be known as {\em
  sporadic} groups.

No other examples of finite simple groups were discovered until the
1950s.  Then a French mathematician, Claude Chevalley, had an
important insight which led him to construct infinitely many new
examples of finite simple groups, using matrix groups as a tool in the
construction. Other examples quickly followed in the 1960s and 1970s,
including new sporadic groups. During those decades, research into
finite group theory was intense, and the focus was on simple
groups. Most of the new finite simple groups discovered during those
years are linear groups or variations thereof.

The study of simple groups has led to a truly colossal theorem: the
{\em classification} of all finite simple groups. This was supposedly
achieved around 1981.\footnote{Actually, although the theorem was
  announced in 1981 as finished, there was a gap in the proof, in that
  a classification of the so-called quasi-thin groups was never
  completely written down.  This gap was only very recently filled in,
  by the 2004 publication of a two-volume work of 1,221 pages by
  Aschbacher and Smith. (Stephen Smith is a professor at the
  University of Illinois at Chicago.)}  

\begin{thm}[the classification theorem]
\index{classification!of~finite~simple~groups}%
  Every finite simple group is isomorphic to one of the following
  groups:
 \begin{enumerate}
 \item A cyclic group of prime order.
 \item An alternating group of degree at least 5.
 \item A simple group of Lie type, including both 
   \begin{itemize}
   \item the classical Lie groups, namely the simple groups related to
     the projective special linear, unitary, symplectic, or orthogonal
     transformations over a finite field;
   \item the exceptional and twisted groups of Lie type (including the
     Tits group).
   \end{itemize}
 \item One of the 26 sporadic simple groups.
 \end{enumerate}
\end{thm}



The proof of this theorem took decades of effort on the part of
hundreds of mathematicians.  According to the preface of a
book\footnote{Gorenstein, Daniel; Lyons, Richard; Solomon, Ronald,
  \emph{The classification of the finite simple groups.}  Mathematical
  Surveys and Monographs, 40.1. American Mathematical Society,
  Providence, RI, 1994.} on the classification:
\begin{quote}
  The existing proof of the classification of the finite simple groups
  runs to somewhere between 10,000 and 15,000 journal pages, spread
  across some 500 separate articles by more than 100 mathematicians,
  almost all written between 1950 and the early 1980s.
\end{quote}
The proof followed a plan that was outlined by Daniel Gorenstein at a
group theory conference at the University of Chicago in 1972. At the
time, the experts were skeptical.  Some experts were on record
predicting that a complete classification of the finite simple groups
would take hundreds of years to achieve.  But a University of Chicago
Ph.D.\ student in the audience, Michael Aschbacher (now a professor at
Cal Tech), went to work on Gorenstein's plan, producing a series of
breakthroughs that eventually, with a lot of help from others, led to
the classification, following rather precisely the plan outlined by
Gorenstein. This took only nine years, and the experts were
astonished!




\begin{examples}[and remarks]
1. Every cyclic group of prime order is simple. This is an easy
consequence of Lagrange's theorem.

2. The alternating groups $\Alt_n$ for $n \ge 5$ are simple. This fact
was first proved by Galois, and the fact that $\Alt_5$ is simple is the
reason why the quintic equation is unsolvable in terms of radicals.

3. In the classification of finite simple groups, most of the simple
groups are linear groups (i.e., groups of matrices) over finite
fields.


4. There are 26 sporadic simple groups. The largest of these is known
as the \emph{Monster}\index{Monster~group}; it was discovered around
1980 by Robert Greiss at the University of Michigan. It has about
$8\times 10^{53}$ elements --- more than the number of atoms in the
universe!  The Monster is a group of rotations in a euclidean space of
196883 dimensions. In other words, we can represent its elements by
$n \times n$ matrices where $n=196883$.

5. In 1998 at the International Congress of Mathematicians in Berlin,
a Fields Medal\footnote{The Fields Medal has generally been regarded
  as the highest honor a mathematician can achieve, since there is no
  Nobel prize for mathematics.} was awarded to Richard Borcherds for
his work on settling the so-called {\em Monstrous Moonshine
  Conjecture} of Conway and Norton, which gives a connection between
the Monster group and a certain function which appears in conformal
field theory in physics. To accomplish this, Borcherds invented a
subject known as ``vertex operator algebras'' and in particular
defined a new Lie algebra known as the Monster Lie algebra. It turns
out that vertex operator algebras have applications to physics.
\end{examples}

Currently there are two teams of mathematicians working on a revision
of the proof of the classification theorem. Their stated goal is to
reduce the length of the proof to between 3,000 and 4,000 pages.



\renewcommand{\thesection}{\arabic{section}}
\end{document}
