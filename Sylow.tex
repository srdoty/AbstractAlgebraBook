\documentclass[11pt,oneside]{article}
\usepackage[nohead,margin=1.50in]{geometry} %set margins
\usepackage{amsmath,amssymb,amsthm,pdiag} %AMS packages for math stuff
\usepackage{multicol} % for use in HW section
\usepackage{enumitem}
  \setlist{topsep=1pt,itemsep=0pt,parsep=1pt}
  \setenumerate[1]{label=(\alph*)}
\usepackage{xr}
\externaldocument{actions}
\externaldocument{products}

\newenvironment{problems}
{
 \begin{enumerate}[topsep=1pt,itemsep=0pt,parsep=2pt,leftmargin=0.6cm,%
 label={\arabic*.}, ref=\arabic*] \small
}
{
 \end{enumerate}
}

%%% Define some theorem and example environments. The starred versions
%%% are un-numbered and the unstarred versions are numbered.
\newtheoremstyle{plain}
  {\topsep}   % ABOVESPACE
  {\topsep}   % BELOWSPACE
  {\slshape}  % BODYFONT
  {0pt}       % INDENT (empty value is the same as 0pt)
  {\bfseries} % HEADFONT
  {.}         % HEADPUNCT
  {5pt plus 1pt minus 1pt} % HEADSPACE
  {}          % CUSTOM-HEAD-SPEC

\swapnumbers
\newtheorem{thm}{Theorem}[section]
\newtheorem{lem}[thm]{Lemma}
\newtheorem{prop}[thm]{Proposition}
\newtheorem{cor}[thm]{Corollary}
\newtheorem*{thm*}{Theorem}
\newtheorem*{lem*}{Lemma}
\newtheorem*{prop*}{Proposition}
\newtheorem*{cor*}{Corollary}

\theoremstyle{definition}
\newtheorem{defn}[thm]{Definition}
\newtheorem{example}[thm]{Example}
\newtheorem{examples}[thm]{Examples}
\newtheorem{rmk}[thm]{Remark}
\newtheorem{rmks}[thm]{Remarks}
\newtheorem{conv}[thm]{Convention}
\newtheorem*{defn*}{Definition}
\newtheorem*{example*}{Example}
\newtheorem*{examples*}{Examples}
\newtheorem*{rmk*}{Remark}
\newtheorem{rmks*}{Remarks}
\newtheorem*{conv*}{Convention}


%%% Define some convenient abbreviations for common mathematical
%%% notations.
\newcommand{\R}{\mathbb{R}} % use \R for the real numbers
\newcommand{\C}{\mathbb{C}} % use \C for the complex numbers
\newcommand{\Z}{\mathbb{Z}} % use \Z for the integers
\newcommand{\Q}{\mathbb{Q}} % use \Q for the rationals
\newcommand{\N}{\mathbb{N}} % use \N for the natural numbers
\newcommand{\F}{{\mathbb F}}
\newcommand{\compose}{\circ} % functional composition
\newcommand{\gen}[1]{\langle #1 \rangle}
\newcommand{\End}{\operatorname{End}}
\newcommand{\GL}{\mathrm{GL}}
\newcommand{\SL}{\mathrm{SL}}
\renewcommand{\O}{\mathrm{O}}
\newcommand{\SO}{\mathrm{SO}}
\newcommand{\U}{\mathrm{U}}
\newcommand{\SU}{\mathrm{SU}}
\newcommand{\g}{\mathfrak{g}}
\newcommand{\transpose}{\mathsf{T}}
\newcommand{\B}{\mathcal{B}}
\newcommand{\Rep}{\operatorname{Rep}}
\newcommand{\Mat}{\operatorname{Mat}}
\newcommand{\inner}[2]{\langle #1, #2 \rangle}
\newcommand{\sgn}{\operatorname{sgn}}
\newcommand{\n}{\underline{\mathbf{n}}}
\newcommand{\Sym}{\mathbb{S}}
\newcommand{\Alt}{\mathbb{A}}
\newcommand{\D}{\mathbb{D}}
\newenvironment{perm}[2]{\left(\begin{smallmatrix}#1 \\ #2}{\end{smallmatrix}\right)}
\newcommand{\lcm}{\operatorname{lcm}}
\newcommand{\res}{\operatorname{res}}
\newcommand{\im}{\operatorname{im}}
\newcommand{\ptn}{\mathfrak{p}}

\allowdisplaybreaks
\parskip=2pt

%\title{Document Title}
%\author{author's name}

\begin{document}%\maketitle
\setcounter{section}{27}



\section{The Sylow Theorems}\noindent
It is rather remarkable that three of the most important general
theorems about finite groups were proved by a high school teacher. He
was Ludwig Sylow, in Norway, and he proved his famous theorems in a
ten page paper published in 1872. If $p^r$ is the \emph{largest} power
of a prime $p$ dividing the order of $G$, then Sylow showed:
\begin{enumerate}[label=(\roman*)]
\item $G$ has at least one subgroup of order $p^r$;
\item any two such subgroups are conjugate;
\item $G$ has $1+kp$ such subgroups, for some non-negative integer $k$.

\end{enumerate}
All of these statements are proved using the theory of group actions.
Let us look at more precise statements of these facts. First we need
some additional terminology. 


\begin{defn}\index{p@$p$-subgroup}\index{Sylow~$p$-subgroup}
Let $p$ be a prime divisor of the order of a finite group $G$.  A
$p$-\emph{subgroup} of $G$ is any subgroup whose order is a power of
$p$.  If $p^r$ is the highest power of $p$ dividing the order of $G$,
then any subgroup of order $p^r$ is called a {\em Sylow $p$-subgroup}
of $G$.
\end{defn}

Note that any Sylow $p$-subgroup is also a $p$-subgroup, but not vice
versa. 

\begin{example}
Suppose $|G| = 250 = 2 \cdot 5^3$. The prime divisors of $G$ are just
$p=2$ and $p=5$. Then the $2$-subgroups of $G$ are just the subgroups
of order $2$, and they are also Sylow $2$-subgroups. The $5$-subgroups
of $G$ are the subgroups of order $5$, $25 = 5^2$, and $125 =
5^3$. The subgroups of order 125 are the Sylow 5-subgroups.
\end{example}



\begin{thm}[First Sylow Theorem]\index{Sylow~theorem!first}
Let $G$ be a finite group and $p$ a prime divisor of $|G|$. Let $p^r$
be the highest power of $p$ dividing $|G|$.  Then for every integer
$s$ satisfying $0\le s \le r$ there exists a subgroup of $G$ of order
$p^s$. In particular, a Sylow $p$-subgroup of $G$ (of order $p^r$)
must exist.
\end{thm}

\begin{proof}
Proceed by induction on the order of $G$. If $G$ has order 1 there is
nothing to prove. So assume $|G| > 1$ and assume inductively that the
theorem has been proved for all groups of order less than $|G|$. By
the inductive hypothesis: If $G$ has a proper subgroup $H$ such that
$p^s$ divides the order of $H$, then $H$ has a subgroup of order
$p^s$, and hence so does $G$.

Now we apply the class equation (Theorem \ref{classeqn}) for $G$,
which states that
\[
   \textstyle |G|=|Z(G)| + \sum [G:Z_G(x_j)]
\]
where the $x_j$ range over a complete set of representatives of the
conjugacy classes of $G$ not contained in the center $Z(G)$. Each
$Z_G(x_j)$ is a proper subgroup of $G$ (or else $x_j$ would be in the
center) so by the inductive hypothesis, if $p^s$ divides the order of
any $Z_G(x_j)$ then $G$ contains a subgroup of order $p^s$ and we are
done.

The remaining case is that $p^s$ does not divide the order of any
$Z_G(x_j)$. Then $p$ divides each index $[G:Z_G(x_j)]$, so from the
class equation we conclude that $p$ must divide $|Z(G)|$. But $Z(G)$
is abelian, so this means (by a standard lemma, proved in Lemma
\ref{Cauchy-abelian} below) that the center $Z(G)$ has an element $a$ of
order $p$. The subgroup $P = \gen{a}$ generated by $a$ has order $p$,
so we are done if $s=1$.

So assume that $s>1$. Observe that $P$ is a normal subgroup of $G$
since $P$ is contained in the center $Z(G)$. Hence the quotient group
$G/P$ is defined. The order of $G/P$ is $(p^rm)/p = p^{r-1}m$, for
some integer $m$, so by the inductive hypothesis $G/P$ has a subgroup
$H$ of order $p^{s-1}$. By the correspondence theorem this subgroup
$H$ corresponds with a subgroup $H'$ of $G$ containing $P$, such that
$H'/P \simeq H$. Thus $|H'| = p^s$ and we are done.
\end{proof}


In order to complete the proof of the first Sylow theorem, we need the
following simple result, which amounts to Cauchy's theorem (see
Theorem \ref{Cauchy}) in the abelian case.

\begin{lem} \label{Cauchy-abelian} 
Let $G$ be a finite abelian group and let $p$ be a prime divisor of
$|G|$. Then $G$ has an element of order $p$.
\end{lem}

\begin{proof}
By induction on the order of $G$. If $|G|=1$ there is nothing to
prove. Assume that $|G|>1$ and that the theorem has been proved
already for all abelian groups of order less than $|G|$. If $G$ has no 
proper subgroup (other than $\{1\}$) then $G$ must be cyclic and the
statement of the theorem is easy to see. 

The remaining case is that $G$ has a proper subgroup $H \ne |{1|}$.
If $p$ divides $|H|$ then we are done by the inductive hypothesis. So
assume that $p$ does not divide $|H|$. Since $G$ is abelian, $H
\triangleleft G$. So $G/H$ is an abelian group of order $|G|/|H|$.
Now $p$ must divide $|G/H|$ since $p$ divides $|G|$ but not $|H|$.
Moreover, $|G/H| < |G|$. Thus, by the inductive hypothesis, $G/H$ has
an element $aH$ of order $p$. (The elements of $G/H$ are left cosets,
by definition.) Let $b=a^k$ where $k = |H|$. Then one can check that
the order of $b$ is $p$, since $(aH)^p = H$ implies that $a^p \in H$,
so by a corollary to Lagrange's theorem $(a^p)^k = 1$, and thus $b^p =
a^{kp} = 1$. This completes the proof, as no smaller positive power of
$b$ can be 1.
\end{proof}


\begin{thm}[Second Sylow Theorem]\index{Sylow~theorem!second}
Let $G$ be a finite group and $p$ a prime divisor of $|G|$. If $Q$ is
a $p$-subgroup of $G$ and $P$ is any Sylow $p$-subgroup of $G$, then
$Q$ must be contained in some conjugate of $P$. In particular, any two
Sylow $p$-subgroups of $G$ must be conjugate.
\end{thm}


\begin{thm}[Third Sylow Theorem]\index{Sylow~theorem!third}
Let $G$ be a finite group and $p$ a prime divisor of $|G|$.  Write
$|G|=p^r m$ where $m$ is not divisible by $p$. If $n_p$ is the number
of Sylow $p$-subgroups of $G$, then:
\begin{enumerate}
\item[(a)] $n_p \equiv 1 \pmod{p}$; and

\item[(b)] $n_p$ divides $m$.
\end{enumerate}
\end{thm} 



Proofs of the second and third Sylow theorems can be found in
virtually any text on abstract algebra, so they are not reproduced
here.

Our first application of Sylow's theorems generalizes Lemma
\ref{Cauchy-abelian} to include the non-abelian case.


\begin{thm}[Cauchy's Theorem]\index{Cauchy's~theorem}\label{Cauchy} 
Let $p$ be a prime divisor of the order of a finite group $G$. Then
$G$ must have an element of order $p$.
\end{thm}

\begin{proof}
By the first Sylow theorem, $G$ has a subgroup of order $p$. Any
element of that subgroup (except the identity) must have order $p$, by
Lagrange's theorem. 
\end{proof}

Notice how easy the proof is! This shows the power of the Sylow
theorems. Note carefully, however, that we needed an independent proof
of Lemma \ref{Cauchy-abelian} in the abelian case, since it is used
in the proof of the first Sylow theorem.  

Recall that we have earlier defined a $p$-group (where $p$ is a prime)
to be any group in which the order of every element is a power of $p$.
It follows (see Exercise \ref{exer:p-gp}) from Lagrange's theorem and
Cauchy's theorem that the order of any finite $p$-group must be of the
form $p^r$ for some $r \ge 1$.

Here is another application of Sylow's theorems.

\begin{prop}[the $pq$ theorem]\index{pq@$pq$-theorem} \label{pq}
Let $|G| = pq$ where $p<q$ and $p,q$ are primes. If $p$ does not
divide $q-1$ then $G$ is cyclic.
\end{prop}


\begin{proof}
Let $H$ be a Sylow $p$-subgroup of $G$ and let $K$ be a Sylow
$q$-subgroup of $G$. By the third Sylow theorem, the number of Sylow
$p$-subgroups has the form $1+kp$ and divides $q$. If this number is
$q$ then $p$ divides $q-1$, a contradiction. Thus the number of Sylow
$p$-subgroups must be 1, so $H\triangleleft G$.

Similarly, one can show there is only one Sylow $q$-subgroup, so $K
\triangleleft G$.  Now $H \cap K = \{1\}$ by Lagrange's theorem, 
so
\[
  |HK| = |H|\cdot|K|/|H\cap K| = pq 
\]
which proves that $HK=G$. Hence $G = H \times K$ is the direct product
of its two Sylow subgroups. Since direct products of abelian groups
are abelian, this shows that $G$ is abelian. In fact, if $x$ is a
generator of the cyclic group $H$ and if $y$ is any generator of the
cyclic group $K$ then $x$ and $y$ commute and the order of $xy$ is the
least common multiple of $p$ and $q$, which is $pq$, so $xy$ has order
$pq$. Thus $xy$ generates $G$, so $G$ is cyclic.
\end{proof}

The last result shows that, up to isomorphism, there is just one group
(which must be the cyclic group) of order $pq$, where $p<q$ are two
distinct primes such that $q \not\equiv 1 \pmod{p}$. So up to
isomorphism we have just one group of order 15, one of order 33, one
of order 35, etc.

\begin{example}
We apply the Sylow theorems to classify groups of order 21. Assume
that $|G|=21 = 3 \cdot 7$. Note that the $pq$-theorem does not apply
since $7$ is congruent to $1$ mod $3$. Let $n_7$ be the number of
Sylow $7$-subgroups of $G$. By the third Sylow theorem,
\[
   n_7 \equiv 1 \pmod{7} \quad\text{and}\quad n_7 \mid 3.
\]
The positive divisors of $3$ are $1, 3$ but only $1$ is congruent to
$1$ mod $3$, so $n_7 = 1$. So there is just one subgroup $H$ of order
7. By the second Sylow theorem, $H$ is stable under conjugation and
hence normal. We have proved that any group of order 21 must have a
normal subgroup of order 7. Such a group is not simple, so at this
point we also know that there is no simple group of order 21.

We can be even more precise. Let $n_3$ be the number of Sylow
$3$-subgroups. Then the third Sylow theorem implies that $n_3 = 1$ or
$7$. Let $x \in G$ be a generator of $H$. Let $y \in G$ be an element
of order 3; such an element must exist by Cauchy's theorem. Note that
$H = \gen{y}$ is a Sylow 3-subgroup. We have $x^7 = 1$, $y^3 = 1$, and
(since $H$ is normal) $yxy^{-1} \in H$. Since $H = \gen{x} = \{1, x,
x^2, \dots, x^6\}$ is cyclic, it follows that $yxy^{-1} = x^k$ for
some positive integer $k<7$.

What are the possibilities for $k$? The fact that $y^3=1$ provides
information on this question, as follows:
\begin{align*}
  x &= y^3xy^{-3} = y^2 (yxy^{-1}) y^{-2} = y^2 x^k y^{-2} = y (y x^k
  y^{-1}) y^{-1} \\ &= y (y x y^{-1})^k y^{-1} = y (x^k)^k y^{-1} = y
  x^{k^2} y^{-1}.
\end{align*}
Thus $x = y x^{k^2} y^{-1} = (yxy^{-1})^{k^2} = (x^k)^{k^2} =
x^{k^3}$. Since $x$ has order 7, it follows that $k^3$ must be
congruent to 1 mod 7, so $k = 1, 2$, or 4. We consider these cases
below. Note that $G = \gen{x,y}$ is generated by $x, y$ by Proposition
\ref{prop:counting-HK}.

Case 1. If $k=1$ then $yxy^{-1} = x$; i.e., $xy=yx$. Since $x,y$ generate
$G$ as noted above, this means that $G$ must be abelian. Furthermore,
this implies that $K$ is normal in $G$, so in fact $G = H \times K$ is
the direct product of $H$ and $K$. This implies that $G \cong Z_7
\times \Z_3 \cong \Z_{21}$ is itself cyclic.

Case 2. If $k=2$ then $yxy^{-1} = x^2$; i.e., $yx = x^2y$. The relations
$x^7 =1$, $y^3 = 1$, and $yx = x^2y$ actually determine $G$
uniquely. The existence of this group can be checked with a little
more work. It is clearly not abelian.

Case 3. If $k=2$ then $yxy^{-1} = x^4$. In this case $y^2 x y^{-2} = y
x^4 y^{-1} = x^{16} = x^2$. Thus if we replace $y$ by $y^{2}$, which
also generates $K$, then we are back in the previous case.

This analysis proves that, up to isomorphism, there are just two
groups of order 21, just one of which is abelian (the cyclic group of
order 21).
\end{example}


\begin{table}[h]
\begin{center}
\begin{tabular}{cc|cc|cc}
  $n$ & number & $n$ & number & $n$ & number\\ \hline
  1 & 1 & 9 & 2 & 17 & 1 \\
  2 & 1 & 10 & 2 & 18 & 5 \\
  3 & 1 & 11 & 1 & 19 & 1\\
  4 & 2 & 12 & 5 & 20 & 5\\
  5 & 1 & 13 & 1 & 21 & 2\\
  6 & 2 & 14 & 2 & 22 & 2\\
  7 & 1 & 15 & 1 & 23 & 1\\
  8 & 5 & 16 & 14 & 24 & 15
\end{tabular}
\end{center}
\caption{The number of groups of order 1--24, up to isomorphism}
\end{table}
We finish this section by displaying a brief table of the number of
groups, up to isomorphism, of order up to 24. This can be justified by
the Sylow theorems in conjunction with other results we have proved,
but the task is not easy.  You can find much more extensive tables
online.




\section*{Exercises}
\begin{problems}

\item Show that the element $b$ in the proof of \ref{Cauchy-abelian}
has order $p$.

\item Let $p$ be a given prime number. Prove that if $G$ is a finite
group with just one Sylow $p$-subgroup, then that subgroup must be
normal.

\item Show that there is only one group of order $33$, up to
  isomorphism. (In other words, every group of order 33 is isomorphic
  to $\Z_{33}$.)

\item\index{p@$p$-group}\label{exer:p-gp} Prove that if $G$ is a
  finite group in which every element has order some power of $p$
  (where $p$ is prime) then $|G| = p^r$ for some $r \ge 1$. (Such
  groups are called $p$-groups.)

\item Prove that no group of order $pq$, where $p,q$ are primes, is
  simple. (Consider the possibility $p=q$ as well as $p\ne q$.)

\item Use the Sylow theorems to prove that there is no simple group
of order $30$.

\item Use the Sylow theorems to prove that there is no simple group
of order $56$.

\item Show that any quotient of a solvable group must be solvable.


\item Prove that if $G/Z(G)$ is cyclic then $G$ must be abelian.

\item Show that if $|G| = 2p$ where $p$ is an odd prime then $G$ must
  be isomorphic to either $\Z_{2p}$ or $\D_{p}$.

\item Show that if $|G| = 12$ then either $G \cong \Alt_4$ or $G$ has
  a normal subgroup of order $3$. [Hint: If $G$ does not have a normal
    subgroup of order 3, consider the action of $G$ on the set $G/H$
    by left multiplication, where $H$ is one of the Sylow
    3-subgroups. This action induces a homomorphism from $G$ into
    $\Sym_4$.]
\end{problems}



\newpage
\section{Simplicity of $\Alt_n$}\noindent
Our final application of the theory of group actions will be to prove
the theorem (due to Galois) that the alternating groups $\Alt_n$ are
all simple, except for $\Alt_4$.  This is the key result that shows
that the general polynomial of degree 5 or higher is not solvable in
terms of radicals.

We begin with an analysis of the conjugacy classes of the symmetric
group $\Sym_n$. We need to examine $\Sym_n$ because Galois proved that
$\Sym_n$ is the symmetry group associated to a general polynomial of
degree $n$.


\begin{example} \label{CC:S5}
Let's start by looking at the example of $\Sym_5$. Here is a list of
all 120 elements of $\Sym_5$, produced by the GAP\footnote{The GAP Group
  (\texttt{http://www.gap-system.org}).} software package: \tiny
\begin{verbatim}
gap> G := SymmetricGroup(5);
Sym( [ 1 .. 5 ] )
gap> List(G);
[ (), (1,5), (1,2,5), (1,3,5), (1,4,5), (2,5), (1,5,2), (1,2), (1,3,5,2), 
  (1,4,5,2), (2,3,5), (1,5,2,3), (1,2,3), (1,3)(2,5), (1,4,5,2,3), (2,4,5), 
  (1,5,2,4), (1,2,4), (1,3,5,2,4), (1,4)(2,5), (3,5), (1,5,3), (1,2,5,3), 
  (1,3), (1,4,5,3), (2,5,3), (1,5,3,2), (1,2)(3,5), (1,3,2), (1,4,5,3,2), 
  (2,3), (1,5)(2,3), (1,2,3,5), (1,3,2,5), (1,4,5)(2,3), (2,4,5,3), 
  (1,5,3,2,4), (1,2,4)(3,5), (1,3,2,4), (1,4)(2,5,3), (3,4,5), (1,5,3,4), 
  (1,2,5,3,4), (1,3,4), (1,4)(3,5), (2,5,3,4), (1,5,3,4,2), (1,2)(3,4,5), 
  (1,3,4,2), (1,4,2)(3,5), (2,3,4), (1,5)(2,3,4), (1,2,3,4,5), (1,3,4,2,5), 
  (1,4,2,3,5), (2,4)(3,5), (1,5,3)(2,4), (1,2,4,5,3), (1,3)(2,4), 
  (1,4,2,5,3), (4,5), (1,5,4), (1,2,5,4), (1,3,5,4), (1,4), (2,5,4), 
  (1,5,4,2), (1,2)(4,5), (1,3,5,4,2), (1,4,2), (2,3,5,4), (1,5,4,2,3), 
  (1,2,3)(4,5), (1,3)(2,5,4), (1,4,2,3), (2,4), (1,5)(2,4), (1,2,4,5), 
  (1,3,5)(2,4), (1,4,2,5), (3,5,4), (1,5,4,3), (1,2,5,4,3), (1,3)(4,5), 
  (1,4,3), (2,5,4,3), (1,5,4,3,2), (1,2)(3,5,4), (1,3,2)(4,5), (1,4,3,2), 
  (2,3)(4,5), (1,5,4)(2,3), (1,2,3,5,4), (1,3,2,5,4), (1,4)(2,3), (2,4,3), 
  (1,5)(2,4,3), (1,2,4,3,5), (1,3,2,4,5), (1,4,3,2,5), (3,4), (1,5)(3,4), 
  (1,2,5)(3,4), (1,3,4,5), (1,4,3,5), (2,5)(3,4), (1,5,2)(3,4), (1,2)(3,4), 
  (1,3,4,5,2), (1,4,3,5,2), (2,3,4,5), (1,5,2,3,4), (1,2,3,4), (1,3,4)(2,5), 
  (1,4)(2,3,5), (2,4,3,5), (1,5,2,4,3), (1,2,4,3), (1,3)(2,4,5), (1,4,3)(2,5) 
 ]
\end{verbatim}\normalsize
As you can see, the list is utter chaos. How can we impose any order
on this chaos, and make sense of the list? Let's ask GAP to compute
the elements of each conjugacy class, and see what happens: \tiny
\begin{verbatim}
gap> CC := ConjugacyClasses(G);
[ ()^G, (1,2)^G, (1,2)(3,4)^G, (1,2,3)^G, (1,2,3)(4,5)^G, (1,2,3,4)^G, 
  (1,2,3,4,5)^G ]
gap> List( CC[1] );
[ () ]
gap> List( CC[2] );
[ (1,2), (1,3), (1,4), (1,5), (2,3), (2,4), (2,5), (3,4), (3,5), (4,5) ]
gap> List( CC[3] );
[ (1,2)(3,4), (1,3)(2,4), (1,4)(2,3), (2,5)(3,4), (2,4)(3,5), (2,3)(4,5), 
  (1,5)(3,4), (1,4)(3,5), (1,3)(4,5), (1,4)(2,5), (1,5)(2,4), (1,2)(4,5), 
  (1,3)(2,5), (1,2)(3,5), (1,5)(2,3) ]
gap> List( CC[4] );
[ (1,2,3), (1,2,4), (1,2,5), (1,3,2), (1,3,4), (1,3,5), (1,4,2), (1,4,3), 
  (1,4,5), (1,5,2), (1,5,3), (1,5,4), (2,3,4), (2,3,5), (2,4,3), (2,4,5), 
  (2,5,3), (2,5,4), (3,4,5), (3,5,4) ]
gap> List( CC[5] );
[ (1,2,3)(4,5), (1,2,4)(3,5), (1,2,5)(3,4), (1,3,2)(4,5), (1,3,4)(2,5), 
  (1,3,5)(2,4), (1,4,2)(3,5), (1,4,3)(2,5), (1,4,5)(2,3), (1,5,2)(3,4), 
  (1,5,3)(2,4), (1,5,4)(2,3), (1,5)(2,3,4), (1,4)(2,3,5), (1,5)(2,4,3), 
  (1,3)(2,4,5), (1,4)(2,5,3), (1,3)(2,5,4), (1,2)(3,4,5), (1,2)(3,5,4) ]
gap> List( CC[6] );
[ (1,2,3,4), (1,2,4,3), (1,3,2,4), (1,3,4,2), (1,4,2,3), (1,4,3,2), 
  (2,3,4,5), (2,4,3,5), (2,4,5,3), (2,5,3,4), (2,3,5,4), (2,5,4,3), 
  (1,3,4,5), (1,4,3,5), (1,4,5,3), (1,5,3,4), (1,3,5,4), (1,5,4,3), 
  (1,4,5,2), (1,5,2,4), (1,2,4,5), (1,4,2,5), (1,5,4,2), (1,2,5,4), 
  (1,5,2,3), (1,3,5,2), (1,5,3,2), (1,2,5,3), (1,2,3,5), (1,3,2,5) ]
gap> List( CC[7] );
[ (1,2,3,4,5), (1,2,3,5,4), (1,2,4,3,5), (1,2,4,5,3), (1,2,5,3,4), 
  (1,2,5,4,3), (1,3,2,4,5), (1,3,2,5,4), (1,3,4,2,5), (1,3,4,5,2), 
  (1,3,5,2,4), (1,3,5,4,2), (1,4,2,3,5), (1,4,2,5,3), (1,4,3,2,5), 
  (1,4,3,5,2), (1,4,5,2,3), (1,4,5,3,2), (1,5,2,3,4), (1,5,2,4,3), 
  (1,5,3,2,4), (1,5,3,4,2), (1,5,4,2,3), (1,5,4,3,2) ]
\end{verbatim}\normalsize
Ah, that's much better. Some order appears in the chaos. It seems that
conjugacy classes might be a nice way to organize the elements in a
large group. Let's analyze the results of the above computer
calculation. What do you notice about each conjugacy class? A quick
look reveals that each class consists of all elements that have the
same {\em cycle type}. Each element of the second class is a 2-cycle,
each element of the third is a product of two 2-cycles, and each
element of the sixth class is a 4-cycle, and so forth.
\end{example}

There is obviously a nice theorem here. Let's formulate and prove it.
First, we need the notion of a partition of $n$, which is an important
concept in combinatorics.

\begin{defn}\index{partition}
  Let $n$ be a given positive integer. A \emph{partition} of $n$ is by
  definition any set $\{\lambda_1, \lambda_2, \dots, \lambda_k\}$ of
  positive integers that add up to $n$. 
\end{defn}

By standard convention, we will write partitions as ordered $k$-tuples
of the form $\lambda = (\lambda_1, \lambda_2, \dots, \lambda_k)$ where
the numbers are ordered from biggest to smallest. With this
convention, the partitions of $n$ for the first few values of $n$ are
displayed in the table below:
\begin{center}\small
\begin{tabular}{|c|l|}\hline
  $n$ & partitions of $n$ \\ \hline
  $1$ & $(1)$\\
  $2$ & $(2)$, $(1^2)$\\
  $3$ & $(3)$, $(2,1)$, $(1^3)$\\
  $4$ & $(4)$, $(3,1)$, $(2^2)$, $(2,1^2)$, $(1^4)$\\
  $5$ & $(5)$, $(4,1)$, $(3,2)$, $(3,1^2)$, $(2^2,1)$, $(2,1^3)$,
        $(1^5)$\\ 
  $6$ & $(6)$, $(5,1)$, $(4,2)$, $(4,1^2)$, $(3^2)$, $(3,2,1)$, 
        $(3,1^3)$, $(2^3)$, $(2^2, 1^2)$, $(2,1^4)$, $(1^6)$\\
\hline
\end{tabular}
\end{center}
In the table, we used the notational trick of writing $a^k$ in place
of $k$ repeated values of $a$. Thus, for instance, $(1^5) =
(1,1,1,1,1)$. This useful shorthand is commonly used by people dealing
with partitions.

Let us display another table, indicating not the partitions
themselves but just counting their number. In fact, the number of
partitions of a given $n$ is usually denoted as $\ptn(n)$, and the
function $\ptn$ is called the \emph{partition function}. 
\begin{center}\small
\begin{tabular}{|c|c|c|c|c|c|c|c|c|c|c|c|c|c|c|c|}\hline
  $n$ & 1&2&3&4&5&6&7&8&9&10&11&12&13&14&15\\ \hline
  $\ptn()n)$ &1&2&3&5&7&11&15&22&30&42&56&77&101&135&176\\
  \hline
\end{tabular}
\end{center}
An interesting (and infamous) open problem in mathematics is to find a
finite closed formula for the number $\ptn()n)$ of partitions of
$n$. Nobody knows how to do this, and it would be astonishing if
someone solved it at this point.

Now that we know about partitions, we can return to the problem of
describing the conjugacy classes in the symmetric group $\Sym_n$.


Let $\alpha$ be a permutation in $\Sym_n$. We know how to write
$\alpha$ as a product of disjoint cycles. For convenience, let's do
this in such a way that every number from 1 to $n$ actually appears,
by inserting 1-cycles for all the fixed points. (Recall that fixed
points are usually omitted in the notation.) For example, if $\alpha
= \begin{perm}{1&2&3&4&5&6&7}{1&7&2&6&5&4&3} \end{perm}$ then we have
$\alpha = (1)(2,7,3)(4,6)(5) = (2,7,3)(4,6)(1)(5)$. Recall that
disjoint cycles commute, so we can reorder the factors any way we
like; we have here chosen to write the product of cycles in decreasing
order by cycle length.  Let us adopt this ordering convention: in this
section  we will always write the product of cycles in order of
decreasing cycle length, from longest to shortest. With this
convention, the cycle type of a permutation $\alpha \in \Sym_n$ is a
partition of $n$.


\begin{defn}\index{cycle~type}
The \emph{cycle type} of a permutation $\alpha$ is the partition
$\lambda = (\lambda_1, \lambda_2, \dots, \lambda_k)$ of lengths of
each cycle in the product.
\end{defn}

For example, if $\alpha = (2,7,3)(4,6)(1)(5)$ as above then the cycle
type of $\alpha$ is the partition $(3,2,1,1) = (3,2,1^2)$.  For
another example, if $\alpha = (7,3,5)(2,1,4)(8,6,10)(9) \in \Sym_{10}$
then the cycle type of $\alpha$ is the partition $(3,3,3,1) =
(3^3,1)$. 

What is the cycle type of the identity permutation in $\Sym_n$? It has
exactly $n$ fixed points, so in our convention it must be written as a
product of $n$ 1-cycles. Thus the cycle type of the identity
permutation in $\Sym_n$ is the partition $(1^n) = (1,1,\dots, 1)$.


The cycle type of any permutation in $\Sym_n$ is a partition of
$n$. For every partition of $n$, there exists a permutation $\alpha
\in \Sym_n$ with that cycle type.

\begin{example}
The various cycle types for permutations in $\Sym_5$ are just the
partitions of 5: $(5)$, $(4,1)$, $(3,2)$, $(3,1^2)$, $(2^2,1)$,
$(2,1^3)$, and $(1^5)$. Notice that there are seven partitions of 5,
and in Example \ref{CC:S5} we found exactly seven conjugacy classes in
$\Sym_5$. This observation suggests the following general result.
\end{example}


\begin{thm}\index{conjugacy~classes~of~$\Sym_n$} \label{cycle-type}
Two permutations in $\Sym_n$ are conjugate if and only if they have
the same cycle type. Hence, the partitions of $n$ label the conjugacy
classes of $\Sym_n$. The number of distinct conjugacy classes is the
same as the number $\ptn()n)$ of partitions of $n$.
\end{thm}


\begin{proof}
We only need to check the first claim, since all the other claims
follow from it immediately.

$(\Rightarrow)$: Assume that $\alpha, \beta \in \Sym_n$ are conjugate
permutations.  This means that there exists some permutation $\gamma
\in \Sym_n$ such that $\beta = \gamma \alpha \gamma^{-1}$. Now I claim
that if
\[
   \alpha = (i_1,\dots, i_r)(j_1,\dots,j_s)(\dots)
\]
is the disjoint cycle decomposition of $\alpha$ then the disjoint
cycle decomposition of $\beta = \gamma \alpha \gamma^{-1}$ is 
\[
  \beta = (i_1\gamma,\dots, i_r\gamma)(j_1\gamma,\dots,j_s\gamma)(\dots)
\]
and this proves that $\alpha$ and $\beta$ have the same cycle type, as
desired.

$(\Leftarrow)$: Conversely, if $\alpha, \beta$ have the same cycle
type, say
\begin{align*}
\alpha &= (i_1,\dots, i_r)(j_1,\dots,j_s)(\dots) \\
\beta &= (k_1,\dots, k_r)(m_1,\dots,m_s)(\dots)
\end{align*}
then define $\gamma$ to be the permutation sending $i_1 \to k_1$,
\dots, $i_r \to k_r$, $j_1 \to m_1$, \dots, $j_s \to m_s$, and so on.
A calculation shows that $\beta = \gamma\alpha \gamma^{-1}$, so
$\alpha, \beta$ are conjugate. This completes the proof.
\end{proof}

\begin{example}
We already calculated the conjugacy classes of $\Sym_5$ in Example
\ref{CC:S5}. Counting the number of elements in each class gives us
the following tabulation for the class equation of $\Sym_5$:
\[
  120 = 1+10+15+20+20+30+24.
\]
(This is just the fact that $G$ is the union of its conjugacy classes;
see \ref{orbit-union}.) Note that the numbers on the right of the
class equation all divide $|G|$. This is no accident, because of the
fundamental orbit-stabilizer theorem (see \ref{osr} and
\ref{conj-elt}), which says that the size of an orbit $C(x)$ is the
same as the index $[G:Z_G(x)]$, which divides the group order $|G|$ by
Lagrange's theorem.
\end{example}

\begin{example}\label{ex:A5}
Now let's fire up GAP to compute the conjugacy class orders in the
alternating group $\Alt_5$:  \tiny
\begin{verbatim}
gap> G := AlternatingGroup(5);
Alt( [ 1 .. 5 ] )
gap> CC := ConjugacyClasses(G);
[ ()^G, (1,2)(3,4)^G, (1,2,3)^G, (1,2,3,4,5)^G, (1,2,3,5,4)^G ]
gap> Size( CC[1] ); Size( CC[2] ); Size( CC[3] ); Size( CC[4] ); Size( CC[5] );
1
15
20
12
12
\end{verbatim}\normalsize
As we see, there are just five classes in $\Alt_5$, and their cycle
types are given by the partitions $(1^5)$, $(2^2,1)$, $(3,1^2)$,
$(5)$, and $(5)$. This is an interesting observation, because it shows
that there are two classes in $\Alt_5$ of the same cycle type. (Note
that Theorem \ref{cycle-type}, which is a theorem about symmetric
groups, fails for alternating groups.) The GAP calculation above gives
us the following for the class equation of $\Alt_5$:
\[
  60 = 1+15+20+12+12.
\]
This calculation is going to be very useful in proving that $\Alt_5$
is a simple group, the main goal of this section.

Before leaving this example, notice that there is only one 1 in each
of the class equations considered here. This proves that the center of
$\Sym_5$ and the center of $\Alt_5$ are trivial, since $x \in Z(G)
\Leftrightarrow Z_G(x)=G \Leftrightarrow |C(x)|=1$. 
\end{example}

We have the following general observation.

\begin{lem} \label{normal-lem}
Let $N$ be a normal subgroup of a group $G$. If $x \in N$ then $C(x)
\subset N$. Thus $N$ is a union of certain conjugacy classes of $G$.
\end{lem}

\begin{proof}
This follows immediately from the definitions. If $N$ is normal then
$N$ is closed under conjugation, so of course $N$ contains all
conjugates of any of its elements. 
\end{proof}

We are finally ready to prove the following result, originally proved
by Galois. This is the culmination of all the information developed in
this section.

\begin{thm}\index{simplicity~of~$\Alt_5$}[Galois] 
The alternating group $\Alt_5$ is a simple group. 
\end{thm}

\begin{proof}
Suppose that $\Alt_5$ has a proper normal subgroup $N$. So the order
$|N|$ satisfies $1 < |N| < 60$.  Also, $|N|$ must be a divisor of 60,
by Lagrange's theorem, so $|N| = $ 2, 3, 4, 5, 6, 10, 12, 15, 20, or
30. By the preceding lemma, $N$ is a union of conjugacy classes, so
$|N|$ is a sum of the numbers 1, 15, 20, 12, 12 because of the class
equation for $\Alt_5$ that we computed in Example \ref{ex:A5}. But of
course we {\em must} include the number 1 in the sum, because $H$ must
contain the identity element and its conjugacy class is a singleton.
This is a contradiction: no such sum equals any divisor of $60$.  This
contradiction proves that $\Alt_5$ has no proper normal subgroup,
hence is simple.
\end{proof}


It is easy to see that $\Alt_2$ and $\Alt_3$ are simple, since
$\Alt_2$ is trivial and $\Alt_3$ is of order 3, thus cyclic of prime
order. (All cyclic groups of prime order are simple.)  See Exercise
\ref{ex:Ansimple} for an outline of a proof that $\Alt_n$ is simple
for all $n\ge 6$. Thus all the alternating groups are simple, except
for $\Alt_4$.


\section*{Exercises}
\begin{problems}
  

\item Find a proper normal subgroup of $\Alt_4$, thus proving that
  $\Alt_4$ is not a simple group.

\item\index{simplicity of $\Alt_n$} \label{ex:Ansimple} Let $N$ be any
  proper normal subgroup of $\Alt_n$ ($n\ge 6$).

(a) Show that $N$ contains an element $\alpha \ne 1$ which has a
fixed point $i \in \{1,\dots, n\}$.

(b) Show that $N$ contains the subset $G_i$ of {\em all} permutations
in $\Alt_n$ which fix $i$.

(c) Show that $G_i$ is isomorphic with $\Alt_{n-1}$. (So $N$ contains
an isomorphic copy of $\Alt_{n-1}$.)

(d) Show that $N$ contains every $G_i$, for each $1 \le i \le n$.

(e) Show that $\Alt_n$ must be simple, by showing that no $N$, with
these properties, exists.




\end{problems}
\end{document}



