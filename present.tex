\documentclass[11pt,oneside]{article}
\usepackage[nohead,margin=1.50in]{geometry} %set margins
\usepackage{amsmath,amssymb,amsthm,pdiag} %AMS packages for math stuff
\usepackage{multicol} % for use in HW section
\usepackage{enumitem}
  \setlist{topsep=1pt,itemsep=0pt,parsep=1pt}
  \setenumerate[1]{label=(\alph*)}

\newenvironment{problems}
{
 \begin{enumerate}[topsep=1pt,itemsep=0pt,parsep=2pt,leftmargin=0.6cm,%
 label={\arabic*.}, ref=\arabic*] \small
}
{
 \end{enumerate}
}

%%% Define some theorem and example environments. The starred versions
%%% are un-numbered and the unstarred versions are numbered.
\newtheoremstyle{plain}
  {\topsep}   % ABOVESPACE
  {\topsep}   % BELOWSPACE
  {\slshape}  % BODYFONT
  {0pt}       % INDENT (empty value is the same as 0pt)
  {\bfseries} % HEADFONT
  {.}         % HEADPUNCT
  {5pt plus 1pt minus 1pt} % HEADSPACE
  {}          % CUSTOM-HEAD-SPEC

\swapnumbers
\newtheorem{thm}{Theorem}[section]
\newtheorem{lem}[thm]{Lemma}
\newtheorem{prop}[thm]{Proposition}
\newtheorem{cor}[thm]{Corollary}
\newtheorem*{thm*}{Theorem}
\newtheorem*{lem*}{Lemma}
\newtheorem*{prop*}{Proposition}
\newtheorem*{cor*}{Corollary}
\newtheorem{prob*}{Problems}

\theoremstyle{definition}
\newtheorem{defn}[thm]{Definition}
\newtheorem{example}[thm]{Example}
\newtheorem{examples}[thm]{Examples}
\newtheorem{rmk}[thm]{Remark}
\newtheorem{rmks}[thm]{Remarks}
\newtheorem{conv}[thm]{Convention}
\newtheorem*{defn*}{Definition}
\newtheorem*{example*}{Example}
\newtheorem*{examples*}{Examples}
\newtheorem*{rmk*}{Remark}
\newtheorem{rmks*}{Remarks}
\newtheorem*{conv*}{Convention}


%%% Define some convenient abbreviations for common mathematical
%%% notations.
\newcommand{\R}{\mathbb{R}} % use \R for the real numbers
\newcommand{\C}{\mathbb{C}} % use \C for the complex numbers
\newcommand{\Z}{\mathbb{Z}} % use \Z for the integers
\newcommand{\Q}{\mathbb{Q}} % use \Q for the rationals
\newcommand{\N}{\mathbb{N}} % use \N for the natural numbers
\newcommand{\F}{{\mathbb F}}
\newcommand{\compose}{\circ} % functional composition
\newcommand{\gen}[1]{\langle #1 \rangle}
\newcommand{\End}{\operatorname{End}}
\newcommand{\GL}{\mathrm{GL}}
\newcommand{\SL}{\mathrm{SL}}
\renewcommand{\O}{\mathrm{O}}
\newcommand{\SO}{\mathrm{SO}}
\newcommand{\U}{\mathrm{U}}
\newcommand{\SU}{\mathrm{SU}}
\newcommand{\g}{\mathfrak{g}}
\newcommand{\transpose}{\mathsf{T}}
\newcommand{\B}{\mathcal{B}}
\newcommand{\Rep}{\operatorname{Rep}}
\newcommand{\Mat}{\operatorname{Mat}}
\newcommand{\inner}[2]{\langle #1, #2 \rangle}
\newcommand{\sgn}{\operatorname{sgn}}
\newcommand{\n}{\underline{\mathbf{n}}}
\newcommand{\Sym}{\mathbb{S}}
\newcommand{\Alt}{\mathbb{A}}
\newcommand{\D}{\mathbb{D}}
\newenvironment{perm}[2]{\left(\begin{smallmatrix}#1 \\ #2}{\end{smallmatrix}\right)}
\newcommand{\lcm}{\operatorname{lcm}}
\newcommand{\res}{\operatorname{res}}
\newcommand{\im}{\operatorname{im}}
\newcommand{\ptn}{\mathfrak{p}}

\allowdisplaybreaks
\parskip=2pt

%\title{Document Title}
%\author{author's name}

\begin{document}%\maketitle
\setcounter{section}{30}

\section{Presentations of groups}\noindent
The idea of presenting a group by generators and relations is known as
\emph{combinatorial group theory}. The idea originated in a paper by
Walther von Dyck in 1882; it is useful in algebraic topology and
geometry. We give a brief crash course in the main ideas.

\begin{defn}\index{free~group}
  The \emph{free group} on generators $a,b,c, \dots$ is the group
  consisting of all possible strings ({\em e.g.}
  $aaba^{-1}bbbac^{-1}$) of the generators and their inverses, taken
  in any order, with the understanding that whenever a symbol $x$ and
  its inverse $x^{-1}$ appear next to one another, we are allowed to
  reduce the string by omitting the pair ({\em i.e.} we cancel
  $xx^{-1}$ and $x^{-1}x$ whenever they appear).  A string is {\em
    reduced} if no such cancellations are possible, and the elements
  of the free group are precisely the reduced strings of any
  length. Obviously, any free group is infinite, since there is no
  limit to the length of the strings.
\end{defn}

By definition, there are no other relations between the generators in
a free group, besides the relation which says we can cancel an element
multiplied by its inverse. In a free group, as in any group, we allow
ourselves to write $a^n$ as a shorthand for the string $aaa\cdots a$
($n$ times repeated) and similarly $a^{-n}$ as a shorthand for
$a^{-1}a^{-1}\cdots a^{-1}$ (repeated $n$ times). 

An important element of the free group is the {\em empty string}
$\epsilon$, which is the only string of length zero. Given two strings
$s_1$ and $s_2$ in a free group, we multiply then by juxtaposition;
that is, $s_1s_2$ is the string obtained by joining the symbols of
$s_1$ with the symbols of $s_2$ to make a new string. For instance,
if $s_1 = a^3b^2a^{-2}bab$ and $s_2 = b^{-1}a^{-2}b^5cb$ then
\begin{align*}
  s_1s_2 &= a^3b^2a^{-2}bab\, b^{-1}a^{-2}b^5cb \\
  &= a^3b^2a^{-2}ba \, a^{-2}b^5cb \\
  &= a^3b^2a^{-2}b a^{-1}b^5cb 
\end{align*}
remembering our rules for reducing adjacent pairs of symbols and their
inverse. It is easy to see that the free group is a group, with
$\epsilon$ serving as its identity element. What is the inverse of a
given string?

The free group on one generator $a$ is obviously abelian, since the
only strings we can form using one symbol $a$ are powers of $a$, and
powers of $a$ surely commute with one another. It is easy to see that
the free group $G$ on a single generator $a$ is isomorphic with the
infinite (additive) cyclic group $\Z$; the isomorphism is given by the
map $G \to \Z$ defined by the rule $a^m \to m$.

A free group with more than one generator is never abelian, since the
equality $ab=ba$ for two generators $a,b$ would be a non-trivial
relation (and thus the group would no longer be ``free'').


\begin{thm}
  Any group is isomorphic to a quotient of some free group.
\end{thm}


\begin{proof} (Sketch)
Here's a sketch of the procedure to prove this fact, and it gives a
method of constructing the group as a quotient of a free group. Given
a group $G$, let $S=\{a,b,c, \dots \}$ be a set of elements of $G$
which generates $G$ (so that $G=\gen{S}$). At worst, we could take $S
= G$ but usually we can pick a much smaller generating set, as we have
seen in numerous examples. Let $F$ be the free group on the generators
in the set $S$, and in the free group $S$ let $N$ be the subgroup
consisting of all strings in the generators which evaluate to the
identity in the given group $G$. Then one can prove that $N$ is a
normal subgroup of $F$, and moreover that $F/N \simeq G$. This is done
by the first isomorphism theorem, using a natural homomorphism from
$F$ onto $G$.
\end{proof}

\begin{rmk}
In practice, one would like to choose the generating set $S$ to be as
small as possible. Also, it is customary to specify the normal
subgroup $N$ by finding a (smallest possible, or at least nice in some
way) set of generators for it. These generators of $N$ are known as
the \emph{defining relations} of $G$. 
\end{rmk}

\begin{example}
An example of a group given by generators and relations is the
dihedral group $\D_n$, which is given by two generators $r,d$ subject
to the defining relations
\[
 r^n=1; \quad d^2 = 1; \quad rdr = d.
\]
It is easy to see that these relations hold in $\D_n$, but rather
harder to show that any other relations between these generators will
be consequences of these relations. To do this, you need to prove that
the corresponding elements
\[
  r^n; \quad d^2; \quad rdrd
\]
generate a normal subgroup $N$ of the free group $F$ on the generators
$r,h$ such that $F/N \simeq \D_n$. This can be shown with a bit of work. 
\end{example}


\begin{example}
The symmetric group $\Sym_n$ is generated by transpositions, we have
proved a long time ago. In fact, it is generated by the the
\emph{adjacent interchanges}; these are the special transpositions
\[
  t_1=(1,2), t_2=(2,3), \dots, t_{n-1} = (n-1,n).
\]
It is not hard to verify that these $n-1$ generators satisfy the
relations
\begin{gather*}
 t_i^2 = 1; \quad t_it_j=t_jt_i \text{ (if $|i-j|>1$)};
 \\ t_it_{i+1}t_i = t_{i+1}t_it_{i+1}
\end{gather*}
for all values of the indices $i,j$ for which the equations make sense
in $\Sym_n$. The last two relations are known as the \emph{braid
  relations}.

It is harder to verify, yet true, that these relations actually {\em
  define} $\Sym_n$, in the sense that every other relation between
generators is a consequence of these. In other words, the
corresponding elements
\begin{gather*}
  t_i^2; \quad t_it_jt_it_j \text{ (if $|i-j|>1$)}; \quad
  t_it_{i+1}t_it_{i+1}t_it_{i+1}
\end{gather*}
obtained from the relations (by writing each relation in the form $R =
1$) generates a normal subgroup $N$ such that $\Sym_n \simeq F/N$
where $F$ is the free group on the symbols $t_1, \dots, t_{n-1}$.
\end{example}



\begin{defn}\index{generators~and~relations}\index{presentation}
In general, if $G$ is a group given by generators $g_1, g_2, \dots$
with defining relations $R_1, R_2, \dots$ then we will write
\[
 G \simeq \gen{g_1, g_2, \dots \mid R_1, R_2, \dots} 
\]
to indicate this.  Such a description of $G$ (by generators and
relations) is called a {\em presentation} of $G$.
\end{defn}

A presentation $G = \gen{g_1, g_2, \dots \mid R_1, R_2, \dots}$
describes $G$ as $F/N$ where $F$ is the free group generated by $g_1,
g_2, \dots$, and $N$ is the normal subgroup of $G$ generated by $R_1,
R_2, \dots$. Note that if one is given a generator $R$ for $N$ then
$R$ is a string in the symbols $g_1, g_2, \dots$ (and their inverses)
which is identity in $G$; sometimes the relation $R$ is expressed as
an equation $R = 1$, or any equation equivalent to it. For instance,
one of the generators of the set of relations in $\D_n$ is $rdrd$, and
it is customary to write this as the relation $rdrd=1$ in $\D_n$,
which is equivalent to the equation $rdr=d$ (since $d^2=1$).

\begin{examples}
In the notation just introduced, we write out presentations for each
of the groups $C_n = $ cyclic group of order $n$, $\D_n =$ the
dihedral group of symmetries of a regular $n$-gon, and $\Sym_n =$
symmetric group on $n$ letters:
\begin{gather*}
 C_n = \gen{x \mid x^n = 1}\\ \D_n = \gen{ r,d \mid r^n=1,\ 
  d^2=1,\ rdr=d }\\
\Sym_n = \gen{t_1, \dots, t_{n-1} \mid t_i^2=1,\ 
  t_it_j=t_jt_i\text{ (if $|i-j|>1$)},\ 
  t_it_{i+1}t_i= t_{i+1}t_it_{i+1} }.
\end{gather*}
The presentation of $\Sym_n$ is called the \emph{Coxeter
  presentation}. There is an elaborate theory of \emph{Coxeter
  groups}, which are groups defined by generators and relations
subject to certain conditions. Coxeter groups have applications to Lie
groups.
\end{examples}

Although it is satisfying to describe a group by means of generators
and relations, there are some issues with this approach. Here are
three fundamental problems formulated by Max Dehn in 1911. These
problems are important for presentation theory as well as
applications.  Let $G$ be a group defined by means of a given
presentation. Dehn's problems are:
\begin{enumerate}[label=\Roman*.]
\item\index{word~problem} (Word problem) For an arbitrary word $w$ in
  the generators, decide in a finite number of steps whether or not
  $w = 1$ in $G$.

\item\index{conjugacy~problem} (Conjugacy problem) For two arbitrary
  words $w_1, w_2$ in the generators, decide in a finite number of
  steps whether or not $w_1$ and $w_2$ are conjugate in $G$.

\item\index{isomorphism~problem} (Isomorphism problem) For an arbitrary
  group $H$ defined by means of another presentation, decide in a
  finite number of steps whether or not $G \cong H$.
\end{enumerate}

Unfortunately, it has been proven that all three problems are in
general \emph{undecidable}\index{undecidable} in the sense of
mathematical logic. Roughly speaking, a problem is undecidable if it
is not possible to design a Turing machine to solve it. It is
generally understood that algorithms are equivalent to Turing
machines, so the undecidability of these problems effectively means
that there is no general algorithm to solve them.

That may sound rather discouraging, but in fact there are many classes
of presentations for which all three problems have been solved, so the
situation is not so dire as it may seem. Much more is known about
combinatorial\index{combinatorial group theory} group theory; the
book \emph{Combinatorial Group Theory} by Magnus, Karass, and Solitar
is a classic reference.


\section*{Exercises}
\begin{problems}
  
\item Find a presentation by generators and relations of the Klein
  four group $K_4$, using two generators. 

\item Show that the group given by the presentation $\gen{a,b \mid a^5
  = b^2 = 1, ba = a^2b}$ is isomorphic to $\Z_2$.

\item Show that the group $G = \gen{x,y \mid x^2=y^n = 1, xyx =
  y^{-1}}$ is isomorphic to $\D_n$.

\item Find a minimal presentation of the group $\Z_2 \times \Z_3$.

\item What is the minimum number of generators needed to generate the
  group $G = \Z_2 \times \Z_2 \times \Z_2$? Find a presentation of
  this group.

\item Show that the quaternion group\index{quaternion~group}
  $Q = \{\pm 1, \pm i, \pm j, \pm k \}$, in which $i^2=j^2=k^2 = -1$,
  $(-1)^2 = 1$, and the symbols $i,j,k$ multiply like standard unit
  vectors according to usual cross-product rules in $\R^3$, is
  presented by $\gen{ a,b \mid a^2=b^2 = (ab)^2 }$.

\item\index{braid~group} Artin's \emph{braid group} $\mathbb{B}_n$ can
  be defined by the presentation
  $$\gen{t_1, \dots, t_{n-1} \mid t_it_j=t_jt_i \text{ (if $|i-j|>1$)
    }; \quad t_it_{i+1}t_i= t_{i+1}t_it_{i+1} }.$$ Show that $\Sym_n$
  is a homomorphic image of $\mathbb{B}_n$ and compute the kernel.

\item 
\begin{enumerate}
  \item Show that the symmetric group $\Sym_n$ is generated by the
    $n$-cycle $c = (1,2,3,4,\dots,n)$ and the transposition $t =
    (1,2)$.

  \item (May be difficult) Find a set of defining relations on these
    generators that gives a presentation of $\Sym_n$ by these two
    generators and the relations you found.
\end{enumerate}

\item (May be difficult) Find a presentation by generators and
  relations for the alternating group $\Alt_n$. [Hint: You may wish to
    start with the fact that $\Alt_n$ is generated by 3-cycles.]


\end{problems}

\end{document}
