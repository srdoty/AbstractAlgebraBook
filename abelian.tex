\documentclass[11pt,oneside]{article}
\usepackage[nohead,margin=1.50in]{geometry} %set margins
\usepackage{amsmath,amssymb,amsthm,pdiag} %AMS packages for math stuff
\usepackage{multicol} % for use in HW section
\usepackage{enumitem}
  \setlist{topsep=1pt,itemsep=0pt,parsep=1pt}
  \setenumerate[1]{label=(\alph*)}

\newenvironment{problems}
{
 \begin{enumerate}[topsep=1pt,itemsep=0pt,parsep=2pt,leftmargin=0.6cm,%
 label={\arabic*.}, ref=\arabic*] \small
}
{
 \end{enumerate}
}

%%% Define some theorem and example environments. The starred versions
%%% are un-numbered and the unstarred versions are numbered.
\newtheoremstyle{plain}
  {\topsep}   % ABOVESPACE
  {\topsep}   % BELOWSPACE
  {\slshape}  % BODYFONT
  {0pt}       % INDENT (empty value is the same as 0pt)
  {\bfseries} % HEADFONT
  {.}         % HEADPUNCT
  {5pt plus 1pt minus 1pt} % HEADSPACE
  {}          % CUSTOM-HEAD-SPEC

\swapnumbers
\newtheorem{thm}{Theorem}[section]
\newtheorem{lem}[thm]{Lemma}
\newtheorem{prop}[thm]{Proposition}
\newtheorem{cor}[thm]{Corollary}
\newtheorem*{thm*}{Theorem}
\newtheorem*{lem*}{Lemma}
\newtheorem*{prop*}{Proposition}
\newtheorem*{cor*}{Corollary}

\theoremstyle{definition}
\newtheorem{defn}[thm]{Definition}
\newtheorem{example}[thm]{Example}
\newtheorem{examples}[thm]{Examples}
\newtheorem{rmk}[thm]{Remark}
\newtheorem{rmks}[thm]{Remarks}
\newtheorem{conv}[thm]{Convention}
\newtheorem*{defn*}{Definition}
\newtheorem*{example*}{Example}
\newtheorem*{examples*}{Examples}
\newtheorem*{rmk*}{Remark}
\newtheorem{rmks*}{Remarks}
\newtheorem*{conv*}{Convention}


%%% Define some convenient abbreviations for common mathematical
%%% notations.
\newcommand{\R}{\mathbb{R}} % use \R for the real numbers
\newcommand{\C}{\mathbb{C}} % use \C for the complex numbers
\newcommand{\Z}{\mathbb{Z}} % use \Z for the integers
\newcommand{\Q}{\mathbb{Q}} % use \Q for the rationals
\newcommand{\N}{\mathbb{N}} % use \N for the natural numbers
\newcommand{\F}{{\mathbb F}}
\newcommand{\compose}{\circ} % functional composition
\newcommand{\gen}[1]{\langle #1 \rangle}
\newcommand{\End}{\operatorname{End}}
\newcommand{\GL}{\mathrm{GL}}
\newcommand{\SL}{\mathrm{SL}}
\renewcommand{\O}{\mathrm{O}}
\newcommand{\SO}{\mathrm{SO}}
\newcommand{\U}{\mathrm{U}}
\newcommand{\SU}{\mathrm{SU}}
\newcommand{\g}{\mathfrak{g}}
\newcommand{\transpose}{\mathsf{T}}
\newcommand{\B}{\mathcal{B}}
\newcommand{\Rep}{\operatorname{Rep}}
\newcommand{\Mat}{\operatorname{Mat}}
\newcommand{\inner}[2]{\langle #1, #2 \rangle}
\newcommand{\sgn}{\operatorname{sgn}}
\newcommand{\n}{\underline{\mathbf{n}}}
\newcommand{\Sym}{\mathbb{S}}
\newcommand{\Alt}{\mathbb{A}}
\newcommand{\D}{\mathbb{D}}
\newenvironment{perm}[2]{\left(\begin{smallmatrix}#1 \\ #2}{\end{smallmatrix}\right)}
\newcommand{\lcm}{\operatorname{lcm}}
\newcommand{\res}{\operatorname{res}}
\newcommand{\im}{\operatorname{im}}
\newcommand{\ptn}{\mathfrak{p}}

\allowdisplaybreaks
\parskip=2pt

%\title{Document Title}
%\author{author's name}

\begin{document}%\maketitle
\setcounter{section}{29}

\section{Classification of finite abelian groups}\noindent
We have now developed enough general theory to classify all the finite
abelian groups, up to isomorphism. This means that we can list all
possibilities for a given order $n$, and prove that any abelian group
of order $n$ must be isomorphic to one of the groups on the list.

Let $p$ be a prime.  By definition, a $p$-\emph{group} is a group in
which every element has order a power of $p$.  Recall that Cauchy's
theorem says that a finite group $G$ has an element of order $p$, for
every prime $p$ dividing the order of $G$.  It follows immediately
from Cauchy's theorem and Lagrange's theorem that any finite $p$-group
must be of order $p^k$ for some $k \ge 1$.



\begin{lem}\label{lem:Sylows}
  Any finite abelian group is isomorphic to the direct product of its
  Sylow subgroups.
\end{lem}

\begin{proof}
Let $G$ be an abelian group of order $n$.  Since any subgroup of an
abelian group is normal, all the Sylow subgroups of $G$ are normal.
By the fundamental theorem of arithmetic, $n$ can be factored in the
form $n = p_1^{k_1} p_2^{k_2} \cdots p_r^{k_r}$ where $p_1, p_2,
\dots, p_r$ are the distinct prime factors of $n$. Let $G_j$ be the
Sylow $p_j$-subgroup of $G$ of order $p_j^{k_j}$. Each Sylow subgroup
is normal since every subgroup of an abelian group is normal. By
Lagrange's theorem, the order of any element of $G_j$ is a power of
$p_j$ and the order of any element of the product of the other Sylow
subgroups is coprime to $p_j$. Thus the intersection of any $G_j$ with
the product of the other Sylow subgroups is trivial, and it follows
that $G = G_1 \times G_2 \times \cdots \times G_r$.
\end{proof}

\begin{rmk}
It is not hard to show that the subgroup $G_j$ appearing in the above
proof may also be described as $G_j = \{ x \in G \mid x^{q_j} = 1 \}$
where $q_j = p_j^{k_j}$. This satisfying concrete description is often
useful for computations.
\end{rmk}


The lemma reduces our task (understanding all finite abelian groups)
to the task of understanding finite abelian $p$-groups. For this it
will be useful to use the language of partitions.

\begin{defn}\index{partition}
A \emph{partition} of a positive integer $k$ is a sequence $\lambda =
(\lambda_1, \lambda_2, \dots, \lambda_m)$ of positive integers such
that $\lambda_1 \ge \lambda_2 \ge \cdots \ge \lambda_m$ and $\lambda_1
+ \cdots + \lambda_m = k$. The number $m$ is called the \emph{length}
of the partition $\lambda$.
\end{defn}

There is just one partition of $1$, namely $(1)$. The partitions of
$2$ are $(2)$ and $1,1)$. The partitions of $3$ are $(3)$, $(2,1)$,
and $(1,1,1)$.  The partitions of $4$ are $(4)$, $(3,1)$, $(2,2)$,
$(2,1,1)$, and $(1,1,1,1)$. When writing partitions, people often use
an \emph{exponential shorthand notation} for repeated entries, in
which $\lambda_j^t$ stands for $\lambda_j$ repeated $t$ times. For
example, in the shorthand notation, $(4,2^2,1^3) = (4,2,2,1,1,1)$.

It is easy to construct abelian groups of order $p^k$, for a given
fixed prime $p$. Pick any partition $\lambda = (\lambda_1, \lambda_2,
\dots, \lambda_m)$ of the exponent $k$, and define
\[
  G(p,\lambda) = \Z_{p^{\lambda_1}} \times \Z_{p^{\lambda_2}} \times
  \cdots \times \Z_{p^{\lambda_m}}.
\]
Since $G(p,\lambda)$ is a direct product of cyclic groups, it is a
direct product of abelian groups, and hence is abelian. Since
$\lambda_1 + \cdots + \lambda_m = k$, the order of $G(p,\lambda)$ is
$p^k$, so $G(p,\lambda)$ is an abelian group of order $p^k$.


\begin{example}
The partitions of $5$ are $(5)$, $(4,1)$, $(3,2)$, $(3,1^2)$,
$(2^2,1)$, $(2,1^3)$, and $(1^5)$. For any prime $p$, we have the
following abelian groups of order $p^5$.
\[
\begin{array}{ll}
  \lambda & G(p,\lambda) \\ \hline
  (5) & \Z_{p^5} \\
  (4,1) & \Z_{p^4} \times \Z_{p} \\
  (3,2) & \Z_{p^3} \times \Z_{p^2} \\
  (3,1^2) & \Z_{p^3} \times \Z_{p} \times \Z_p \\
  (2^2,1) & \Z_{p^2} \times \Z_{p^2} \times \Z_p \\
  (2,1^3) & \Z_{p^2} \times \Z_{p} \times \Z_p \times \Z_p \\
  (1^5) & \Z_{p} \times \Z_p \times \Z_p\times \Z_p \times \Z_p .
\end{array}
\]
In this example, it is not difficult to verify that all the groups in
the list are pairwise non-isomorphic, because you can always find an
element in one product group of a different order than all elements of
the other. (For example, $\Z_{p^5}$ has an element of order $p^5$, but
none of the others do, so $\Z_{p^5}$ is not isomorphic to any of the
others.)
\end{example}

Based on the example, one naturally expects that if $\lambda \ne \mu$
are distinct partitions of $k$ then $G(p,\lambda) \ncong
G(p,\mu)$. This is indeed true, but somewhat awkward to prove at the
moment, so we defer the proof until later in the analysis.

It would be nice if the abelian $p$-groups we just constructed, namely
the ones of the form $G(p, \lambda)$, where $\lambda$ is a partition,
give \emph{all} the finite abelian $p$-groups up to isomorphism. We
will prove that this is indeed the case, following a note by
G.~Navarro\footnote{Navarro, Gabriel: \emph{On the fundamental theorem
    of finite abelian groups.}  Amer. Math. Monthly 110 (2003), no. 2,
  153--154.} published in 2003. 

The key fact we need to prove is that any abelian $p$-group is
isomorphic to a product of cyclic groups. The proof rests on the
following pair of lemmas.


\begin{lem}\index{p@$p$-group}\label{lem:A}
  Suppose that $G$ is a finite abelian $p$-group, where $p$ is a
  prime. If $G$ has a unique subgroup of order $p$ then $G$ is cyclic.
\end{lem}

\begin{proof}
By induction on $|G|$. Consider the homomorphism $f: G \to G$ given by
$f(x) = x^p$. This is a homomorphism because $G$ is abelian. Let $K$
be the kernel of $f$. By hypothesis, $K$ is the only subgroup of $G$
of order $p$. Then $G/K \cong f(G)$. If $K=G$ then $G$ is cyclic and
we are done. Otherwise, $K$ is a proper subgroup of $G$ and hence
$f(G)$ is not the trivial subgroup. Every subgroup of $f(G)$ is a
subgroup of $G$, so $f(G)$ has a unique subgroup of order $p$, and
thus is cyclic by the inductive hypothesis. So $G/K$ is cyclic. Then
there is some $y \in G$ such that $yK$ generates $G/K$. Clearly $y \ne
1$ as $yK$ has order $|G|/p$ in $G/K$. Let $H = \gen{y}$ be the cyclic
subgroup of $G$ generated by $y$. Then $HK = G$. (Otherwise the order
of $HK/K$ would be strictly less than $|G\/p$, in violation of the
fact that $yK$ has order $|G|/p$.)  Now by Cauchy's theorem $H$ has a
subgroup of order $p$, which must be $K$. Hence $K \subset H$ and $G =
HK = H = \gen{y}$, so $G$ is cyclic.
\end{proof}


\begin{lem}\label{lem:B}
  If $G$ is a finite abelian $p$-group, let $C$ be a cyclic subgroup
  of maximal order. Then $G = C \times B$ for some subgroup $B$.
\end{lem}

\begin{proof}
Again we use induction on $|G|$. If $G$ is cyclic then $G = C \times
\{1\}$ and we are done. Otherwise, by the previous lemma $G$ has at
least two subgroups of order $p$, but $C$ has only one. Let $K$ be a
subgroup of order $p$ which is not contained in $C$. Then $C \cap K =
\{1\}$. No homomorphic image of $G$ has a cyclic subgroup of order
larger than $|C|$, so $CK/K \cong C$ is cyclic of maximal order in
$G/K$. By the inductive hypothesis, $G/K = CK/K \times B/K$ for some
subgroup $B$ of $G$. Since $K \subset B$, it follows that $G = (CK)B =
CB$.  Furthermore, $C \subset CK$ and $B \cap CK = K$, so $C \cap B =
C \cap B \cap CK = C \cap K = \{1\}$. Hence $G = C \times B$.
\end{proof}






\begin{thm}[classification of abelian $p$-groups]
\label{thm:abelian-p}%
  Suppose that $G$ is a finite abelian $p$-group, of order $p^k$ for
  some $k \ge 1$. Then there is some partition $\lambda$ of $k$ such
  that $G \cong G(p,\lambda)$. Furthermore, $G(p,\lambda) \ncong
  G(p,\mu)$ for $\lambda \ne \mu$.
\end{thm}

\begin{proof}
The proof is by induction on $|G|$.  If $G$ is cyclic, then $G \cong
\Z_{p^k}$ and we are finished. Otherwise, let $C$ be a cyclic subgroup
of $G$ of maximum possible order. Then $|C| = p^{\lambda_1}$ where
$\lambda_1 < k$. By Lemma \ref{lem:B}, $G = C \times B$ with $|B| <
|G|$. Note that $B$ is also a $p$-group; in fact $|B| =
p^{k-\lambda_1}$. By the inductive hypothesis, $B$ is isomorphic to a
product of cyclic groups of the form $G(p, (\lambda_2, \dots,
\lambda_m))$, where $(\lambda_2, \dots, \lambda_m)$ is a partition of
$k-\lambda_1$. Thus $G = C \times B \cong G(p,\lambda)$, where
$\lambda = (\lambda_1, \lambda_2, \dots, \lambda_m)$. This proves the
first claim.  It also proves the second claim, since if $C \times B =
C \times B'$ then $B \cong B'$. 
\end{proof}


As an easy corollary, we obtain the desired classification theorem for
finite abelian groups.

\begin{thm}[classification of finite abelian groups]
\index{classification!of~finite~abelian~groups}%
  Any finite abelian group is isomorphic to a direct product of cyclic
  groups. If $n = p_1^{k_1} \cdots p_t^{k_r}$ is the prime power
  factorization of $n$ then the number of isomorphism classes of
  finite abelian groups of order $n$ is $\ptn(k_1) \cdots \ptn(k_r)$,
  where $\ptn(k)$ is the number of partitions of $k$.
\end{thm}

\begin{proof}
Combine Lemma \ref{lem:Sylows} and Theorem \ref{thm:abelian-p}. 
\end{proof}

The function $\ptn(k)$ appearing in the last theorem is called the
\emph{partition function}. There is no known explicit formula for
$\ptn(k)$.  We are now finished with the classification of all finite
abelian groups.


\begin{example}
Let $n = 1200 = 2^4 \cdot 3 \cdot 5^2$. Since $\ptn(4) = 5$,
$\ptn(1)=1$, and $\ptn(2) = 2$, it follows that there are exactly $10
= 5 \cdot 1 \cdot 2$ different abelian groups of order $1200$, up to
isomorphism.  We can easily list all ten isomorphism types of the
abelian groups of order $1200$.  They are indexed by ordered pairs
$(\lambda, \mu)$ of partitions, such that $\lambda$ is a partition of
$4$ and $\mu$ is a partition of $2$ (we can omit the unique partition
of $1$ from our indexing), as follows:
\[
\begin{array}{ll}
  (\lambda, \mu) & G(2,\lambda) \times G(3,(1)) \times G(5,\mu)
  \\ \hline ((4), (2)) & \Z_{2^4} \times \Z_{3} \times \Z_{5^2}
  \\ ((4), (1^2)) & \Z_{2^4} \times \Z_{3} \times \Z_{5} \times \Z_{5}
  \\ ((3,1), (2)) & \Z_{2^3} \times \Z_2 \times \Z_{3} \times \Z_{5^2}
  \\ ((3,1), (1^2)) & \Z_{2^3} \times \Z_2 \times \Z_{3} \times \Z_{5}
  \times \Z_{5} \\ ((2^2), (2)) & \Z_{2^2} \times \Z_{2^2} \times
  \Z_{3} \times \Z_{5^2} \\ ((2^2), (1^2)) & \Z_{2^2} \times \Z_{2^2}
  \times \Z_{3} \times \Z_{5} \times \Z_{5} \\ ((2, 1^2), (2)) &
  \Z_{2^2} \times \Z_{2} \times \Z_{2} \times \Z_{3} \times \Z_{5^2}
  \\ ((2, 1^2), (1^2)) & \Z_{2^2} \times \Z_{2} \times \Z_{2} \times
  \Z_{3} \times \Z_{5} \times \Z_{5} \\ ((1^4), (2)) & \Z_{2} \times
  \Z_{2} \times \Z_{2} \times \Z_{2} \times \Z_{3} \times \Z_{5^2}
  \\ ((1^4), (1^2)) & \Z_{2} \times \Z_{2} \times \Z_{2} \times \Z_{2}
  \times \Z_{3} \times \Z_{5} \times \Z_{5}.
\end{array}
\]
The orders of the individual factors in each product above are called
the \emph{elementary divisors}\index{elementary~divisors} of the
group.  Each isomorphism type is uniquely determined by its set of
elementary divisors.


The acute reader might be alarmed to notice that the cyclic group of
order $1200$ appears to be missing from the above list. However, it is
isomorphic to the first group on the list, so it is not actually
missing.

Because of the fact that $\Z_{mn} \cong \Z_m \times \Z_n$ whenever
$m,n$ are relatively prime, there are many other ways to express each
of the above isomorphism types as products of cyclic groups. With this
in mind, we rewrite the above table in a different looking form (but
still isomorphic correspondingly):
\[
\begin{array}{ll}
  (1200) & \Z_{1200} \\ (5 \mid 240) & \Z_5 \times \Z_{240} \\ (2 \mid
  600) & \Z_2 \times \Z_{600} \\ (10 \mid 120) & \Z_{10} \times
  \Z_{120} \\ (4 \mid 300) & \Z_4 \times \Z_{300} \\ (20 \mid 60) &
  \Z_{20} \times \Z_{60} \\ (2 \mid 2 \mid 300) & \Z_{2} \times \Z_{2}
  \times \Z_{300} \\ (2 \mid 10 \mid 60) & \Z_{2} \times \Z_{10}
  \times \Z_{60} \\ (2 \mid 2 \mid 2 \mid 150) & \Z_2 \times \Z_2
  \times \Z_2 \times \Z_{150} \\
  (2 \mid 2 \mid 10 \mid 30) & \Z_2 \times \Z_2
  \times \Z_2 \times \Z_{10} \times \Z_{30}.
\end{array}
\]
Although the second table looks very different from the first, it is
merely another way of describing the isomorphism types as products of
cyclic groups. In this description, the products are indexed by
sequences $(d_1 \mid d_2 \mid \cdots)$ of divisors of the group order
such that each $d_i$ divides the next $d_{i+1}$ and such that the
product of all the $d_i$ is equal to the group order. Such sequences
are called \emph{invariant factors}\index{invariant~factors} of the
group. There is an algorithm for going going back and forth between
elementary divisors and invariant factors, which we leave to the
reader.
\end{example}


It would be nice to go on to solve the problem of classifying
non-abelian finite groups up to isomorphism.  Alas, it is unknown how
to do this in general. This remains an unsolved problem in group
theory.




\section*{Exercises}
\begin{problems}

\item Prove that $\Z_m \times \Z_n \cong \Z_{mn}$ if $m,n$ are
  relatively prime. 

\item Use the result of the previous exercise to express $\Z_{330}$ as
  an isomorphic product of simple groups. What are the composition
  factors of $\Z_{330}$?

\item Show that the composition factors of any abelian $p$-group are
  all isomorphic to $\Z_p$.

\item List all the partitions of 6 and 7. What is $\ptn(6)$ and
  $\ptn(7)$?

\item List all abelian groups of order $30$, up to isomorphism, giving
  both the elementary divisors and invariant factors for each type.

\item List all abelian groups of order $48$, up to isomorphism, giving
  both the elementary divisors and invariant factors for each type.

\item List all abelian groups of order $64$, up to isomorphism, giving
  both the elementary divisors and invariant factors for each type.

\item List all abelian groups of order $100$, up to isomorphism, giving
  both the elementary divisors and invariant factors for each type.

\item Prove that the number of isomorphism types of abelian groups of
  order 128 is 15.

\item Prove that there are 35 different abelian groups of order $2592
  = 2^5 \cdot 3^4$ up to isomorphism. 

\item Show that the generating function for the number $\ptn(n)$ of
  partitions\index{partition} of $n$ is
  $$\sum_{n=0}^{\infty} \ptn(n)x^{n} = \prod_{k=1}^{\infty}
  \left({\frac {1}{1-x^{k}}}\right).$$ This means that you can compute
  $\ptn(n)$ by taking the coefficient of $x^n$ in the product of the
  various geometric series expansions on the right hand side. This
  formula is due to Euler.
\end{problems}

\end{document}
