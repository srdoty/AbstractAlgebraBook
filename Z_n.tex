\documentclass[11pt]{article}
\usepackage[nohead,margin=1.50in]{geometry} %set margins
\usepackage{amsmath,amssymb,amsthm,pdiag} %AMS packages for math stuff
\usepackage{multicol} % for use in HW section
\usepackage{enumitem}
  \setlist{topsep=1pt,itemsep=0pt,parsep=1pt}
  \setenumerate[1]{label=(\alph*)}

\newenvironment{problems}
{
 \begin{enumerate}[topsep=1pt,itemsep=0pt,parsep=2pt,leftmargin=0.6cm,%
 label={\arabic*.}, ref=\arabic*] \small
}
{
 \end{enumerate}
}

%%% Define some theorem and example environments. The starred versions
%%% are un-numbered and the unstarred versions are numbered.
\newtheoremstyle{plain}
  {\topsep}   % ABOVESPACE
  {\topsep}   % BELOWSPACE
  {\slshape}  % BODYFONT
  {0pt}       % INDENT (empty value is the same as 0pt)
  {\bfseries} % HEADFONT
  {.}         % HEADPUNCT
  {5pt plus 1pt minus 1pt} % HEADSPACE
  {}          % CUSTOM-HEAD-SPEC

\swapnumbers
\newtheorem{thm}{Theorem}[section]
\newtheorem{lem}[thm]{Lemma}
\newtheorem{prop}[thm]{Proposition}
\newtheorem{cor}[thm]{Corollary}
\newtheorem*{thm*}{Theorem}
\newtheorem*{lem*}{Lemma}
\newtheorem*{prop*}{Proposition}
\newtheorem*{cor*}{Corollary}

\theoremstyle{definition}
\newtheorem{defn}[thm]{Definition}
\newtheorem{example}[thm]{Example}
\newtheorem{examples}[thm]{Examples}
\newtheorem{rmk}[thm]{Remark}
\newtheorem{rmks}[thm]{Remarks}
\newtheorem{conv}[thm]{Convention}
\newtheorem*{defn*}{Definition}
\newtheorem*{example*}{Example}
\newtheorem*{examples*}{Examples}
\newtheorem*{rmk*}{Remark}
\newtheorem{rmks*}{Remarks}
\newtheorem*{conv*}{Convention}

%%% Define some convenient abbreviations for common mathematical
%%% notations.
\newcommand{\R}{\mathbb{R}} % use \R for the real numbers
\newcommand{\C}{\mathbb{C}} % use \C for the complex numbers
\newcommand{\Z}{\mathbb{Z}} % use \Z for the integers
\newcommand{\Q}{\mathbb{Q}} % use \Q for the rationals
\newcommand{\N}{\mathbb{N}} % use \N for the natural numbers
\newcommand{\F}{{\mathbb F}}
\newcommand{\compose}{\circ} % functional composition
\renewcommand{\implies}{\Rightarrow}
\renewcommand{\iff}{\Leftrightarrow}
\newcommand{\gen}[1]{\langle #1 \rangle}
\newcommand{\End}{\operatorname{End}}
\newcommand{\GL}{\mathrm{GL}}
\newcommand{\SL}{\mathrm{SL}}
\renewcommand{\O}{\mathrm{O}}
\newcommand{\SO}{\mathrm{SO}}
\newcommand{\U}{\mathrm{U}}
\newcommand{\SU}{\mathrm{SU}}
\newcommand{\g}{\mathfrak{g}}
\newcommand{\transpose}{\mathsf{T}}
\newcommand{\B}{\mathcal{B}}
\newcommand{\Rep}{\operatorname{Rep}}
\newcommand{\Mat}{\operatorname{Mat}}
\newcommand{\inner}[2]{\langle #1, #2 \rangle}
\newcommand{\sgn}{\operatorname{sgn}}
\newcommand{\n}{\underline{\mathbf{n}}}
\newcommand{\Sym}{\mathbb{S}}
\newcommand{\Alt}{\mathbb{A}}
\newenvironment{perm}[2]{\left(\begin{smallmatrix}#1 \\ #2}{\end{smallmatrix}\right)}
\newcommand{\lcm}{\operatorname{lcm}}
\newcommand{\res}{\operatorname{res}}


\allowdisplaybreaks
\parskip=2pt

%\title{Document Title}
%\author{author's name}

\begin{document}%\maketitle
\setcounter{section}{10}


\section{Modular arithmetic}\noindent
Modular arithmetic is a novel finite system of arithmetic used in
public-key cryptography\index{cryptography} in order to provide
security for internet transactions. It is also used in algebraic
coding theory, the mathematical theory underlying the encoding of
information on DVDs, satellite communications, etc. And it provides
new examples of groups. We begin with some elementary number theory.

\begin{lem}[division algorithm]\index{division~algorithm}
Let $n$ be a positive integer.  Let $m$ be any integer.
There exist unique integers $q$, $r$ such that
\[
m = qn + r \quad\text{and}\quad 0 \le r < n. 
\]
The integers $q$, $r$ are called the {\em quotient} and {\em
  remainder}, respectively.
\end{lem}

\begin{proof}\index{well-ordering principle}
Recall the \emph{well-ordering principle} of natural numbers, which
states that any nonempty subset of $\N = \{0,1,2,\dots\}$ must have a
least element. Let
\[
  S = \{ m - kn \mid k \in \Z \text{ and } m - kq \ge 0 \}.
\]
By construction, $S$ is a subset of $\N$. We need to show it is
nonempty. If $m \ge 0$ then $m = m-0\cdot n \in S$. If $m < 0$ then
$m-mn = m(1-n) \ge 0$ because $1-n \le 0$, hence $m-mn$ is in $S$. In
either case $S$ is not empty. By the well ordering principle, the set
$S$ has a least element. Let $r$ be the least element of $S$ and let
$q$ be the corresponding value of $k$. Then $m - qn = r$, so $m =
qn+r$. Moreover, $r \ge 0$ since $r \in S$. Finally, $r < n$ since
otherwise $r-n$ would be in $S$, contradicting the fact that $r$ is
the least element of $S$.
\end{proof}

When $m$ is positive, the usual procedure by which we find the
quotient $q$ and remainder $r$ is called long division, as you
undoubtedly recall. 

Given a real number $x$, there is a unique integer $k$ such that $k
\le x < k+1$. The integer $k$ is called the \emph{floor}\index{floor}
of $x$, written as $\lfloor x \rfloor = k$. The floor function is also
known as the \emph{greatest integer} function.  Then for a given pair
of integers $m,n$ we have:
\[
  q = \lfloor m/n \rfloor, \quad r = m-qn.
\]
Note that when $m<0$ we need to pay attention. For instance, we have
$20 = 2 \cdot 7 + 6$ for $20 \div 7$, but $-20 = -3 \cdot 7 + 1$ for
$-20 \div 7$. 





\begin{defn}
  Let $n$ be a positive integer greater than 1. We call $n$ the
  \emph{modulus}\index{modulus}. The set \index{Z@$\Z_n$}$\Z_n = \{0,
  1, \dots, n-1 \}$ is called the set of \emph{residues} modulo $n$.
  This is the set of remainders for long division by $n$, so
  \emph{residue}\index{residue} is another word for \emph{remainder}.
\end{defn}


\begin{defn} [residue function]
  Let a modulus $1<n \in \Z$ be given. Given an integer $m$, let $q,r$
  be the unique integers such that $m = qn + r$ and $0 \le r < n$.
  Then set $\res_n(m) = r$. The resulting function
  $\res_n: \Z \to \Z_n$ is a surjection from the set $\Z$ of integers
  onto the set $\Z_n$ of residues modulo $n$.
\end{defn}

We should read $\res_n(m)$ as ``the residue of $m$ modulo $n$.'' It is
just the remainder of dividing $m$ by $n$, so we can always compute it
by long division.

Now we define addition and multiplication on the set $\Z_n$. 

\begin{defn}
  Let $n>1$ be a fixed integer modulus. Given residues $a,b \in \Z_n$,
  we define
  \[
  a \oplus b = \res_n(a+b), \qquad a \odot b = \res_n(ab).
  \]
  We call the binary operations $\oplus, \odot$ \emph{residue
    addition} and \emph{residue multiplication}.
\end{defn}

The residue addition and multiplication rules just defined can be
remembered as a two-step procedure: first, add or multiply in $\Z$ as
usual, then take the residue of the result modulo $n$.

\begin{rmk}
  We use new symbols $\oplus, \odot$ for residue addition and
  multiplication, in order to distinguish these new and novel
  operations from the usual ones. Eventually, we will drop the extra
  circle around the symbol and use the usual addition and
  multiplication symbols. For now, it is useful to use different
  notation in order to avoid confusion.
\end{rmk}


\begin{example} 
The addition and multiplication tables for $\Z_2$ are given below. 
\[
\begin{array}{|c|cc|} \hline
\oplus&0&1\\ \hline 0&0&1\\ 1&1&0\\ \hline
\end{array}
\qquad \qquad
\begin{array}{|c|cc|} \hline
\odot &0&1\\ \hline
0&0&0\\
1&0&1\\
\hline
\end{array}
\]
We will see later that the above addition table defines a group. This
group appears in computer science as the table defining the behavior
of a binary half-adder circuit, which is a basic circuit used in all
digital computers. 
\end{example}



\begin{example} 
The addition and multiplication tables for $\Z_4$ are compiled below. 
\[
\begin{array}{|c|cccc|} \hline
\oplus&0&1&2&3\\ \hline 
0&0&1&2&3\\ 1&1&2&3&0\\ 
2&2&3&0&1\\ 3&3&0&1&2\\ \hline
\end{array}
\qquad \qquad
\begin{array}{|c|cccc|} \hline
\odot &0&1&2&3\\ \hline
0&0&0&0&0\\
1&0&1&2&3\\ 
2&0&2&0&2\\ 
3&0&3&2&1\\ \hline
\end{array}
\]
This is more interesting. Notice that the tables are ``closed''
systems, in which the combination of two elements produces another one
in the set $\Z_4$. Also notice the novel equation $2 \odot 2 = 0$. In
residue arithmetic, two nonzero elements can multiply to produce $0$,
which is somewhat strange at first sight.
\end{example}

Residue arithmetic is also called \emph{modular arithmetic}. Looking
at the addition tables (for the operation $\oplus$) in the above
examples, we suspect that they might be groups, because the tables
``look'' similar to multiplication tables for other groups we have
seen, and that is indeed the case, as we will see later. But not so
for the multiplication tables (for the operation $\odot$). They are
\emph{never} examples of groups, because there is no inverse of $0$
in the tables. Nevertheless, there is always at least one group inside
the multiplication table, if you know how to look for it. We will
return to this question later, once we have developed a better
definition of group. For now, the focus is on understanding modular
arithmetic.


It is notable that we can make sense of subtraction in $\Z_n$. Here
are the appropriate definitions, which follow the same pattern as the
definitions of negatives and subtraction in the integers.

\begin{defn}
  The \emph{negative} $\ominus a$ of a residue $a \in \Z_n$ is defined to be
  \[
  \ominus a = 
  \begin{cases}
    0 & \text{ if } a = 0 \\
    n-a & \text{ if } a \ne 0.
  \end{cases}
  \]
\end{defn}

Once we have negatives, we can define subtraction in terms of adding
the negative, which is how subtraction is defined for integers and
real numbers.

\begin{defn}
  For any $a,b \in \Z_n$, we define $a \ominus b = a \oplus (\ominus b)$. 
\end{defn}

Notice that $a \oplus ( \ominus a) = 0$ and $(\ominus a) \oplus a =
0$, for any $a \in \Z_n$.


\begin{thm}\label{thm:ressub}
   For any $a,b \in \Z_n$, we have $a \ominus b = \res_n(a-b)$. 
\end{thm}

The proof is a straightforward exercise.

The theorem gives us another way to compute $a \ominus b$. Depending
on the situation, one way might be more convenient than the other, so
it can be useful to have both.

\begin{rmk}[notational convention]
  It soon becomes tedious to always write $\oplus, \odot, \ominus$ for
  the residue operations. The convention is to replace these symbols
  by their corresponding uncircled symbols $+, \cdot, -$ from ordinary
  arithmetic. This convention results in equations such as:
  \[
    5+5 = 0,\quad 6+7 = 3,\quad 6 \cdot 7 = 2,\quad 4 \cdot 5 = 0,
    \quad -6 = 4,\quad 5-6 = 9
  \]
  all of which are valid equations in $\Z_{10}$. This is confusing
  only if you forget that $+$ really means $\oplus$, $\cdot$ means $\odot$,
  and $-$ means $\ominus$.  

  Furthermore, it is conventional to omit the multiplication symbol
  $\cdot$ in cases where doing so will not cause confusion. So, if
  $a,b$ are variables representing values in some residue system
  $\Z_n$, we usually interpret $ab$ as $a \cdot b$, just as we do in
  ordinary algebra.
\end{rmk}

Modular arithmetic\index{modular~arithmetic} is actually similar to
arithmetic everyone already carries out with clocks. In a standard
clock, there are 12 numbers arranged in a circle, and we all
understand that 3 hours after 11 o'clock is 2 o'clock. This is the
same as the equation $11+3 = 2$ in $\Z_{12}$. The only difference
between addition in $\Z_{12}$ and clock addition is that in $\Z_{12}$
we have renamed the modulus $12$ to $0$.

We can visualize modular arithmetic modulo $n$ similarly, by thinking
of a clock with the residues $0, 1, \dots, n-1$ equally spaced around
its face. Whenever we reach $n$ hours, the clock resets to $0$. 




\section*{Exercises}
\begin{problems}

\item The \emph{floor} of a real number $x$ (written $\lfloor x
  \rfloor$) is the greatest integer $k$ such that $k \le x$.  Given
  integers $m,n$ with $n > 0$, show that $q = \lfloor m/n \rfloor$ is
  the integer quotient such that $m = qn+r$ with $0 \le r < n$.


\item A pocket calculator (or computer) that produces decimal
  approximations for quotients can often be used to carry out the
  division algorithm in the set of integers. Given integers $m,n$ with
  $n > 0$, to find the integer quotient $q$ such that $m = qn+r$ with
  $0 \le r < n$, simply calculate $m \div n$ on the device and then
  take $q$ to be the integer part (the floor) of the result. Once you
  have found $q$, of course $r = m - qn$. Use this to find the integer
  quotient $q$ and remainder $r$ for each of the following division
  problems:
  \begin{multicols}{2}
  \begin{enumerate}[topsep=0pt,parsep=0pt]
  \item $1000 \div 31$.
  \item $-1000 \div 31$.
  \item $16075 \div 652$.
  \item $-16075 \div 652$.
  \end{enumerate}
  \end{multicols}
  This approach is slightly dangerous, because if the numbers $m,n$
  are big enough, machine calculation can be subject to roundoff
  error. If you are using a pocket calculator, try to find a case
  where it produces an incorrect integer quotient.

\item Compute the following in the modular system $\Z_{12}$.
  \begin{enumerate}
  \item $7+5$, $5+5$, $11+11$, $-1 + (-1)$.
  \item $3 \cdot 3$, $7 \cdot 5$, $5 \cdot 5$, $11 \cdot 11$, $(-1)
    \cdot (-1)$.
  \item $3-3$, $5-7$, $-2-10$, $5-(-3)$, $5-(-11)$,
    $-2-(-10)$.
  \end{enumerate}

\item Make addition and multiplication tables for $\Z_3$.

\item Make addition and multiplication tables for $\Z_5$.

\item Make addition and multiplication tables for $\Z_6$.

\item Make addition and multiplication tables for $\Z_7$.

\item Make addition and multiplication tables for $\Z_8$.

\item Compute the following in the modular system $\Z_{1201}$.
  \begin{enumerate}
  \item $75+500$, $800+500$, $-1 + (-1)$.
  \item $30 \cdot 30$, $800 \cdot 500$, $9^2$, $(-1)
    \cdot (-1)$, $1200^2$.
  \item $500-800$, $-500-800$, $-500-(-800)$.
  \end{enumerate}

\item Prove Theorem \ref{thm:ressub}.

\item \label{exer:python} The open source \texttt{Python} programming
  language provides built-in commands to compute the integer quotient
  and remainder for integers of any size. If \texttt{a}, \texttt{n}
  are integers, then the useful commands and their effects are:
  \begin{center}
  \begin{tabular}{cl}
    \verb$a % n$ & returns $\res_n(a)$\\
    \texttt{a // n} &  returns the integer quotient of $a \div n$\\
    \texttt{divmod(a,n)} & returns the pair $(q,r)$ such that $a
      = qn+r$ and $0 \le r < n$.
  \end{tabular}
  \end{center}
  Note that the \texttt{divmod} command combines the effect of the
  other two. \texttt{Python} is installed by default on all Mac and
  Linux computers, but it must be downloaded (from \texttt{python.org})
  and installed on a Windows computer. Once \texttt{Python} is
  installed on your computer, open up a command-line terminal and type
  \texttt{python} to start a \texttt{Python} interpreter session. Then
  you can type commands in order to do computations, similar to typing
  commands on a calculator. Hit Enter at the end of each command to
  get results.

  Use \texttt{Python} to compute the integer quotient $q$ and
  remainder $r$ for the following problems:
  \begin{enumerate}
  \item $1234567890987654321 \div 6090609$.
  \item $-1234567890987654321 \div 6090609$.
  \item $12345678909876543211234567890 \div 1234567890987654321$.
  \end{enumerate}



\item After reading Exercise \ref{exer:python}, go to a computer and
  fire up a \texttt{Python} session in order to compute the following:
  \begin{enumerate}
  \item $123456789 - 987654321$ in $\Z_{999750750}$.
  \item $123456789^2$ in $\Z_{999750750}$.
  \item $123456789^3$ in $\Z_{999750750}$.
  \end{enumerate}
  Note: In \texttt{Python}, the symbol \verb$**$ is used instead of
  \verb$^$ for computing powers; i.e., to compute $a^b$ you must type
  \verb$a ** b$ instead of \verb$a ^ b$.

\end{problems}



\newpage\section{Commutative rings}\noindent
Our main goal in this section is to prove that the modular system
$\Z_n$ (along with its operations of addition and multiplication)
forms a commutative ring.\index{ring}

We begin with the definition of commutative ring, which is based on
the properties of the system $\Z$ of integers.


\begin{defn} \label{def:comm-ring}
A {\em commutative ring} is a set $R$ with two binary
operations, addition $+$ and multiplication $\cdot$, such that for all
$a,b,c$ in the set $R$:
  \begin{enumerate}
  \item additive associativity: $a + (b+c) = (a+b)+c$.
  \item additive commutativity: $a+b = b+a$. 
  \item additive identity: There is some element $0 \in R$ such that
    $a+0 = a = 0+a$.
  \item additive inverse: for every $a\in R$, there exists $-a \in R$
    such that $a + (-a) = 0 = (-a)+a$.
  
  \item multiplicative associativity: $a \cdot (b\cdot c) = (a\cdot
    b)\cdot c$.
  \item multiplicative commutativity: $a\cdot b = b\cdot a$. 
  \item multiplicative identity: There is an element $1 \in R$ such
    that $a\cdot 1 = a = 1\cdot a$.
 
  \item distributivity: $a\cdot (b+c) = a\cdot b + a \cdot c$ and
    $(b+c)\cdot a = b\cdot a + c \cdot a$.
  \end{enumerate}
The element $0$ is called the \emph{additive identity}, while $1$ is
the \emph{multiplicative identity}. The element $-a$ is the
\emph{additive inverse} of $a$, or the \emph{negative} of $a$.
\end{defn}

If we remove axiom (f) from the definition then we get the definition
of a \emph{ring}. In general, multiplication in a ring is not required
to be commutative. For now, all of our rings will be commutative.

The set of integers $\Z$ is a commutative ring under ordinary addition
and multiplication.  So are the sets of rational numbers $\Q$, real
numbers $\R$, and complex numbers $\C$.  Indeed, the ring axioms are
modeled on the fundamental properties of the ordinary number systems.

\begin{rmk}
  We can subtract in any ring, because additive inverses exist, so we
  can define $a-b = a+(-b)$. Thus, we can view a commutative ring as a
  set of elements along with a way to add, subtract, and multiply
  them, subject to the familiar properties of number systems. But note
  that division may not always be possible in a ring. In $\Z$, we
  cannot divide $1$ by $2$, as the fraction $1/2$ is not an integer.
\end{rmk}


The set $\N$ of natural numbers under ordinary addition and
multiplication is \emph{not} a ring, because most elements of $\N$ do
not have an additive inverse; i.e., axiom 1(d) fails. In other words,
the set $\N$ is not a ring because subtraction doesn't make sense in
$\N$.

Here are some formal consequences of the definition of ring. These
properties hold in all rings, commutative or not.

\begin{thm} \label{thm:ring-basics} 
Let $R$ be a ring. For any $a,b,c \in R$ we have:
\begin{enumerate}
\item $a \cdot 0  = 0 = 0 \cdot a$.

\item $a(-b) = -(ab) = (-a)b$.

\item $(-a)(-b) = ab$.

\item $a(b-c) = ab - ac$.

\item $(-1)a = -a$.
\end{enumerate}
\end{thm}

Notice that we have omitted the symbol $\cdot$ in the products in
parts (b)--(e) above. It is standard to omit the multiplication symbol
in situations where it leads to no confusion.

The proof of these basic facts is an exercise. For instance, to prove
(a) one would begin with the equality $0 + 0 = 0$, which comes from
axiom (h) of Definition \ref{def:comm-ring} by taking $a=0$ there,
then multiply both sides by $a$, and so on.




Theorem \ref{thm:Z_n-is-a-ring} below says that the set $\Z_n$ is a
commutative ring. So modular arithmetic provides examples of finite
commutative rings. The main goal of this section is to prove this
important result. 

To accomplish our goal we need to study another, more sophisticated,
construction of $\Z_n$ in terms of equivalence classes. This approach
gives more powerful information about the properties of $\Z_n$. 

We start by saying a bit more about equivalence classes.  The
following important definition from set theory will be used again in
the future.

\begin{defn*}\index{equivalence relation}
Let $A$ be a set. An \emph{equivalence relation} on $A$ is a relation
$\sim$ on $A$ which is reflexive ($a \sim a$, for all $a \in A$),
symmetric ($a \sim b$ implies $b \sim a$, for all $a,b \in A$), and
transitive ($a \sim b$ and $b \sim c$ implies $a \sim c$, for all
$a,b,c \in A$).
\end{defn*}

It is a general fact in set theory that an equivalence relation $\sim$
on a set $A$ always induces a partition of the set into disjoint
equivalence classes:
\[
  A = \bigcup_{a \in A} [a]
\]
where $\overline{a} = [a] = \{b\in A \mid a \sim b \}$ is the
equivalence class of $a$. (We write $\overline{a}$, or $[a]$, for the
equivalence class containing $a$. That class is the set of all
elements that are equivalent to $a$.)
Another general fact is that
\[
  [a] = [b] \iff a \sim b.
\]
Thus, there can be many different ways to write the same equivalence
class. The element used to write a class is called a
\emph{representative}\index{representative} of the class.


We need the following general concept, which will be used again later.

\begin{defn*}[quotient set]\index{quotient}
  Let $\sim$ be an equivalence relation on a set $A$, and write $[a]$
  (or $\overline{a}$) for the equivalence class containing $a$. The
  set $A/\!\!\sim \ = \{ [a] : a \in A \}$, the set of all equivalence
  classes, is called the \emph{quotient} of $A$ by $\sim$.
\end{defn*}





Now we return to the task of proving that $\Z_n$ is a ring. We need to
recall some basic properties about the set $\Z$ of integers.

Recall that if $a,b$ are integers then $a$ \emph{divides} $b$ (written
as $a \mid b$) if and only if there is some $k \in \Z$ such that $b =
ak$.  If $a$ divides $b$ we also say that $b$ \emph{is divisible by}
$a$.

\begin{defn}[Gauss]\index{congruence}
  Let $n$ be a fixed positive integer, and $a,b \in \Z$.  We say that
  $a$ is \emph{congruent} to $b$ modulo $n$ if and only if $n \mid
  (a-b)$. We write $a \equiv b \pmod{n}$ to mean that $a$ is congruent
  to $b$ modulo $n$.
\end{defn}

Here are the basic properties of congruences, all of which are routine
to prove.

\begin{thm} \label{thm:cong-prop-1}
Let $n$ be a fixed positive integer. Let $a, b, c$ be integers. Then:
\begin{enumerate}
  \item $a \equiv a \pmod{n}$.

  \item $a \equiv b \pmod{n}$ implies $b \equiv a \pmod{n}$.

  \item $a \equiv b \pmod{n}$ and $b \equiv c \pmod{n}$ implies $a
    \equiv c \pmod{n}$.
\end{enumerate}
\end{thm}


Theorem \ref{thm:cong-prop-1} says that congruence modulo $n$ is an
equivalence relation on the set $\Z$. More precisely, suppose we fix
$n$ and write $a \sim b$ if and only if $a \equiv b \pmod{n}$. Then
$\sim$ is an equivalence relation on the set $\Z$.

The equivalence relation $\sim$ on the set $\Z$ of integers induces a
partition of $\Z$ into disjoint equivalence classes:
\[
  \Z = \bigcup_{a \in \{0,1,\dots, n-1\}} [a]
\] 
where the equivalence class $[a]$ is defined by 
\[
   [a] = \{ b \in \Z \mid a \sim b \} = \{b \in \Z \mid a \equiv b
   \pmod{n} \}.
\]
The equivalence class $[a]$ of $a$ is the collection of all integers
which are congruent mod $n$ to $a$. We say that $a$ is a
\emph{representative} of its equivalence class $[a]$. Note that a
given equivalence class has infinitely many representatives, because
\[
  [a] = [b] \iff a \sim b.
\]
For example, if $n=12$ then $[2] = [14] = [-10]$ because $2 \equiv 14
\equiv -10$ (mod $12$); i.e., $2 \sim 14 \sim -10$. 

\begin{defn}
  Equivalence classes for the equivalence relation $\sim$ defined by
  congruence modulo $n$ are called \emph{congruence classes}.
\end{defn}


The following gives a nice characterization of the numbers in a
congruence class. The proof is an exercise.

\begin{thm}\label{thm:cong-form}
  Let $a \in \Z$ and let $r = \res_n(a)$. Then the congruence class
  $[a]$ is the collection of all integers of the form $kn+r$ where
  $k \in \Z$.
\end{thm}

\begin{defn}
We define addition and multiplication of congruence classes (for some
fixed modulus $n$) by the rules:
\[
  [a]+[b] = [a+b], \qquad [a]\cdot[b] = [a b].
\]
\end{defn}


The following easy to prove result means that the definition makes
sense; i.e., addition and multiplication of congruence classes are
well-defined.


\begin{thm}\label{thm:cong-prop-2} 
Let $a,b,c,d \in \Z$. Suppose that $a \equiv b \pmod{n}$ and $c \equiv
d \pmod{n}$; i.e., $a \sim b$ and $c \sim d$ Then:

(a) $a+c \equiv b+d \pmod{n}$; i.e., $a+c \sim b+d$.

(b) $ac \equiv bd \pmod{n}$; i.e., $ac \sim bd$.
\end{thm}



Applying the quotient construction to the set $\Z$ of integers with the
equivalence relation $\sim$ defined by congruence modulo $n$, we
obtain:
\[
  \Z/\!\!\sim \ = \{ [a] : a \in \Z \} = \{ [a] : a = 0, 1, \dots, n-1
  \}.
\]
Note that the distinct classes in $\Z/\!\!\sim$ are those listed in
the rightmost set above, because of the aforementioned fact that $[a]
= [b] \iff a \sim b$.

Thus there is a bijection from the set $\Z_n = \{0, 1, \dots, n-1 \}$
onto the set $\Z/\!\!\sim$, defined by $a \mapsto [a]$. Furthermore,
it can be shown that this bijection preserves addition and
multiplication, in the sense that:
\[
  a \oplus b \mapsto [a]+[b]; \qquad a \odot b \mapsto [a]\cdot [b]
\]
for all $a,b$. (Here we have momentarily reverted to the circle
notation, in order to avoid confusion.) This proves that
\[
  \Z_n \cong \Z/\!\!\sim
\]
i.e., the structure $\Z_n$ is isomorphic to the quotient
$\Z/\!\!\sim$, where $\sim$ is the equivalence relation defined by
congruence modulo $n$.



The quotient construction makes it easier to prove the following
result.

\begin{thm}\label{thm:Z_n-is-a-ring}
  Fix some integer $n > 1$, and let $\sim$ be the equivalence relation
  defined by congruence modulo $n$.  Arithmetic in the quotient set
  $$\Z/\!\!\sim \ = \{[a]: a = 0, 1, \dots, n-1 \}$$ satisfies the
  following properties, holding for all $a, b, c \in \Z$:
  \begin{enumerate}
  \item  $[a] + ([b+[c]]) = ([a]+[b])+[c]$.
  \item  $[a]+[b] = [b]+[a]$. 
  \item  $[a]+[0] = [a] = [0]+[a]$.
  \item  $[a] + [-a] = [0] = [-a]+[a]$. 

  \item  $[a] \cdot ([b\cdot[c]]) = ([a]\cdot[b])\cdot[c]$.
  \item  $[a]\cdot[b] = [b]\cdot[a]$. 
  \item  $[a]\cdot[1] = [a] = [1]\cdot[a]$.

  \item $[a]\cdot ([b]+[c]) = [a]\cdot[b] + [a] \cdot [c]$ and
  $([b]+[c])\cdot [a] = [b]\cdot[a] + [c] \cdot [a]$.
  \end{enumerate}
  In other words, $\Z/\!\!\sim$ is a commutative ring (and so is its
  isomorphic cousin $\Z_n$).
\end{thm}


\begin{proof}
The proof uses the fact that all of these properties correspond to
similar properties of ordinary integers. For instance, to prove (b)
just use the fact that addition in $\Z$ is commutative. Then
\[
  [a] + [b] = [a+b] = [b+a] = [b]+[a]
\]
where the middle equality comes from the fact that $a+b=b+a$ for
integers $a,b$.  This proves (b). All the other properties are proved
similarly.
\end{proof}



\begin{rmks}
1. The easy proof of the above theorem is a direct consequence of the
power of equivalence classes. Working with equivalence classes takes
some getting used to, but the effort pays dividends, in that it makes
proofs easier.

2. In the construction of $\Z_n$, the elements are the residues
$0,1,\dots, n-1$ and the operations $\oplus, \odot$ are defined in a
special way. In the construction of $\Z/\!\!\sim$, the elements are
the congruence classes $[0],[1],\dots, [n-1]$ and the operations
$+, \cdot$ are defined in terms of the usual addition and
multiplication of integers. The two constructions turn out to be
equivalent, thus providing two perspectives on $\Z_n$.

3. In both constructions of $\Z_n$, we have chosen the set
$\{ 0, 1, \dots, n-1 \}$ of residues modulo $n$ as a complete set of
representatives of the quotient set. But it is often useful to work
instead with a different complete set of representatives:
\[
  \Z_n =
  \begin{cases}
    \{0, \pm 1, \dots, \pm(N-1), N \}, &\text{ if } n=2N \\
    \{0, \pm 1, \dots, \pm N\}, &\text{ if } n=2N+1
  \end{cases}
\]
where $N$ is an integer.  For example, we can write
$\Z_7 = \{ 0, \pm1, \pm 2, \pm 3\}$.
\end{rmks}

\section*{Exercises}
\begin{problems}

\item Prove that $a \equiv b \pmod{n}$ if and only if $a$ and $b$ have
  the same integer residue (remainder) when divided by $n$.

\item Prove Theorem \ref{thm:ring-basics}. 

\item Define a relation $\approx$ on the set $P$ of all living people
  by $a \approx b$ if the age of $a$ (in years) is equal to the age of
  $b$. 
  \begin{enumerate}
  \item Prove that $\approx$ is an equivalence relation on $P$.
  \item If Sally is age 18 years, then describe the class [Sally] in words.
  \item Describe the quotient set $P/\!\!\approx$ in words.
  \item Is $P/\!\!\approx$ finite or infinite?
  \end{enumerate}

\item Let $A = \{a, b, c, \dots, x, y, z \}$ be the usual English
  alphabet. For the purpose of this problem, we define a \emph{word}
  over $A$ to be any string of one or more letters from the
  alphabet. For example, $bab$, $abba$, and $dogfrog$ are words
  according to our definition. Define a relation $\approx$ on the set
  $W$ of words by $\alpha \approx \beta$ if and only if the first
  letter of $\alpha$ equals the first letter of $\beta$ (where
  $\alpha, \beta \in W$).
  \begin{enumerate}
  \item Prove that $\approx$ is an equivalence relation on $W$.
  \item Describe the quotient set $W/\!\!\approx$ in words. Find a
    nice set of representatives of the classes in $W/\!\!\approx$.
  \item Compute $|W/\!\!\approx|$. 
  \end{enumerate}




\item Define a relation $\approx$ on the set $\R$ of real numbers by
  $a \approx b \iff a - b \in \Z$. 
  \begin{enumerate}
  \item Prove that $\approx$ is an equivalence relation on $\R$.
  \item Compute the class $[1/2]$ of $1/2$. What is its cardinality?
  \item Compute the class $[3/2]$ of $3/2$. How is $[3/2]$ related to
    $[1/2]$?
  \item Describe the quotient set $\R/\!\!\approx$ and find a complete
    set of representatives for the quotient set elements.
  \item Is $\R/\!\!\approx$ finite or infinite?
  \end{enumerate}

\item Prove Theorem \ref{thm:cong-prop-1}.

\item Prove Theorem \ref{thm:cong-form}.

\item Prove Theorem \ref{thm:cong-prop-2}.

\item Prove parts (a), (h) of Theorem \ref{thm:Z_n-is-a-ring}.


\end{problems}




\newpage\section{Fields}\noindent
The main goal of this section is to show that the ring $\Z_n$ is a
field\index{field} if and only if the modulus $n$ is a prime number.

\begin{defn}
  A \emph{multiplicative inverse}\index{inverse} of $a$ is an element
  $a^{-1}$ such that $a a^{-1} = 1 = a^{-1} a$.  An element $a$ in a
  ring $R$ is said to be a \emph{unit}\index{unit} (or
  \emph{invertible element}) if there is a multiplicative inverse of
  $a$ in the ring.
\end{defn}

If a multiplicative inverse exists in a ring then it is unique.  Every
nonzero element is a unit in the rings $\Q, \R$, and $\C$. The only
units in the ring $\Z$ are $\pm 1$.

\begin{defn}
  A \emph{field} is a commutative ring in which $1 \ne 0$ and every
  nonzero element is a unit.
\end{defn}

Thus, $\Q, \R, \C$ are fields. But $\Z$ is \emph{not} a field. 

In a field, the existence of multiplicative inverse means that we can
define division, by:
\[
  b/a = b \cdot a^{-1} \quad (a \ne 0).
\]
So we can think of a field as a system in which the usual operations
of addition, multiplication, subtraction, and division makes sense
and obey the usual laws of algebra.



The central question for this section is: for which values of $n$ is
the ring $\Z_n$ a field? To study this question, we look at a more
general question: what is the set of units in $\Z_n$?

\begin{defn}
  The set of all units in a ring $R$ will be denoted by
  $R^\times$\index{R@$R^\times$} (or $R^*$). This is called the
  \emph{multiplicative group of
    units}\index{multiplicative~group~of~units} in the ring.
\end{defn}
 
The set $R^\times$ provides another example of a group. 
In terms of this notation, the definition of a field can be rephrased
as: a commutative ring $R$ is a field if and only if $R^\times =
R-\{0\}$.

Recall that the notation $\gcd(a,b)$ stands for the \emph{greatest
  common divisor}\index{gcd} of integers $a,b$. Furthermore, recall
that $\gcd(a,b)$ can be calculated using the \emph{Euclidean
  algorithm}.\index{Euclidean~algorithm} Finally, recall that the
\emph{extended} Euclidean algorithm produces integers $x,y$ such that
\[
   ax+by = \gcd(a,b).
\]
It is often said that this equation expresses the gcd as a
\emph{linear combination} of $a,b$. Using these properties of the gcd,
we can prove the following important result.


\begin{thm}\label{thm:gcd}
  Let $n$ be a positive integer. A congruence class $[a] \in \Z_n$ is
  a unit if and only if $\gcd(a,n) = 1$. Hence, the set $\Z_n^\times$
  of units in $\Z_n$ is equal to $\{ [a] \in \Z_n : \gcd(a,n) = 1 \}$.
\end{thm}

\begin{proof}
Suppose that $\gcd(a,n)=1$. Use the Euclidean algorithm to find
integers $x,y$ such that $ax+ny = 1$. This can be done because
$\gcd(a,n)=1$ by hypothesis. Then $ax-1 = ny$, so $n \mid (ax-1)$ and
thus $ax \equiv 1 \pmod{n}$. This means that $[a] \cdot [x] = [1]$ in
$\Z_n$. So $[x] = [a]^{-1}$ and $[a]$ is a unit.

On the other hand, if $\gcd(a,n) = g > 1$, then I claim that $[a]$ is
not a unit. To see this, assume it is a unit. Then $[a]^{-1}$ exists
in $\Z_n$ and $[a]\cdot [a]^{-1} = [1]$. But the condition $\gcd(a,n)
= g > 1$ means that $g \mid a$ and $g \mid n$, so $a = a_1 g$ and $n =
n_1 g$ for some integers $a_1, n_1$. Now
\[
  [a]\cdot [n_1] = [an_1] = [a_1gn_1] = [a_1n] = [0].
\]
If we multiply the equation $[a]\cdot [n_1] = [0]$ by $[a]^{-1}$ we
obtain $[a]^{-1} \cdot [a] \cdot [n_1] = [0]$, i.e., $[n_1] = [0]$.
But this is a contradiction, since the equation $n = n_1 g$ implies
that $0 < n_1 < n$. Thus our assumption must be incorrect, and $[a]$
is not a unit.
\end{proof}


\begin{cor}
  Let $n$ be a positive integer. Then $\Z_n$ is a field if and only if
  $n$ is a prime number.
\end{cor}

\begin{proof}
Suppose that $n$ is prime. Then for all $[a] \ne [0]$ we have
$\gcd(a,n) = 1$. Each such $[a]$ is a unit in $\Z_n$ by the
theorem. So $\Z_n$ is a field.

Suppose that $n$ is not prime. Then $n = ab$ for some integers $1 <
a,b < n$. Then $\gcd(a,n) = a$, so $[a]$ is not a unit in $\Z_n$. This
is a nonzero element of $\Z_n$ which is not a unit, so $\Z_n$ is not a
field.
\end{proof}


\begin{defn}[notation]
  From now on, we write $\F_p = \Z_p$\index{F@$\F_p$} for the field of
  $p$ elements, when $p$ is a prime number. An alternative notation
  for $\F_p$ is $\text{GF}(p)$\index{GF@$\text{GF}(p)$}. In this
  context, GF stands for \emph{Galois field}.  The finite fields
  $\F_p$ of $p$ elements are called Galois fields\index{Galois~field}.
\end{defn}

It should be noted that there are other finite fields besides the ones
we have constructed. We leave that for later.

\begin{examples}
 $\Z_4^\times = \{[1], [3] \}$,
 $\Z_7^\times = \{[1],[2],[3],[4],[5],[6] \}$, and
\\
 $\Z_{15}^\times = \{[1], [2], [4], [7], [8], [11], [13], [14] \}$.
\end{examples}



\section*{Exercises}
\begin{problems}

\item Prove that a commutative ring $R$ is a field if and only if
  $R^\times = R-\{0\}$.

\item There is one and only one pair $(a,b)$ of integers for which
  $\gcd(a,b)$ is not defined. What pair is it?


\item Compute the set $\Z_n^\times$ of units for the following cases:
  \begin{multicols}{3}
  \begin{enumerate}[topsep=0pt,parsep=0pt]
  \item $n=2$.
  \item $n=3$.
  \item $n=5$.
  \item $n=6$.
  \item $n=8$.
  \item $n=9$.
  \end{enumerate}
  \end{multicols}

\item Use guess-and-check (trial and error) to solve the equation
  $3x=4$ in $\Z_{10}$, if possible. (When the modulus is small, there
  are not many cases to try.)

\item Use guess-and-check (trial and error) to solve the equation
  $4x=5$ in $\Z_{10}$, if possible.

\item For which of the following values of $n$ is the ring $\Z_n$ a
  field:
  \begin{multicols}{3}
  \begin{enumerate}[topsep=0pt,parsep=0pt]
  \item $n=2$.
  \item $n=3$.
  \item $n=5$.
  \item $n=6$.
  \item $n=8$.
  \item $n=9$.
  \end{enumerate}
  \end{multicols}

\item Use guess-and-check (trial and error) to calculate the following
  inverses:
  \begin{multicols}{3}
  \begin{enumerate}[topsep=0pt,parsep=0pt]
  \item $[2]^{-1}$ in $\Z_{9}$.
  \item $[3]^{-1}$ in $\Z_{10}$.
  \item $[4]^{-1}$ in $\Z_{15}$.
  \end{enumerate}
  \end{multicols}



\item Prove that the equation $ax=b$ is solvable in $\Z_n$ whenever
    $a \in \Z_n$ is a unit. Explain exactly \emph{how} to solve the
    equation. 



\item The Euclidean algorithm\index{Euclidean~algorithm}, which was
  described in the last book of Euclid's \emph{Elements}, works as
  follows, for any given pair of integers $a,b$ such that not both are
  zero:
  \begin{enumerate}[label=\arabic*),ref=\arabic*]\sf
  \item If $b < 0$ replace $b$ by $|b|$. Do the same for $a$.
  \item\label{here} If $b=0$ then the gcd\index{gcd} is $a$, so return
    $a$ and stop.
  \item If $b \ne 0$, compute the unique integers $q,r$ such that
    $a = qb+r$ and $0 \le r < b$. Then replace the pair $(a,b)$ by the
    pair $(b,r)$. Go to Step \ref{here}.
  \end{enumerate}
  Use the Euclidean algorithm to compute the following:
  \begin{multicols}{2}
  \begin{enumerate}[topsep=0pt,parsep=0pt]
  \item $\gcd(87031, 4750)$.
  \item $\gcd(48157656,541541)$.
  \end{enumerate}
  \end{multicols}

\item Refer to the previous problem for a description of the Euclidean
  algorithm. 
  \begin{enumerate}
  \item Prove that whenever the pair $(a,b)$ of integers is not
    $(0,0)$ then the Euclidean algorithm must stop after finitely many
    steps.
  \item Explain what happens in the first stage of the algorithm if
    $0<a<b$. What are the next values of $(a,b)$?
  \end{enumerate}


\item The extended Euclidean algorithm returns a triple
  $\text{xgcd}(a,b) = (g,x,y)$ of integers such that $g = \gcd(a,b)$
  and $g = ax+by$. We assume that both $a,b$ are given positive
  integers. Then the algorithm works as follows:
  \begin{enumerate}[label=\arabic*),ref=\arabic*]\sf
  \item Set $(x_0, y_0) = (1,0)$. Set $(x,y) = (0,1)$. 
  \item\label{step2} If $b = 0$, return $(a,x_0, y_0)$.
  \item If $b>0$, let $q,r$ be the unique integers such that $a=qb+r$
    and $0 \le r < b$. Replace $(a,b)$ by $(b,r)$.
  \item Replace $(x,y)$ by $(x_0-qx, y_0-qy)$. Replace $(x_0,y_0)$ by
    $(x,y)$. Go to Step \ref{step2}. 
  \end{enumerate}
  Use the extended Euclidean algorithm to compute the following:
  \begin{multicols}{3}
  \begin{enumerate}[topsep=0pt,parsep=0pt]
  \item $\text{xgcd}(23,301)$.
  \item $\text{xgcd}(87031, 4750)$.
  \item $\text{xgcd}(48157656,541541)$.
  \end{enumerate}
  \end{multicols}


\item Use the results of the previous problem to calculate the
  following inverses:
  \begin{multicols}{2}
  \begin{enumerate}[topsep=0pt,parsep=0pt]
  \item $[23]^{-1}$ in $\Z_{301}$.
  \item $[4750]^{-1}$ in $\Z_{87031}$. 
  \item $[541541]^{-1}$ in $\Z_{48157656}$. 
  \end{enumerate}
  \end{multicols}


\end{problems}

\end{document}
