These are my lecture notes for a first course in abstract algebra, which I have taught a number of times over the years. Typically, the course attracts students of varying background and ability. The notes assume some familiarity with linear algebra, in that matrices are used frequently.

The main focus of the course is on group theory, with the goal of getting to the Sylow theorems and the classification of finite abelian groups. The very beginnings of ring theory are also treated here, with a focus on commutative rings, in order to discuss the finite rings and fields inherent in modular arithmetic. Students have little trouble understanding the ring axioms, probably due to prior exposure to the number systems of basic mathematics. 

The organization of the material is somewhat novel, in that the main classes of examples are introduced and studied first, before the abstract group axioms are given. This seems to offer some advantages over the standard approach of clobbering unsuspecting students with the abstract group axioms before they have had any experience with examples. In my experience, even very capable students find the group axioms quite difficult at first.

Modulo preliminaries, the course starts with permutation groups, which are defined as nonempty sets of permutations closed under products and inverses; this of course includes the symmetric and alternating groups. Next the dihedral groups are introduced as symmetry groups of regular polygons and the groups of rotational symmetries of the platonic solids are also discussed briefly without proof. Next comes modular arithmetic, including axioms for commutative rings and fields, and a proof that the ring of integers modulo $n$ is a field if and only if $n$ is prime. The last class of concrete examples are linear groups, defined as nonempty sets of matrices closed under products and inverses. A fairly detailed analysis of the rotation group $\SO(2)$ is given, along with the full orthogonal group $\Orth(2)$ of orthogonal $2 \times 2$ matrices. 

Only then are the axioms for abstract groups introduced. By this point students have seen enough examples of groups to be able to appreciate the utulity value of the axiomatic approach. The rest of the course proceeds as usual, covering all the standard main topics, including subgroups, cyclic groups, quotients, homomorphisms, products, group actions, Sylow theorems, and finite abelian groups. Brief discussions of simple groups, composition series, and generators and relations are included.

I wish to thank all the students over the years who used various incarnations of these notes; their feedback has been incorporated into the notes in many ways.


