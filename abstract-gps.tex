\documentclass[11pt]{article}
\usepackage[nohead,margin=1.50in]{geometry} %set margins
\usepackage{amsmath,amssymb,amsthm,pdiag} %AMS packages for math stuff
\usepackage{multicol} % for use in HW section
\usepackage{enumitem}
  \setlist{topsep=1pt,itemsep=0pt,parsep=1pt}
  \setenumerate[1]{label=(\alph*)}

\newenvironment{problems}
{
 \begin{enumerate}[topsep=1pt,itemsep=0pt,parsep=2pt,leftmargin=0.6cm,%
 label={\arabic*.}, ref=\arabic*] \small
}
{
 \end{enumerate}
}

%%% Define some theorem and example environments. The starred versions
%%% are un-numbered and the unstarred versions are numbered.
\newtheoremstyle{plain}
  {\topsep}   % ABOVESPACE
  {\topsep}   % BELOWSPACE
  {\slshape}  % BODYFONT
  {0pt}       % INDENT (empty value is the same as 0pt)
  {\bfseries} % HEADFONT
  {.}         % HEADPUNCT
  {5pt plus 1pt minus 1pt} % HEADSPACE
  {}          % CUSTOM-HEAD-SPEC

\swapnumbers
\newtheorem{thm}{Theorem}[section]
\newtheorem{lem}[thm]{Lemma}
\newtheorem{prop}[thm]{Proposition}
\newtheorem{cor}[thm]{Corollary}
\newtheorem*{thm*}{Theorem}
\newtheorem*{lem*}{Lemma}
\newtheorem*{prop*}{Proposition}
\newtheorem*{cor*}{Corollary}

\theoremstyle{definition}
\newtheorem{defn}[thm]{Definition}
\newtheorem{example}[thm]{Example}
\newtheorem{examples}[thm]{Examples}
\newtheorem{rmk}[thm]{Remark}
\newtheorem{rmks}[thm]{Remarks}
\newtheorem{conv}[thm]{Convention}
\newtheorem*{defn*}{Definition}
\newtheorem*{example*}{Example}
\newtheorem*{examples*}{Examples}
\newtheorem*{rmk*}{Remark}
\newtheorem{rmks*}{Remarks}
\newtheorem*{conv*}{Convention}


%%% Define some convenient abbreviations for common mathematical
%%% notations.
\newcommand{\R}{\mathbb{R}} % use \R for the real numbers
\newcommand{\C}{\mathbb{C}} % use \C for the complex numbers
\newcommand{\Z}{\mathbb{Z}} % use \Z for the integers
\newcommand{\Q}{\mathbb{Q}} % use \Q for the rationals
\newcommand{\N}{\mathbb{N}} % use \N for the natural numbers
\newcommand{\F}{{\mathbb F}}
\newcommand{\compose}{\circ} % functional composition
\newcommand{\gen}[1]{\langle #1 \rangle}
\newcommand{\End}{\operatorname{End}}
\newcommand{\GL}{\mathrm{GL}}
\newcommand{\SL}{\mathrm{SL}}
\renewcommand{\O}{\mathrm{O}}
\newcommand{\SO}{\mathrm{SO}}
\newcommand{\U}{\mathrm{U}}
\newcommand{\SU}{\mathrm{SU}}
\newcommand{\g}{\mathfrak{g}}
\newcommand{\transpose}{\mathsf{T}}
\newcommand{\B}{\mathcal{B}}
\newcommand{\Rep}{\operatorname{Rep}}
\newcommand{\Mat}{\operatorname{Mat}}
\newcommand{\inner}[2]{\langle #1, #2 \rangle}
\newcommand{\sgn}{\operatorname{sgn}}
\newcommand{\n}{\underline{\mathbf{n}}}
\newcommand{\Sym}{\mathbb{S}}
\newcommand{\Alt}{\mathbb{A}}
\newcommand{\D}{\mathbb{D}}
\newenvironment{perm}[2]{\left(\begin{smallmatrix}#1 \\ #2}{\end{smallmatrix}\right)}
\newcommand{\lcm}{\operatorname{lcm}}
\newcommand{\res}{\operatorname{res}}

\allowdisplaybreaks
\parskip=2pt



%\title{Document Title}
%\author{author's name}

\begin{document}%\maketitle
\setcounter{section}{16}

\section{Abstract groups}\noindent
The axiomatic definition of abstract group\index{abstract~group} is
based on special classes of examples of groups, such as permutation
groups and matrix groups.  Those examples share some common features:
closure under products and inverses, associativity, and identity. In
the definition we are about to give, closure under products is
implicit in the definition of binary operation, while closure under
inverses is an axiom.

We start with the concept of a binary
operation\index{binary~operation} on a set. Intuitively, a binary
operation is a \emph{law of combination} which combines two elements
of a set to produce another element of the set. Ordinary addition and
multiplication are canonical examples.

\begin{defn}
Let $S$ be any given set. Any function from $S \times S$ to $S$ is
called a {\em binary operation} or \emph{law of combination} on the
set $S$. If $f$ is a binary operation then tradition demands that we
write $x\,f\,y$ for the value\footnote{Writing a function between its
  arguments is called {\em infix} notation in computer science.} at
the input pair $(x,y)$ instead of the usual $f(x,y)$.
\end{defn}

In this context the word \emph{binary} refers to the fact that the
function depends on two input variables. By the same token, a
\emph{unary operation} on the set $S$ would be a function from $S$ to
itself. For example, the function that sends each integer to its
negative is a unary operation on the set $\Z$ of integers.

\begin{examples}
1. Addition ($+$) is a binary operation on any of the usual number
sets $\N$, $\Z$, $\Q$, $\R$, $\C$. Multiplication ($\cdot$) is another
binary operation on any of the sets $\N$, $\Z$, $\Q$, $\R$, $\C$.

2. Matrix multiplication is a binary operation on the set
$\text{Mat}_n(F)$ of all $n \times n$ matrices with entries in a given
field $F$.

3. Composition of functions is a binary operation on the set $\Sym_n$
of permutations of $\underline{\mathbf{n}} = \{1, \dots, n\}$.


4. More generally, composition of functions is a binary operation on
the set $S^S$ of all self-maps $S \to S$ of \emph{any} set $S$.

5. Here is a binary operation $\#$ on the finite set $S=\{a,b,c,d\}$
which is defined by means of a ``multiplication'' table as follows:
$$
\begin{array}{|c|cccc|}\hline
\#& a&b&c&d\\ \hline
a & a&a&d&d\\
b & b&b&c&c\\
c & a&d&b&c\\
d & b&c&a&d\\
\hline
\end{array}
$$ This table defines a law of combination for pairs of elements of
$S$. For instance, it says that $a\#c = d$ and $c\#b = d$. For finite
sets $S$, we can always define a binary operation (law of combination)
on $S$ by a table.
\end{examples}



It is important to realize that closure under products is built in to
the definition of a binary operation: to say that $*$ is a binary
operation on $S$ \emph{means} that $S$ is closed under $*$, since all the
values $x*y$ must fall again within $S$, for any $x,y \in S$. This is
just another way of saying that $*$ is a function mapping $S \times S
\to S$.



\begin{defn}
\index{definition~of~group}\index{group~axioms}\label{def:group}
A {\em group} is a set $G$ along with a given binary operation $*$
on $G$, such that the following three axioms hold:
\begin{enumerate}
\item[(G1)] The operation $*$ is {\em associative}: $(a*b)*c =
  a*(b*c)$ for all $a,b,c \in G$.

\item[(G2)] There is an {\em identity} element $e \in G$ satisfying:
  $e*a = a = a*e$ for all $a\in G$.

\item[(G3)] Each $a \in G$ has an {\em inverse} in $G$: given $a \in
  G$ there exists some $a' \in G$ such that $a*a' = a'*a = e$.
\end{enumerate}
\end{defn}


Note that closure under $*$ is implicit in this definition, since $*$
is a binary operation on $G$. Moreover, closure under inverses is the
content of axiom (G3).


When describing a group we should specify not only the set $G$ but
also the binary operation $*$ on $G$, since a given set can have many
different binary operations defined on it. People often use a notation
such as $(G,*)$ to denote a group. If the binary operation $*$ is
implied by context, then we often just write $G$ for simplicity.


\begin{defn}\index{abelian~group}
An {\em abelian}\footnote{In honor of Niels Henrik Abel
  (1802--1829).} group is any group $(G,*)$ in which the commutative
law holds: $a*b=b*a$, for all $a,b\in G$.
\end{defn}

As we have seen, matrix and permutation groups are usually not
abelian, because matrix multiplication and functional composition are
usually not commutative. 


\begin{thm}[Basic properties]\label{gpprop} 
Let $(G,*)$ be any group, with binary operation $*$, and let $a,d,x,y
\in G$.
\begin{enumerate}
\item The identity element $e \in G$ in axiom {\rm(G2)} is unique.

\item The inverse of any $a \in G$ in axiom {\rm(G3)} is unique.

\item The equation $a*x=a*y$ implies that
  $x=y$.  (This is called {\em left cancellation}.)

\item The equation $x*a=y*a$ implies that
  $x=y$.  (This is called {\em right cancellation}.)

\item Each of the equations $a*x=b$, $x*a = b$ ($a,b\in G$) has a
  unique solution $x \in G$.

\item\index{inverse~of~a~product} The inverse of a product is the
  product of the inverses in reverse order.
\end{enumerate}
\end{thm}


\begin{proof} 
(a) Suppose that $e$, $f$ are identity elements of $G$. Then by
  axiom (G2) we have $e*a=a$ and $a=a*f$ for all $a\in G$. In
  particular, taking $a=f$ in the first equality and $a=e$ in the
  second, we get $e*f=f$ and $e=e*f$. Hence $e=f$. This proves
  uniqueness of identity.

(b) Suppose that $b$, $c$ are both inverses of a given $a \in G$. Then
  $a*b=e=b*a$ and $a*c=e=c*a$ by axiom (G3). By the associative law
  (G1) we have $c*(a*b) = (c*a)*b$, so $c*e=e*b$, so $c=b$ by
  (G2). This proves uniqueness of inverses.

(c) Suppose $a*x=a*y$. Then $a'*(a*x) = a'*(a*y)$ where $a'$ is the
  inverse of $a$. By (G1) this implies that $(a'*a)*x = (a'*a)*y$, so
  by (G3) we have $e*x = e*y$, which implies by (G2) that $x=y$.


(d) This is proved similarly to (c), except we multiply by the
  inverse $a'$ on the right instead of on the left.

(e) Suppose that $a*x=b$. Then by left multiplication by the inverse
  $a'$ of $a$ we have $a'*(a*x) = a'*b$, so by (G1) we have $(a'*a)*x
  = a'*b$. Thus by (G3) we have $e*x = a'*b$, so by (G2) we obtain $x
  = a'*b$. This is the unique solution.  This proves the first
  claim. The other claim is proved similarly, using right
  multiplication instead of left multiplication.

(f) Given two elements $a,b$ of $G$ let their respective inverses be
  $a',b'$. Then by axiom (G3) we have $a*a'=e$, $b*b'=e$. Thus 
  \[
    (a*b)*(b'*a') = a*(b*b')*a' = (a*e)*a' = a*a' = e,
  \]
  where we used generalized associativity for the first equality. Let
  $z$ be the inverse of $a*b$. Then $(a*b)*z = e$ by (G3). So $(a*b)*z
  = (a*b)*(b'*a')$. By left cancellation we obtain $z = b'*a'$, i.e.,
  the inverse of $a*b$ is $b'*a'$. This proves the statement for
  products of length two, and it easily extends to products of more
  than two elements, by induction on the length of the product.
\end{proof}



\subsection*{Additive versus multiplicative notation}
\index{additive~notation}\index{multiplicative~notation}%
If the binary operation $*$ is written as addition ($+$) or
multiplication (\,$\cdot$\,) then the group is known as an \emph{additive}
group or a \emph{multiplicative} group, respectively.

If $(G,+)$ is an additive group, it is customary to denote the
identity element by the symbol $0$ and the inverse of $a$ by the
symbol $-a$.  In this case the group axioms take the following form:

(G1)\quad $(a+b)+c = a+(b+c)$; 

(G2)\quad $0+a=a=a+0$; 

(G3)\quad $a+(-a) = 0 = (-a)+a$.

\noindent
It is customary to use the additive notation for a group {\em only for
  abelian groups} and we shall follow that convention in this course.

If $(G, \cdot)$ is a multiplicative group, it is customary to
abbreviate products $a \cdot b$ by $ab$. In this case we usually
denote the identity element by the symbol $1$ and the inverse of $a$
by the symbol $a^{-1}$. Then the group axioms take the form:

(G1)\quad $(ab)c = a(bc)$; 

(G2)\quad $1a=a=a1$; 

(G3)\quad $aa^{-1} = 1 = a^{-1}a$.

The default notation for the group operation is multiplicative
notation, but additive groups appear frequently as well.



\begin{examples} \label{ex}
(a) Any permutation group is a group. Any matrix group is a group. 

(b) The dihedral group $\D_n$ is a group. All symmetry groups are
  groups.

(c) The \emph{abstract cyclic group} is the multiplicative group $C_n$
  generated by a symbol $x$ subject to the relation $x^n = 1$. As a
  set, $C_n = \{1, x, x^2 \dots, x^{n-1}\}$, so $|C_n| = n$.

(d) Any vector space\index{vector~space} $V$ gives an additive abelian
  group $(V,+)$ under vector addition. The identity element is the
  zero vector $\mathbf{0}$ in $V$ and the additive inverse of a vector
  $\mathbf{v} \in V$ is the vector $-\mathbf{v}$. In particular,
  $(\R^n, +)$ is an example of such a group, and more generally we
  have the group $(F^n, +)$ where $F$ is any field.

(e) Any ring\index{ring} $R$ (commutative or not) contains \emph{two}
  groups. One is the additive abelian group $(R,+)$, in which $0$ is
  the additive identity and the inverse of $a$ is written as $-a$. In
  particular, $(\Z, +)$, $(\Q, +)$, $(\R,+)$, $(\C, +)$ are all
  additive abelian groups. Also, $(\Z_n, +)$ is an additive abelian
  group, for any positive integer $n$.

(f) Recall that $R^\times = R^*$\index{R@$R^\times$} is the set of
  units\footnote{A \emph{unit} is an invertible element.} in a given
  ring $R$. The second group in the ring $R$ is the
  \emph{multiplicative group of units} in $R$, i.e., the group
  $(R^\times, \cdot)$ or just $R^\times$ for short.  In particular,
  for any positive integer $n$, we have the multiplicative group
  $\Z_n^\times$ of units in the ring $\Z_n$. The abelian group
  $\Z_n^\times$ is of extreme importance for modern cyptography.

(g) Let $F$ be any field\index{field}. Then by definition $F^\times =
  F-\{0\}$, the set of all nonzero elements of $F$, since every
  nonzero element of a field is invertible. So the previous example
  gives in this case the group $(F^\times, \cdot) = (F-\{0\}, \cdot)$,
  which is called the {\em multiplicative group of the field} $F$.

(h) The {\em trivial group}\index{trivial~group} is the set $\{e\}$
  consisting of only one element, with $e*e=e$.
\end{examples}

The number of elements of a group is called its
\emph{order}.

\begin{defn}\index{order!of~a~group}
  Let $(G,*)$ be a group. The {\em order} of the group is the
  cardinality $|G|$ of the set $G$. If the set $G$ is an infinite set
  then we often write $|G| = \infty$ and call $G$ an infinite group,
  otherwise $G$ is a finite group.
\end{defn}

The word \emph{order} is also used in group theory in another way, as
follows, when speaking about an element of a group.

\begin{defn}\label{def:order-elt}\index{order!of~an~element}
  Let $a \in G$ where $(G,*)$ is a group.  The {\em order} of $a$ is
  the least positive integer $r$ such that $a$ combined with itself
  $r$ times yields the identity element $e$.  If no such $r$ exists,
  then the order is defined to be $\infty$.
\end{defn}

In any group, the order of the identity element is always 1.  Since
the word \emph{order} has two different meanings within group theory,
we always have to determine its usage from the context.

\begin{lem}\label{lem:inverse-if-finite}
  If $a$ is an element of order $r$ in a group $(G,*)$, then the
  inverse of $a$ is equal to $a* \cdots *a$ ($r-1$ factors) obtained
  by combining $a$ with itself $r-1$ times.
\end{lem}

\begin{proof}
  Let $b$ be equal to $a* \cdots *a$ ($r-1$ factors). Then it is clear
  that $a*b = e = b*a$. It follows from uniqueness of inverses that
  $b$ equals the inverse of $a$.
\end{proof}



Now we discuss \emph{laws of exponents}\index{laws~of~exponents} in
groups. We need to distinguish between multiplicative and additive
groups, which use different notation.  If $a$ is an element of a
multiplicative group, then we define $a^n$ to be $aa\cdots a$ ($n$
times repeated) for any positive integer $n$, we define $a^0 = 1$, and
we define $a^{-n} = (a^{-1})^n$.  Note that $(a^n)^{-1} =
a^{-n}$. Moreover, we have
\[
  a^m\,a^n = a^{m+n}, \qquad (a^m)^n = a^{mn}
\]
for any $m,n \in \Z$.

In an additive group we have to use a different notation. In this case we
define $n a$ to be $a+a+\cdots+a$ ($n$ summands) for any positive
integer $n$, we define $0 a = 0$, and we define $(-n) a = n(-a)$. 
Note that $-(na) = (-n)a$.  Moreover, 
\[
  ma + na = (m+n)a, \qquad n(ma) = (nm)a
\]
for any $m,n \in \Z$. In an additive group, ``powers'' are written as
multiples, because in ordinary arithmetic repeated addition is written
as a multiple while repeated multiplication is written as a power.


\begin{rmk}
We can rephrase Definition \ref{def:order-elt} in terms these
notations as follows.  In a multiplicative group the order of $a$ is
the least positive integer $r$ such that $a^r = 1$. In the additive
case the order of $a$ is the least positive integer $r$ such that $ra
= 0$.
\end{rmk}

Now we define the important notion of isomorphism of groups. 

\begin{defn}
  Let $(G,*)$ and $(H,\#)$ be given groups. An \emph{isomorphism} of
  $G$ onto $H$ is any bijection $f \colon G \to H$ such that $f(a*b) =
  f(a)\#f(b)$ for all $a,b \in G$. Whenever such an $f$ exists then we
  say that $G$ is \emph{isomorphic} to $H$, and write $G \cong H$ or
  $G \simeq H$ (interchangeably).
\end{defn}

So an isomorphism is a bijective mapping from one group to the other
which matches up products in the two groups. If one group is an
additive group and the other a multiplicative group then this means
that sums get matched with products.

If two groups are isomorphic then {\em they are essentially the same
  group}, except for the form of their elements. In particular,
isomorphic groups must have the same structural properties (i.e, they
have same order, the same number of subgroups, etc). The following is
easy to check.


\begin{thm}\label{thm:iso}\index{isomorphism}
  Isomorphism of groups is an equivalence relation on the class of
  groups: it is reflexive, symmetric, and transitive.
\end{thm}

\begin{example}\index{R@$\R^+$}
It is easy to check that the set $(\R^+, \cdot)$ of all positive real
numbers is a group under multiplication. We claim that the
multiplicative group $(\R^+, \cdot)$ is isomorphic to the additive
group $(\R, +)$ of real numbers. The isomorphism is given by the
natural logarithm function $x \mapsto \ln x$. We know this function is
invertible (its inverse is the exponential function $x \mapsto e^x$)
so it is a bijection of $\R^+$ onto $\R$. Furthermore, the equation
$\ln(ab) = \ln(a) + \ln(b)$ says that products in $\R^+$ match up with
sums in $\R$, so the function $\ln$ is indeed a group isomorphism, as
claimed.
\end{example}


If a group $(G,*)$ is finite then it may be described by giving its
complete multiplication table. (Replace multiplcation by addition if
it is an additive group.) For instance, the addition table of
$(\Z_4,+)$ and the multiplication table of the cyclic group $G =
\gen{\alpha}$ generated by a $4$-cycle $\alpha$ are displayed in the
tables below
\[
\begin{array}{|c|cccc|} \hline
+&0&1&2&3\\ \hline 
0&0&1&2&3\\ 1&1&2&3&0\\ 
2&2&3&0&1\\ 3&3&0&1&2\\ \hline
\end{array}
\qquad \qquad
\begin{array}{|c|cccc|} \hline
\cdot&1&\alpha&\alpha^2&\alpha^3\\ \hline 
1&1&\alpha&\alpha^2&\alpha^3\\ \alpha&\alpha&\alpha^2&\alpha^3&1\\ 
\alpha^2&\alpha^2&\alpha^3&1&\alpha\\ \alpha^3&\alpha^3&1&\alpha&\alpha^2\\ 
\hline
\end{array}
\]
where we write the numbers $0, 1, 2, 3$ as a shorthand for the
corresponding residue classes $[0], [1], [2], [3]$ in $\Z_4$. Note
that the correspondence 
\[
  [0] \to 1 = \alpha^0, \quad [1] \to \alpha = \alpha^1, \quad [2] \to
  \alpha^2, \quad [3] \to \alpha^3
\] 
defines an isomorphism between the two groups. In a more succinct
notation, the isomorphism is defined by $f([x]) = \alpha^x$ for $x =
0,1,2,3$.


The multiplication table of a finite group has the property that the
elements in any row of the table form a permutation of the elements of
any other row. The same is true of the columns of the table. In
particular, no element appears twice in any row or any column. You
should be able to show how these claims follow from the group axioms.

Consider a group of order $2$. Assume that its binary operation is
written multiplicatively, so $G = \{1, a\}$ as a set, where $1$ is the
identity element and $a \ne 1$ (else the group would have order
$1$). Then necessarily $a^2=1$, because $a^2=a$ implies $a=1$. So the
group multiplication table must look like the one on the left below:
\[
\begin{array}{|c|cc|} \hline
\cdot&1&a\\ \hline 
1&1&a\\
a&a&1\\
\hline
\end{array}
\qquad \qquad
\begin{array}{|c|cc|} \hline
+&0&1\\ \hline 
0&0&1\\
1&1&0\\
\hline
\end{array}
\]
The table on the right is the addition table for the additive group
$(\Z_2,+)$. It should be clear that the two groups are
isomorphic. This analysis shows that any group of two elements must be
isomorphic to $\Z_2$. This argument can be extended to prove that any
group of three elements must be isomorphic to $\Z_3$.


%\newpage
\section*{Exercises}
\begin{problems}\small

\item Explain your reasoning for:
\begin{enumerate}
\item Is $\N = \{1,2,3,\dots \}$ a group under addition? If we
  include $0$ is it a group?
\item Is $\N= \{1,2,3,\dots \}$ a group under multiplication?
\item Is the set $\Z$ of integers a group under addition? 
\item Is $\Z$ a group under multiplication? 
\item Is $\Z-\{0\}$ a group under multiplication? 
\end{enumerate}


\item What is wrong with writing $\frac{a}{b}$ for $ab^{-1}$ in a
  (nonabelian) multiplicative group? If you think there is nothing
  wrong with it, then how will you write $b^{-1} a$ when $ab^{-1} \ne
  b^{-1}a$?

\item Prove that any group of three elements must be isomorphic to
  the additive group $\Z_3$ by analyzing its multiplication table.

\item Suppose that $G = \{1,a,b,c\}$ is a multiplicative group of four
  elements in which $1$ is the identity element. By analyzing the
  possible multiplication tables, prove that $G$ is isomorphic to
  either $(Z_4,+)$ or to a group in which $a^2=b^2=c^2=1$. (The latter
  group is called the Klein 4-group\index{Klein~four~group}.)

\item List the elements in the following multiplicative groups:

(a) $(\Z^\times,\cdot)$,\qquad
(b) $(\Z_6^\times, \cdot)$,\qquad
(c) $(\Z_8^\times, \cdot)$,\qquad
(d) $(\Z_{15}^\times, \cdot)$.

\item Give multiplication tables for the groups in the previous
  problem.

\item Prove that $|\Z_n^\times| = \varphi(n)$, where $\varphi(n)$ is
  \emph{Euler's phi-function} from number theory.

\item Prove that the elements in any row of the group multiplication
  table of a finite group $G$ form a permutation of the elements of
  the first row. Then do the same for columns. 

\item In this problem, we write $\Z_n$ for the additive group
  $(\Z_n,+)$. Find the order of:

(a) 1 in $\Z_7$, \qquad 
(b) 2 in $\Z_7$, \qquad 
(c) 1 in $\Z_{10}$, \qquad 
(d) 2 in $\Z_{10}$, \qquad 
(e) 3 in $\Z_{10}$.

\item In this problem, we write $\Z_n$ for the additive group
  $(\Z_n,+)$. Find the order of any $a \in \Z_n$ and prove your
  answer.

\item In this problem, we write $\Z_n^\times$ for the multiplicative
  group $(\Z_n^\times, \cdot)$ of units. Find the orders of:

(a) $1,2,3,4,5,6$ in $\Z_7^\times$, \qquad
(b) $1,2,4,5,7,8$ in $\Z_{9}^\times$, \qquad
(c) $1,3,7,9$ in $\Z_{10}^\times$.


\item\label{prob:circle-gp} (The circle group) Let $S^1$ be the set of
  all points on the usual unit circle in the plane $\R^2$.  Show that
  $S^1$ is a group under the law of combination given by
  \[
    (\cos \theta, \sin \theta) * (\cos \theta', \sin \theta') = (\cos
    (\theta + \theta'), \sin(\theta + \theta')).
  \] 
  Be sure to give a formula for the inverse of elements of this group,
  and prove that they really are inverses. 

\item Show that the circle group of the previous problem is isomorphic
  to the matrix group $\SO(2)$.


\item Find an isomorphism of the multiplicative group $\Z^\times$ onto
  the additive group $\Z_2$.



\item Consider the set $G$ of all $2 \times 2$ real matrices of the
  form $[\begin{smallmatrix} a&-b\\b&a \end{smallmatrix}]$.
  \begin{enumerate}
  \item Show that $G$ is a group under ordinary matrix addition, and
    find an isomorphism from the additive group $(\C, +)$ onto $G$.
  \item Now let $G'$ be the subset of $G$ consisting of all elements
    of $G$ except the zero matrix. Show that $G'$ is a group under
    ordinary matrix multiplication. 
  \item Find an isomorphism from the multiplicative group $(\C^\times,
    \cdot)$ onto $G'$.
  \end{enumerate}

\item Prove part (f) of Theorem \ref{gpprop} using right
cancellation instead of left cancellation.


\item Prove that isomorphism of groups is an equivalence relation
on the class of groups (Theorem \ref{thm:iso}).


\item Prove that if $G$ is a group in which every element (except the
  identity) has order 2 then $G$ must be abelian.

\item \label{prob:monoids} (Monoids) A \emph{monoid} is a set $M$
  along with a binary operation $*: M \times M \to M$ such that $*$ is
  associative and there is an identity element $e \in M$. Show that
  $(\N, +)$ and $(\Z, \cdot)$ are monoids but not groups.


  

\item Show that a set $R$ with two binary operations $+, \cdot$ is a
  ring\index{ring} if and only if the following three properties hold:
  \begin{enumerate}
  \item $(R,+)$ is an additive abelian group. Denote its identity
    element by $0$.
  \item $(R, \cdot)$ is a multiplicative monoid (see Problem
    \ref{prob:monoids} for the definition of monoid). Denote its
    identity element by $1$.
  \item Addition and multiplication are connected by the distributive
    laws: $a(b+c) = ab+ac$ and $(b+c)a = ba+ca$, for all $a,b,c \in
    R$.
  \end{enumerate}
  Thus, one could take the three properties as the definition of ring.



\item Show that a set $F$ with two binary operations $+, \cdot$ is a
  field\index{field} if and only if the following four properties hold:
  \begin{enumerate}
  \item $(F,+)$ is an additive abelian group. Denote its identity
    element by $0$.
  \item $(F-\{0\}, \cdot)$ is a multiplicative abelian group. Denote
    its identity element by $1$.
  \item $1 \ne 0$.
  \item Addition and multiplication are connected by the distributive
    law: $a(b+c) = ab+ac$, for all $a,b,c \in F$.
  \end{enumerate}
  Thus, one could take the four properties as the definition of field.

\end{problems}

 

\newpage
\section{Subgroups}\noindent
Finding subgroups inside known groups is an important way of finding
new examples of groups. 

\begin{defn}\index{subgroup}
  Let $(G,*)$ be a group. A subset $H$ of $G$ is called a
  \emph{subgroup} of $G$ if $(H,*)$ is a group in its own right. We
  write $H<G$ (or $H \le G$ interchangeably) to denote that $H$ is a
  subgroup of $G$. A subgroup $H$ is a \emph{proper} subgroup of $G$
  (written as $H \lneqq G$) if $H \ne G$.
\end{defn}


By definition, permutation groups are subgroups of some $\Sym_n$ and
matrix groups are subgroups of some $\GL_n(F)$, where $F$ is a field.
So we have already seen many examples of subgroups.

Note that every group is regarded as a subgroup of itself. (So in the
above notation, writing $G<G$ is perfectly valid.) In any group the
subset consisting solely of the identity element is always a subgroup;
this subgroup is called the \emph{trivial} subgroup.

In order for a given subset $H$ of a group $G$ to be a subgroup, it is
clearly necessary that the binary operation $*: G \times G \to G$
restricts to a binary operation $*: H \times H \to H$. This is just
another way of saying that $H$ must be closed under products.

The following theorem covers both the multiplicative and additive
cases together. In the latter case, you should read ``sum'' for
``product'' in the theorem, because $a*b = a+b$ when the operation $*$
is equal to $+$. 


\begin{thm}[The subgroup criterion]\index{subgroup~criterion}
  Let $H$ be a nonempty subset of a given group $(G,*)$.  Then $H$ is
  a subgroup of $G$ if and only if $H$ is closed under products and
  inverses. (Closure under products means that $a*b \in H$ whenever
  $a,b \in H$, and closure under inverses means that $H$ contains the
  inverse of each of its elements.)
\end{thm}

\begin{proof}
  ($\implies$) Suppose that $H$ is a subgroup of $G$. Then the fact
  that $H$ is a group in its own right means that when we restrict the
  operation $*\colon G \times G \to G$ to the subset $H \times H$, it
  induces a binary operation $*\colon H \times H \to H$. This is
  equivalent to saying that $H$ is closed under products. Also, the
  fact that axiom (G3) holds for $H$ means that each $a \in H$ has an
  inverse $a'$ in $H$; that must also be its inverse in $G$ since
  inverses in $G$ are unique.  This implies that $H$ is closed under
  inverses.

  ($\impliedby$) Suppose that $H$ is a nonempty subset of $G$ which is
  closed under products and inverses. Then the restriction of $*$ to
  $H \times H$ maps into $H$, and thus defines a binary operation on
  $H$. Axiom (G1) is automatic in $H$ since it holds in the bigger set
  $G$. Axiom (G3) for $H$ is just closure under inverses, which is
  true by assumption. Finally, closure under products and inverses
  implies that the identity $e$ is in the set $H$, since there must be
  some $a \in H$ (because $H$ is nonempty) and then its inverse $a'$
  is in $H$ and thus $a*a' = e \in H$ by closure under products. This
  proves axiom (G2) for $H$.
\end{proof}

The following result is a slightly simplified version of the subgroup
criterion. 


\begin{thm}[Simplified subgroup criterion]
  Let $H$ be a nonempty subset of a given group $G$, and write $b'$
  for the inverse of $b \in G$.  Then $H$ is a subgroup of $G$ if and
  only if $a*b' \in H$ for all $a,b \in H$.
\end{thm}

\begin{proof} 
($\implies$) Suppose $H<G$. Then by the subgroup criterion, for any
  $a,b \in H$ it follows that $b' \in H$ and hence $a*b' \in H$.

($\impliedby$) For the converse, suppose that $H$ is a nonempty
  subset and $a*b' \in H$ for all $a,b \in H$. Since $H$ is non-empty
  there is at least one element $c \in H$. Hence $c*c' \in H$, so the
  identity $e \in H$.  Hence $b' = e*b' \in H$ for every $b\in H$,
  proving that $H$ is closed under inverses. Finally, if $a,b$ are any
  elements of $H$, then $b'\in H$ as we have just shown. Note that
  $(b')' = b$, so $b = d'$ where $d = b' \in H$. Hence the product
  $a*b = a*d'$ must be an element of $H$. This shows that $H$ is
  closed under products. So $H<G$ by the subgroup criterion.
\end{proof}


The criterion for finding subgroups of a \emph{finite} group is even
simpler: we only have to check closure under products.

\begin{cor}[Subgroup criterion for finite groups]
  Let $H$ be a nonempty subset of a given finite group $G$.
  Then $H$ is a subgroup of $G$ if and only if $a*b \in H$ for all
  $a,b \in H$.
\end{cor}

\begin{proof}
Every element of a finite group must have finite order, so closure
under products implies also closure under inverses. (By Lemma
\ref{lem:inverse-if-finite}, if $a$ has order $r$ then the inverse
of $a$ is obtained by combining $a$ with itself $r-1$ times.)
\end{proof}


\begin{example}
We compute the subgroups of $\Sym_3$, the symmetric group on 3 letters,
using the finite subgroup criterion. We have (in the
cycle notation)
\[
\Sym_3 = \{(1), (1,2), (2,3), (1,3), (1,2,3), (3,2,1) \}.
\] 
Here we use $(1)$ for the identity permutation. The smallest subgroup
of $\Sym_3$ is the trivial subgroup $\{(1)\}$. Next we have the two
element subgroups $\{(1), (1,2) \}$, $\{(1), (2,3)\}$, and
$\{(1),(1,3)\}$. The subgroup $\{(1),(1,2,3), (3,2,1) \}$ is of order
3. Finally, we have $\Sym_3$ itself, a subgroup of order 6. It is easy
to check that these are the only subgroups of $\Sym_3$.
\end{example}


\begin{thm}\index{intersection~of~subgroups}
  The intersection of any number of subgroups of a given group $G$ is
  always a subgroup of $G$.
\end{thm}

\begin{proof}
This is an application of the subgroup criterion. Suppose that $I$ is
some indexing set and $H_i \le G$ for each $i \in I$. Then we need to
show that $K = \bigcap_{i \in I} H_i$ is a subgroup of $G$. Note that
$K$ is nonempty since the identity element belongs to each subgroup
$H_i$ and hence belongs to the intersection $K$. Suppose that $x,y \in
K$. Then $x, y \in H_i$ for all $i \in I$. Since $H_i$ is a subgroup,
this means that both $x*y$ and the inverse of $x$ are in $H_i$, for
all $i \in I$, so $x*y \in K$ and the inverse of $x$ is in $K$. By the
subgroup criterion, $K$ is a subgroup of $G$.
\end{proof}

In contrast, unions of subgroups are usually \emph{not} subgroups.


\begin{defn}\index{subgroup~generated~by~a~set}
  If $S$ is any set of group elements in some group $G$ then $\gen{S}$
  is the smallest subgroup of $G$ containing the elements of $S$. The
  subgroup $\gen{S}$ is called the subgroup \emph{generated by} the
  set $S$. In particular, if $a \in G$ is a group element, then we
  write $\gen{a}$ short for $\gen{\{a\}}$; this is called the
  \emph{cyclic subgroup generated by} $a$.
\end{defn}

\begin{examples}
1. If $a \in G$ has finite order $r$, then $\gen{a} = \{ 1, a, a^2,
\dots, a^{r-1} \}$ if $G$ is a multiplicative group, and $\gen{a} = \{
0, a, 2a, \dots, (r-1)a \}$ if $G$ is an additive group. In either
situation, $\gen{a}$ is isomorphic to the abstract cyclic group $C_r$
of order $r$.

2. In the additive group $\Z$ of integers, we have $\gen{1} = \Z$ and
$\gen{-1} = \Z$. Hence $\Z$ is a cyclic group under
addition). Furthermore, $\gen{0} = \{0\}$ (the trivial group) and
$\gen{2} = \gen{-2} = 2\Z$ (the subgroup of even integers).

3. In the multiplicative group $\R^\times$ of nonzero real numbers,
the subgroup $\gen{\pi} = \{ \pi^k \mid k \in \Z \}$. This group is a
proper subgroup of $\R^\times$, and it is isomorphic to the additive
group $\Z$. More generally, for \emph{any} chosen element $a \in
\R^\times$, it can be seen that $\gen{a} = \{ a^k \mid k \in \Z \}$.

4. Even more generally, suppose that $a \in R^\times$ is an element of
infinite order in the multiplicative group of units in a ring
$R$. Then $\gen{a} = \{ a^k \mid k \in \Z \}$ is an infinite cyclic
group isomorphic to the additive group $\Z$.
\end{examples}


\begin{defn}\index{generators}
  If $S$ is a set of elements of a group $G$, we say that $G$ is
  \emph{generated by} $S$ if $G = \gen{S}$. If there is a
  \emph{finite} set $S$ with this property then we say that $G$ is
  \emph{finitely generated}. A group is called \emph{cyclic} if it is
  generated by a single element: i.e., if $G = \gen{a}$ for some $a
  \in G$.
\end{defn}

If $G$ is generated by a set $S$, then every element of $G$ can be
expressed as a product of elements of $S$ and their inverses.


\begin{examples}
1. The symmetric group $\Sym_n$ is generated by the set of
transpositions it contains.  

2. The general linear group $\GL(n)$ is generated by the set of
elementary matrices.

3. The dihedral group $\D_n$ is generated by the two elements $r, d$
defined earlier, so $\D_n = \gen{r,d}$. (Recall that $r$ is a basic
rotation and $d$ a reflection.)

4. The additive group $(\Z,+)$ of integers is cyclic, because $\Z =
\gen{1}$. So is the additive group $(\Z_n,+)$ of integers modulo $n$,
because $\Z_n = \gen{[1]}$.

5. Every finite group is finitely generated. So is every cyclic group
(including any infinite cyclic group). 

6. The additive group $\R$ of real numbers is not finitely generated.

7. The matrix group $\GL(n)$ is not finitely generated. (We have
proved that the set $S$ of all elementary matrices generates $\GL(n)$,
but $S$ is an infinite set. It turns out that no finite generating set
exists, but it isn't so easy to prove.)
\end{examples}

Next we investigate another way to find subgroups from subsets of
elements of a given group.

\begin{defn}\index{centralizer}\index{center}
  Suppose that $S$ is a set of elements of a group $(G,*)$. The
  \emph{centralizer} of $S$ in $G$ is the subgroup $Z_G(S) = \{ x \in
  G \mid x*s = s*x \text{ for all } s \in S \}$. The \emph{center} of
  $G$ is $Z(G) = Z_G(G) = \{ x \in G \mid x*g = g*x \text{ for all } g
  \in G \}$, the centralizer of $G$ in itself. If $a \in G$ then we
  write $Z_G(a)$ short for $Z_G(\{a\})$.\index{ZG@$Z(G)$}
\end{defn}

It is an exercise to verify that centralizers really are subgroups. In
particular, this implies that the center $Z(G)$ of a group $G$ is
always a subgroup. The center is, by definition, the set of elements
that commute with all the elements of the group.

\section*{Exercises}

\begin{problems}

\item Show that the set $\{ \pm 1, \pm i\}$ is a subgroup of the
  multiplicative group $\C^\times$. Is it a cyclic group? 

\item Show by example that a union of two subgroups need not be a
  subgroup.

\item Show that the set $2\Z$ of all even integers is a subgroup of
  the additive group $\Z$, but the set $2\Z+1$ of all odd integers is
  not a subgroup.

\item Show that for any integer $n$, 
  \begin{enumerate}
  \item the set $n\Z$ of all multiples of $n$ is a subgroup of the
    additive group $\Z$.
  \item the subgroup $n\Z$ is isomorphic to $\Z$ itself. 
  \item these are the only subgroups of $\Z$; i.e., every subgroup of
    $\Z$ is of the form $n\Z$ for some integer $n$.
  \end{enumerate}

\item If $G$ is any subgroup of $\GL(n)$, let $H = \{ A \in G \mid
  \det A = \pm 1 \}$. Prove that $H < G$.
  
\item Let $F$ be a field. If $G$ is any subgroup of $\GL_n(F)$, let $H
  = \{ A \in G \mid \det A = \pm 1 \}$. Prove that $H < G$.

\item\index{Klein~four~group} \label{prob:Klein-4-group} (Abstract
  Klein 4-group) The abstract \emph{Klein 4-group} $K$ may be defined
  as the (unique) group $\{1,a,b,c\}$ of four elements such that $1$
  is the identity and $a, b, c$ all have order 2.
  \begin{enumerate}
  \item Show that this description determines the group $K$ uniquely,
    by writing out its only possible multiplication table.
  \item Find all the subgroups of $K$.
  \end{enumerate}


\item Prove that if $\alpha$ is an element of order $n$ in a
  permutation group $G$ then the subgroup $\gen{\alpha}$ generated by
  $\alpha$ is isomorphic to the additive group $(\Z_n, +)$.



\item\index{cyclic~group} \label{exer:cyclic} (Classification of
  cyclic groups) Prove that if $(G,*)$ is a cyclic group then it is
  isomorphic with either the additive group $\Z_n$ for some $n$ or
  with the additive group $\Z$ of all integers.

\item Apply the previous exercise to deduce that the additive group
  $(\R,+)$ is not cyclic. 

\item Show by contradiction that the additive group $(\Q, +)$ is not
  cyclic.

\item Show that every subgroup of a cyclic group must be
  cyclic. [Hint: Use the result of Exercise \ref{exer:cyclic}.]

\item Show that the group $(\Z_n,+)$ is generated by $[a] \in \Z_n$ if
  and only if $a,n$ are relatively prime. Use this to deduce that a
  cyclic group of order $n$ has exactly $\varphi(n)$ generators.
  [Hint: Use the result of Exercise \ref{exer:cyclic} for the second
    part.]


\item If $H$ is a subgroup of a group $(G,*)$ and $a \in G$, let
  $a*H*a' = \{a*h*a' \mid h \in H \}$, where $a'$ is the inverse of
  $a$. 
  \begin{enumerate}
  \item Show that $a*H*a'$ is a subgroup of $G$. 
  \item If $H$ is finite, say $|H|=n$, then what is $|a*H*a'|$?
  \end{enumerate}

\item Show that the dihedral group $\D_n$ ($n \ge 3$) is not cyclic.

\item Show that if $G$ is a group of order $n$ then $G$ is cyclic if
  and only if it has an element of order $n$.

\item Write $\Z_n$ for the additive group $(\Z_n,+)$. Show that $\Z_n
  = \gen{a}$ for $a \in \Z_n$ if and only if $\gcd(a,n) = 1$.

\item Show that the multiplicative group $\F_7^\times$ is cyclic by
  finding a generator. Do the same for $\F_{13}^\times$.

\item Is the multiplicative group $\Z_8^\times$ cyclic? Same question
  for $\Z_{10}^\times$. Justify your answers.

\item Find a minimal generating set for the Klein 4-group (see Exercise
  \ref{prob:Klein-4-group}).

\item Show that the matrix group $\O(2)$ is generated by the set
  $\SO(2) \cup \{A\}$, where $A \in \O(2)$ is any improper orthogonal
  matrix.

\item Show that if $a,b$ are elements of some multiplicative group $G$
  then $\gen{a} < \gen{b}$ if and only if $a = b^k$ for some integer
  $k$.


\item\index{quaternion~group} (The quaternion group) The quaternion
  group is the group $Q = \{ \pm 1, \pm i, \pm j, \pm k\}$ of order 8
  defined by the following multiplication table:
  \[
  \begin{array}{r|rrrrrrrr|}
  \cdot&1&-1&i&-i&j&-j&k&-k\\ \hline
  1 & 1&-1&i&-i&j&-j&k&-k\\
  -1& -1&1&-i&i&-j&j&-k&k\\
  i & i&-i&-1&1&k&-k&-j&j\\
  -i& -i&i&1&-1&-k&k&j&-j\\
  j & j&-j&-k&k&-1&1&i&-i\\
  -j& -j&j&k&-k&1&-1&-i&i\\
  k & k&-k&j&-j&-i&i&-1&1\\
  -k& -k&k&-j&j&i&-i&1&-1\\ \hline
  \end{array}
  \]
  in which $1$ is the identity element, $i,j,k$ all behave like
  imaginary units in that $i^2 = j^2 = k^2 = -1$, and products of any
  pair chosen from $i,j, k$ behave like cross products of the standard
  unit vectors in $\R^3$.

  \begin{enumerate}
  \item Find the cyclic subgroups $\gen{-1}$, $\gen{i}$, $\gen{j}$,
    and $\gen{k}$.
  \item Show that $Q$ is not cyclic.
  \item Find a minimal set of generators of $Q$, and justify your
    answer.
  \item Compute the center $Z(Q)$.
  \end{enumerate}
  
  Note: The quaternion group $Q$ is related to Hamilton's
  \emph{quaternions}, which puts a division ring structure on
  Euclidean four dimensional space.

\item Let $C = \{z \in \C \colon |z| = 1\}$ be the set of all complex
  numbers of unit norm, where as usual the \emph{norm} (length) of a
  complex number $z = x+iy$ is defined to be $|z| = \sqrt{x^2+y^2}$.
  By Euler's identity $e^{i\theta} = \cos \theta + i \sin \theta$
  (valid for all $\theta \in \R$) it follows that $C = \{ e^{i\theta}
  \mid \theta \in \R \}$.
  \begin{enumerate}
  \item Prove that $C$ is a subgroup of the multiplicative group
    $\C^\times$. 
  \item Find an isomorphism from $S^1$ onto $C$, where $S^1$ is the
    circle group defined in a previous exercise.
  \end{enumerate}


\item If $G = \D_n$ find $Z_G(r)$ where $r$ is the basic
  rotation. Then find $Z_G(f)$ where $f$ is any reflection.

\item Prove that if $a \in G$ then $\gen{a} < Z_G(a)$. 

\item Prove that $Z_G(S) < G$ for any set $S$ of elements in a group
  $G$. Why does this also prove that the center $Z(G) < G$?

\item Compute the center of $\D_4$ and justify your answer.

\item Show that $Z(\Sym_n)$ for $n \ge 3$ is the trivial group.  What
  is $Z(\Sym_2)$?

\item Prove that $Z(G)$\index{center}\index{ZG@$Z(G)$} is always
  abelian.\index{abelian~group}

\item Show that $Z(\D_n)$ has order 1 or 2 depending whether $n$ is
  odd or even, respectively.


\item Prove that $Z(G) = G$ if and only if $G$ is abelian. 

\item \label{exer:adjacent} This problem is about permutations, written
  in terms of the cycle notation.
 \begin{enumerate}
  \item Show that $(1,3) = (2,3)(1,2)(2,3)$. 
  \item Show that $(1,4) = (3,4)(2,3)(1,2)(2,3)(3,4)$.
  \item Prove that for $j > 1$ we have $(1,j) =$
  $$(j-1,j)(j-2,j-1)\cdots(1,2)\cdots(j-2,j-1)(j-1,j).$$
  \item Prove that for $i < j$ we have $(i,j) =$
  $$(j-1,j)(j-2,j-1)\cdots(i,i+1)\cdots(j-2,j-1)(j-1,j).$$

 \noindent This shows that it is possible to write any transposition
 as a product of {\em adjacent} ones; i.e., ones of the form
 $(k,k+1)$.
 \end{enumerate}


\item \index{generators}\label{ex:adjgen} Prove that $\Sym_n$ is
  generated by the set $\{ (1,2), (2,3), \dots, (n-1,n) \}$ of
  \emph{adjacent} transpositions.  [Hint: Use Problem
    \ref{exer:adjacent}.]

\item\index{generators} \label{ex:twogen} 
  \begin{enumerate}
  \item Show that if $\alpha = (1,2)$, $\beta = (1,2,\dots,n)$ are
    permutations written in the cycle notation then for any $1 < i <n$
    we have $(i,i+1) = \beta^{i-1} \alpha (\beta^{i-1})^{-1} =
    \beta^{i-1} \alpha\beta^{n-i+1}$.

  \item Prove that $\Sym_n$ is generated by the set $S = \{ (1,2),
    (1,2,3,\dots,n) \}$. [Hint: Use part (a) and the result of the
    preceding exercise.]
  \end{enumerate}
\end{problems}

\end{document}
