\documentclass[11pt]{article}
\usepackage[nohead,margin=1.50in]{geometry} %set margins
\usepackage{amsmath,amssymb,amsthm,pdiag,amscd,epic} %packages
\usepackage{graphicx}  
\usepackage{enumitem}
\setlist{topsep=1pt,itemsep=0pt,parsep=1pt,leftmargin=0.7cm}
\setenumerate[1]{label=(\alph*)}

\newenvironment{problems}
{
 \begin{enumerate}[topsep=1pt,itemsep=0pt,parsep=2pt,leftmargin=0.6cm,%
 label={\arabic*.}, ref=\arabic*] \small
}
{
 \end{enumerate}
}

%%% Define some theorem and example environments. The starred versions
%%% are un-numbered and the unstarred versions are numbered.
\newtheoremstyle{plain}
  {\topsep}   % ABOVESPACE
  {\topsep}   % BELOWSPACE
  {\slshape}  % BODYFONT
  {0pt}       % INDENT (empty value is the same as 0pt)
  {\bfseries} % HEADFONT
  {.}         % HEADPUNCT
  {5pt plus 1pt minus 1pt} % HEADSPACE
  {}          % CUSTOM-HEAD-SPEC

\swapnumbers
\newtheorem{thm}{Theorem}[section]
\newtheorem{lem}[thm]{Lemma}
\newtheorem{prop}[thm]{Proposition}
\newtheorem{cor}[thm]{Corollary}
\newtheorem*{thm*}{Theorem}
\newtheorem*{lem*}{Lemma}
\newtheorem*{prop*}{Proposition}
\newtheorem*{cor*}{Corollary}

\theoremstyle{definition}
\newtheorem{defn}[thm]{Definition}
\newtheorem{example}[thm]{Example}
\newtheorem{examples}[thm]{Examples}
\newtheorem{rmk}[thm]{Remark}
\newtheorem{conv}[thm]{Convention}
\newtheorem*{defn*}{Definition}
\newtheorem*{example*}{Example}
\newtheorem*{examples*}{Examples}
\newtheorem*{rmk*}{Remark}
\newtheorem*{conv*}{Convention}

%%% Define some convenient abbreviations for common mathematical
%%% notations.
\newcommand{\R}{\mathbb{R}} % use \R for the real numbers
\newcommand{\C}{\mathbb{C}} % use \C for the complex numbers
\newcommand{\Z}{\mathbb{Z}} % use \Z for the integers
\newcommand{\Q}{\mathbb{Q}} % use \Q for the rationals
\newcommand{\N}{\mathbb{N}} % use \N for the natural numbers
\newcommand{\compose}{\circ} % functional composition
\renewcommand{\implies}{\Rightarrow}
\renewcommand{\iff}{\Leftrightarrow}
\newcommand{\F}{{\mathbb F}}
\newcommand{\gen}[1]{\langle #1 \rangle}
\newcommand{\End}{\operatorname{End}}
\newcommand{\GL}{\mathrm{GL}}
\newcommand{\SL}{\mathrm{SL}}
\renewcommand{\O}{\mathrm{O}}
\newcommand{\SO}{\mathrm{SO}}
\newcommand{\U}{\mathrm{U}}
\newcommand{\SU}{\mathrm{SU}}
\newcommand{\g}{\mathfrak{g}}
\newcommand{\transpose}{\mathsf{T}}
\newcommand{\B}{\mathcal{B}}
\newcommand{\Rep}{\operatorname{Rep}}
\newcommand{\Mat}{\operatorname{Mat}}
\newcommand{\inner}[2]{\langle #1, #2 \rangle}
\newcommand{\sgn}{\operatorname{sgn}}
\newcommand{\n}{\underline{\mathbf{n}}}
\newcommand{\Sym}{\mathbb{S}}
\newcommand{\Alt}{\mathbb{A}}
\newenvironment{perm}[2]{\left(\begin{smallmatrix}#1 \\ #2}{\end{smallmatrix}\right)}
\newcommand{\Map}{\operatorname{Map}}
\newcommand{\Rot}{\Theta}
\newcommand{\D}{\mathbb{D}}


\allowdisplaybreaks
\parskip=2pt

%\title{Document Title}
%\author{author's name}

\begin{document}%\maketitle

%\appendix
\setcounter{section}{0}
\renewcommand{\thesection}{\Alph{section}}
\section{Appendix: Symmetries of polynomials} 
\noindent
Symmetry also plays an important role in the study of polynomial
equations\index{polynomial}, where the symmetry groups turn out to be
permutation groups. This was Lagrange's motivation to study
permutations.  But the definition of the symmetry group
\index{symmetry~group} of a polynomial, which is now known as its
\emph{Galois group}\index{Galois~group}, is not as accessible as the
symmetry group of a geometric object such as a polygon.  In this
appendix, we indicate the definition only in an intuitive way.

Our approach is difficult to use, so it turns out to be useful to
replace it by something less intuitive, but ultimately easier to work
with. The modern approach uses the theory of {\em field
  extensions}. Field extensions are typically studied in a second
course in abstract algebra, but they will not be considered here.



Suppose we have an $n$th degree polynomial $p(x)$ with real number
coefficients, say
\[
  p(x) = x^n + c_1x^{n-1} + c_2x^{n-2} + \cdots + c_{n-1}x + c_n . \tag{$*$}
\] 
The coefficients are the real numbers $c_j$ for $j = 1, \dots, n$.
According to the fundamental theorem of algebra
\index{fundamental~theorem!of~algebra}, the polynomial $p(x)$
has exactly $n$ roots $z_1, z_2, \dots, z_n$ in the complex number
system $\C$, where we agree to list multiple roots according to their
multiplicity. 

\begin{examples}
1. The polynomial $(x-1)^3 = x^3 - 3x^2+3x-1$ has three identical
roots $1, 1, 1$. There is one root of multiplicity three (a triple root).

2. The polynomial $(x^2-2)^2 = x^4 - 4x^2 + 4$ has roots $\sqrt{2}$,
$\sqrt{2}$, $-\sqrt{2}$, $-\sqrt{2}$. There are two real roots, each
of multiplicity two (two double roots).

3. The polynomial $(x^2+1)^2 = x^4+2x^2+1$ has roots $i, i, -i, -i$.
There are two non-real complex roots, each of multiplicity two (two
double roots). Here $i$ stands for the \emph{imaginary unit} in the
complex number system $\C$. (By definition, $i^2= -1$.) 

4. The polynomial $x(x^3-3x+2) = x^4 - 3x^2 + 2x$ has roots
$0,1,1,-2$.  This has one root of multiplicity two (one double
root). You can verify the roots by expanding the product
$x(x-1)^2(x+2)$.
\end{examples}

For definiteness, we assume that the polynomial $p(x)$ in equation
($*$) above has \emph{rational} coefficients, and that $p(x)$ is
irreducible, meaning that it cannot be factored into polynomials (with
rational coefficients) of strictly smaller degree. If $p(x)$ had such
a factorization, then we could deal separately with the two smaller
polynomials in order to solve $p(x) = 0$, so it makes sense to
restrict our attention to irreducible polynomials. 

We also assume, for simplicity, that the roots of $p(x)$ are all
\emph{distinct}, i.e., all the roots have multiplicity one.

\begin{defn}
Under the above assumptions, we define the {\em Galois
  group}\index{Galois~group} of the polynomial $p(x)$ to be the
permutation group consisting of all permutations of the roots $\{z_1,
\dots, z_n\}$ which preserve any algebraic relations among them.
\end{defn}



For a general polynomial\footnote{A general polynomial is one whose
  coefficients are represented by variables. For instance, $ax^2 + bx
  + c$ is the general polynomial of degree 2 (the general quadratic),
  $ax^3+bx^2+cx+d$ is the general polynomial of degree $3$ (the
  general cubic), and so on.} of degree $n$ there will be only trivial
relations among the roots, and thus the Galois group of the polynomial
will be unconstrained; it will thus be the full symmetric group
$\Sym_n$ consisting of all permutations of the $n$ roots. For special
polynomials, however, there can be non-trivial relations among the
roots imposing constraints on the permutations in the group; in such
cases the Galois group of the polynomial is often smaller than
$\Sym_n$. In all cases, the Galois group of an $n$th degree polynomial
will be a permutation group contained in $\Sym_n$.

Let's look at some concrete examples.

\begin{example}\label{ex:1}
Consider the polynomial $p(x)=x^4+x^3+x^2+x+1$.  Clearly we have
$$
p(x) = (x^5-1)/(x-1)
$$ as you can see by clearing denominators and expanding the resulting
product. The roots of $p(x)$ are $z_1=e^{2\pi i/5}$, $z_2 = e^{4\pi
i/5}$, $z_3 = e^{6\pi i/5}$, $z_4= e^{8\pi i/5}$. These complex
numbers all lie on the unit circle in the complex plane $\C$. (Recall
that $e^{i \theta} = \cos \theta + i \sin \theta$ for any angle
$\theta$, where the imaginary unit $i$ satisfies $i^2 = -1$.)
The roots satisfy the relations
$$
 z_2 = z_1^2, \quad z_3 = z_1^3, \quad z_4 = z_1^4 
$$ and any other relations (e.g., $z_3^2 = z_1$) are consequences of
these, along with the fact that $z_i^5 = 1$ for any $i = 1, \dots,
4$. We are looking for permutations of the four roots that preserve
these relations. If $\alpha$ is such a permutation, then
$\alpha(z_2)$, $\alpha(z_3)$, and $\alpha(z_4)$ will be determined by
$\alpha(z_1)$. Now $\alpha(z_1)$ can be $z_1$, $z_2$, $z_3$, or $z_4$.
We consider these possibilities separately.

If $\alpha(z_1)=z_1$ then $\alpha(z_2) = \alpha(z_1)^2 = z_2$,
$\alpha(z_3) = \alpha(z_1)^3 = z_3$, and $\alpha(z_4) = \alpha(z_1)^4
= z_4$. Thus $\alpha = (1)$ is the identity
permutation. 

If $\alpha(z_1)=z_2$ then $\alpha(z_2) = \alpha(z_1)^2 = z_2^2 = z_4$,
$\alpha(z_3) = \alpha(z_1)^3 = z_2^3 = z_1$, and $\alpha(z_4) =
\alpha(z_1)^4 = z_2^4 = z_3$. Thus $\alpha$ is the 4-cycle
$(1,2,4,3)$.

If $\alpha(z_1)=z_3$ then $\alpha(z_2) = \alpha(z_1)^2 = z_3^2 = z_1$,
$\alpha(z_3) = \alpha(z_1)^3 = z_3^3 = z_4$, and $\alpha(z_4) =
\alpha(z_1)^4 = z_3^4 = z_2$. Thus $\alpha$ is the 4-cycle
$(1,3,4,2)$.

Finally, if $\alpha(z_1)=z_4$ then $\alpha(z_2) = \alpha(z_1)^2 =
z_4^2 = z_3$, $\alpha(z_3) = \alpha(z_1)^3 = z_4^3 = z_2$, and
$\alpha(z_4) = \alpha(z_1)^4 = z_4^4 = z_1$. Thus $\alpha =
(1,4)(2,3)$.

From these calculations it follows that the symmetry group of the
polynomial $x^4+x^3+x^2+x+1$ is the cyclic group generated by
the cycle $(1,2,4,3)$. This Galois group $G$ has order $4$. 
\end{example}


\begin{example} \label{ex:2}
Now consider the polynomial $p(x) = x^4 -10x^2 + 1$.
We can factor $p(x)$ as follows:
\begin{align*}
p(x) = (x^4 - 2x^2 + 1) - 8x^2 &= (x^2-1)^2 - (x\sqrt{8})^2\\
&= (x^2-1-2\sqrt{2}x)(x^2-1+2\sqrt{2}x) \\
&= (x^2-2\sqrt{2}x-1)(x^2+2\sqrt{2}x-1)
\end{align*}
and we can find its roots by setting each quadratic factor to zero and
using the quadratic formula. The roots are $z_1 = \sqrt{2}+\sqrt{3}$,
$z_2 = -\sqrt{2}+\sqrt{3}$, $z_3 = \sqrt{2}-\sqrt{3}$, $z_4 =
-\sqrt{2}-\sqrt{3}$. Notice that all the roots are real numbers in
this case.

The roots satisfy the relations:
\begin{align*}
z_1+z_4 &= 0 \\
z_2+z_3 &= 0 \\
(z_1+z_2)^2 &= 12 \\
(z_1+z_3)^2 &= 8 \\
(z_2+z_4)^2 &= 8 \\
(z_3+z_4)^2 &= 12 
\end{align*}
and again all other relations are consequences of these. We want all
permutations of the four roots which preserve these relations. You can
check that the permutations $(1,2)(3,4)$, $(1,3)(2,4)$, $(1,4)(2,3)$,
and the identity $(1)$ preserve the relations. No other permutation
does. Thus the symmetry group of the polynomial $p(x) = x^4 -10x^2 +
1$ is $$G = \{ (1), (1,2)(3,4), (1,3)(2,4), (1,4)(2,3) \}.$$ This is a
group of order four. It turns out that it is isomorphic with the Klein
four-group.\index{Klein~four~group}
\end{example}


\setcounter{equation}{0} 
\subsection*{The cubic formula}
We now consider the cubic formula, the degree three analogue of the quadratic
formula.  Given a general cubic polynomial equation:
\begin{equation}\label{eq:cubic}
  x^3 - bx^2 + cx - d = 0
\end{equation} 
with undetermined (i.e., general) coefficients, we denote its complex
roots by $z_1$, $z_2$, and $z_3$. To solve the equation, we first
substitute $x=y+b/3$ and obtain (after expanding) the {\em reduced
  cubic} equation\index{cubic equation}
\begin{equation}\label{eq:reduced-cubic}
  y^3 +py - q = 0
\end{equation}
where $p = c-b^2/3$ and $q = d - bc/3 + 2b^3/27$. The roots of the
reduced cubic are given by \emph{Cardano's formula}, first published
in the year 1545.

\begin{thm*}[Cardano 1545]\index{Cardano's formula}
The roots of the reduced cubic $y^3 +py - q = 0$ are given by
\begin{equation}\label{eq:Cardano}
\begin{aligned}
y_1 &= \sqrt[3]{\frac{q}{2}+\sqrt{R}} + \sqrt[3]{\frac{q}{2}-\sqrt{R}}\\
y_2 &= \omega^2 \sqrt[3]{\frac{q}{2}+\sqrt{R}} 
     + \omega \sqrt[3]{\frac{q}{2}-\sqrt{R}}\\
y_3 &= \omega \sqrt[3]{\frac{q}{2}+\sqrt{R}} 
     + \omega^2 \sqrt[3]{\frac{q}{2}-\sqrt{R}}\\
\end{aligned}
\end{equation}
where 
\[
  R = \frac{q^2}{4} + \frac{p^3}{27}; \qquad \omega = \frac{-1 + i
    \sqrt{3}}{2}.
\]
\end{thm*}

Note that $\omega = e^{i 2\pi/3}$ and therefore $\omega^3 =
1$. Moreover, $\omega^2 + \omega + 1 = 0$. The complex number $\omega$
is called a {\em primitive cube root of unity}.

To find the roots $z_i$ ($i = 1,2,3$) of the original cubic we only
have to add $b/3$ to the $y_i$.  This solves the original cubic equation,
and the solution is in terms of certain radicals (square and cube
roots), namely those that occur in the expressions for the $y_i$. 



\subsection*{Lagrange's method for solving a cubic}
\index{Lagrange's~method}%
In an important paper published in 1770, Lagrange noticed that the
radicals appearing in the solution to the
original cubic are expressible as functions of the roots $z_i$
themselves. This observation was the beginning of the link between
group theory and polynomial equations.

To see this for the cube roots, we multiply the equations for the
$y_i$ by 1, $\omega$, $\omega^2$ respectively and add, obtaining
$$
3 \sqrt[3]{\frac{q}{2}+\sqrt{R}} = y_1 + \omega y_2 + \omega^2 y_3.
$$
Thus if we substitute $y_i = z_i - b/3$ we obtain
$$
3 \sqrt[3]{\frac{q}{2}+\sqrt{R}} = z_1 + \omega z_2 + \omega^2 z_3.
$$
Denote this value by $\varphi_1$, and set
$\varphi_3=\omega^2\varphi_1$, $\varphi_5=\omega\varphi_1$, so that we
have equalities
\begin{align*}
\varphi_1 &= 3 \sqrt[3]{\frac{q}{2}+\sqrt{R}} \\
\varphi_3 &= 3 \omega^2 \sqrt[3]{\frac{q}{2}+\sqrt{R}} \\
\varphi_5 &= 3 \omega \sqrt[3]{\frac{q}{2}+\sqrt{R}}.
\end{align*}
Going back and multiplying the equations for $y_1$, $y_2$, $y_3$ by
$1$, $\omega^2$, $\omega$ respectively, we obtain after adding them
and substituting for $y_i = z_i - b/3$ the equality
$$
3 \sqrt[3]{\frac{q}{2}-\sqrt{R}} = z_1 + \omega^2 z_2 + \omega z_3.
$$ 
Let us denote this value by $\varphi_2$, and set $\varphi_4 =
\omega \varphi_2$, $\varphi_6 = \omega^2 \varphi_2$. Then we have
equalities 
\begin{align*}
\varphi_2 &= 3 \sqrt[3]{\frac{q}{2}-\sqrt{R}} \\
\varphi_4 &= 3 \omega \sqrt[3]{\frac{q}{2}-\sqrt{R}} \\
\varphi_6 &= 3 \omega^2 \sqrt[3]{\frac{q}{2}-\sqrt{R}}.
\end{align*}
We can also express $\sqrt{R}$ in terms of the roots $z_i$ by cubing
the equations for $\varphi_1$ and $\varphi_2$ and subtracting. Doing
this, we obtain
$$ 54 \sqrt{R} = (z_1+\omega z_2+\omega^2 z_3)^3 - (z_1+\omega^2
z_2+\omega z_3)^3
$$
and after expanding, simplifying, and factoring we have
$$
18\sqrt{R} = 3i (z_1-z_2)(z_1-z_3)(z_2-z_3).
$$
Thus all the radicals appearing in the formulas for the roots are
expressible as nice functions of the roots themselves. 


Lagrange noticed that the roots of the general cubic are obtainable
from the six values $\varphi_i$ ($i = 1, \dots, 6$), because we have
the system of linear equations 
\begin{equation}\label{eq:Lagrange-system}
\begin{aligned}
\varphi_1 &= z_1 + \omega z_2 + \omega^2 z_3\\
\varphi_2 &= z_1 + \omega^2 z_2 + \omega z_3\\
b & = z_1 + z_2 +z_3
\end{aligned}
\end{equation}
which can easily be solved for the roots $z_1$, $z_2$, $z_3$ by first
adding them as they stand, then adding them after multiplying by
$\omega^2$, $\omega$, 1, and then adding them again after multiplying
by $\omega$, $\omega^2$, 1, respectively.

Lagrange also pointed out that we can compute the six values
$\varphi_i$ as follows. Set $A_1 = q/2 + \sqrt{R}$ and $A_2 = q/2 -
\sqrt{R}$. Then we have the equations
$$
\left(\frac{\varphi_i}{3}\right)^3 = A_1 \qquad (i = 1,3,5)
$$
and 
$$
\left(\frac{\varphi_i}{3}\right)^3 = A_2 \qquad (i = 2,4,6).
$$ 
In other words, the six $\varphi_i$ are obtained by solving the
equations $X^3 = 27A_1$ and $X^3 = 27A_2$. (Note that if $\varphi$ is one
root, say the cube root of $A_i$, then the other two roots must be $\omega
\varphi$ and $\omega^2 \varphi$.)

Examining the expressions defining $A_1$ and $A_2$, we see that we can
find the $A_i$ by solving the quadratic equation
\begin{equation}\label{eq:Lagrange-resolvant}
   A^2 - q A - p^3/27 = 0
\end{equation}
for $A$. In this way Lagrange reduced the solution of the cubic to the
solution of a quadratic. (He called this associated quadratic equation
the {\em resolvent}.) By first solving the resolvent quadratic and
then solving the system \eqref{eq:Lagrange-system} you can solve the
cubic using Lagrange's method.


Lagrange also similarly analyzed the general
quartic\index{quartic~equation} (degree 4) polynomial equation,
reducing it to a cubic resolvent equation. He must have been quite
excited at that point, thinking that he had discovered a general
scheme to solve all polynomial equations inductively by reducing them
to a resolvent equation of degree one less than the given equation.
However, when he looked at the very next case, the general quintic
(degree 5) equation, he found that the resolvent equation had degree
6. This was a strong indication that the general quintic might be
unsolvable in terms of radicals, and indeed N.\ H.\ Abel proved that
very result in 1826.

\begin{thm*}[Abel 1826]\index{Abel's theorem}
  The general quintic (degree five) polynomial equation cannot be
  solved in terms of radicals.
\end{thm*}

Galois further developed group theory and the theory of polynomial
equations. He was able to give necessary and sufficient conditions on
the symmetry group of an arbitrary polynomial (of any degree) for it
to be solvable in terms of radicals. His theorem is a vast
generalization of Abel's theorem, in that it applies to polynomials of
any degree. Galois discovered the theorem in 1830 at the age of 18; he
was shot and killed in a duel at the age of 20. The theorem remained
unpublished until 1846, and it wasn't until the late 1800s that
mathematicians generally understood the significance of his results.





\section*{Exercises}

\begin{problems}

\item Solve the polynomial equation $p(x)=0$ of \ref{ex:2} by setting
  $y=x^2$ and solving the resulting quadratic equation, and then
  finding $x$ by taking square roots of $y$.  Compare your answer with
  the roots of $p(x)$ given in \ref{ex:2}. Can you explain?

\item One strategy for solving a polynomial equation $p(x)=0$ is to
  guess a root $z_1$ of the equation. The guess can be checked by
  substitution in $p(x)$. If $p(z_1)=0$ then $z_1$ is a root. Once you
  have found a root $z_1$, you can use long division of polynomials to
  divide $p(x)$ by $x-z_1$. Since roots correspond to linear factors,
  the fact that $z_1$ is a root means that when you divide you will
  obtain a quotient polynomial $q(x)$ such that $p(x) = (x-z_1) q(x)$.
  Now you have reduced the original problem, of solving $p(x)=0$, to
  the smaller problem of solving $q(x)=0$. By repeating the method,
  you can eventually factor $p(x)$ completely into linear factors,
  thus solving the polynomial equation.  The problem with this
  strategy is that it is not always possible to guess a
  solution. Thus, the method may never get started, or it may stall
  somewhere along the way. However, it works in a surprising number of
  cases. Use this method to solve the following cubic equations, using
  the given guess:
  \begin{enumerate}
  \item $x^3+1 = 0$; guess $z_1 = -1$.
  \item $x^3-3x^2+3x-1 = 0$; guess $z_1 = 1$.
  \item $x^3+2x+3 = 0$; find your own guess.
  \end{enumerate}

\item If a polynomial $p(x)$ with \emph{integer} coefficients has a
  rational root (a root of the form $\frac{r}{s}$ where $r,s$ are
  integers), then we can always find it using the \emph{rational roots
  theorem}.
  \begin{thm*}
    Suppose that
    $p(x) = a_0 x^n + a_1 x^{n-1} + \cdots + a_{n-1} x + a_n$, where
    $a_0 \ne 0$ and $a_k \in \Z$ for all $k = 0, 1, \dots, n$. If
    $z_1 = \pm \frac{r}{s}$ is a rational root of $p(x)$ then
    $s \mid a_0$ ($s$ divides $a_0$) and $r \mid a_n$ ($r$ divides
    $a_n$).
  \end{thm*}
  Use the rational roots theorem to find all rational roots (if any)
  of the following polynomials:
  \begin{enumerate}
  \item $3x^3+4x-7$.
  \item $x^3+x^2+x+2$.
  \item $x^3+8$.
  \end{enumerate}

\item Find all the roots of the polynomials in parts (a), (c) of the
  previous problem.

\item Show that the symmetry group $G$ of Example \ref{ex:2} is
  isomorphic with the Klein 4-group. (The Klein 4-group was introduced
  in Section 7.)

\item Use Cardano's formula or Lagrange resolvents to solve the
  following cubic equations:
  \begin{enumerate}
  \item $x^3 - 3x + 2 = 0$. 
  \item $x^3 - 9x^2 + 24x - 16 = 0$.
  \item $x^3 + 3x^2 + 6x + 2 = 0$.
  \item $x^3 + 3 x - 4 = 0$. 
  \end{enumerate}
Show the steps of your calculations. Notice that by inspection $1$ is
a root of the first and last equations; did your calculations in part
(d) reveal that fact? Do you think something is wrong?


\item Let $\omega = e^{2\pi i/n} = \cos(2\pi/n) + i \sin(2\pi/n)$ be a
  primitive $n$th root of unity. Prove that the roots of $x^n - 1 = 0$
  are the complex numbers $z_k = \omega^k$ for $k = 0, 1, \dots,
  n-1$. (This is a complete list of $n$ distinct roots.) 
  \begin{enumerate}
  \item Compute $\omega^n$.
  \item Why is the set $G$ of roots of the polynomial a group? What type
    of group is it?
  \item If you plot the roots in the complex plane, they form the set
    of vertices of what geometric figure?
  \end{enumerate}

\end{problems}
\renewcommand{\thesection}{\arabic{section}}
\end{document}







\subsection*{Galois group of the general cubic} 
Let us determine the Galois group of the general cubic. We continue to
use all the notation introduced in \ref{ss:cubic}. Tabulating all the
expressions for the $\varphi_i$ we have
\begin{align*}
\varphi_1 &= z_1+\omega z_2 + \omega^2 z_3 \\
\varphi_3 &= z_2+\omega z_3 + \omega^2 z_1 \\
\varphi_5 &= z_3+\omega z_1 + \omega^2 z_2 \\
\varphi_2 &= z_1+\omega z_3 + \omega^2 z_2 \\
\varphi_4 &= z_2+\omega z_1 + \omega^2 z_3 \\
\varphi_6 &= z_3+\omega z_2 + \omega^2 z_1.
\end{align*}
If we examine these expressions, we might notice that all the six
values $\varphi_i$ are obtained from $\varphi_1$ by permuting in all
possible ways the symbols $z_1, z_2, z_3$. This observation led
Lagrange to investigate permutation groups.

The six equations above are (some of) the relations which the roots of
the general cubic equation must satisfy, and any permutation of the
roots will preserve them. This suggests that the Galois group of the
general cubic should be the entire symmetric group $\Sym_3$, and
indeed that is so (but not so easy to prove). Some other relations
satisfied by the roots are
\begin{align*}
b &= z_1+z_2+z_3\\
c &= z_1z_2+z_1z_3+z_2z_3\\
d &= z_1 z_2 z_3.
\end{align*}
Clearly these relations are also preserved by all permutations of the
roots.

%%%%%%%%%%%%%%%%%%%%%%%%%%%%%%% exercises %%%%%%%%%%%%%%%%%%
\item Prove that $r h r = h$ in $\D_n$ by giving a algebraic argument
  with matrices. Use a rotation matrix to represent $r$ and use an
  appropriate matrix to represent $h$ viewed as a reflection across
  the horizontal coordinate axis. (We can always position the regular
  $n$-gon so that this reflection is a symmetry, so we lose no
  generality by this assumption.)

\item We know that the rotation subgroup $\Rot_n$ of $\D_n$ is cyclic for
  any $n \ge 3$, generated by the basic rotation of $2\pi/n$
  radians. Let $\Rot_\infty$ be the rotation subgroup of $\D_\infty$. Is
  $\Rot_\infty$ cyclic, too? Justify your answer.

\item A {\em permutation matrix} is a matrix obtained from the
  identity matrix by some permutation of its columns. It is a matrix
  which has just one 1 in each row and column, and all other entries
  are zero.
\begin{enumerate}
\item Prove that $\Sym_n$ is isomorphic with the subgroup of
  permutation matrices in $\GL_n(F)$ by considering the matrix
  representation $f\colon \Sym_n \to \GL_n(F)$ given by $f(\alpha) =
  P_\alpha$, where for $\alpha \in \Sym_n$ we let $P_\alpha$ be the
  permutation matrix obtained by applying the permutation $\alpha$ to
  the columns of the identity matrix $I$.
\item Show that any permutation representation $\delta: G \to \Sym_n$
gives rise to a corresponding matrix representation $\rho: G \to
\GL_n(F)$ for any field $F$. (Thus permutation representations are
subsumed in the theory of matrix representations.)
\end{enumerate}

\item\label{exer:present-S_n} (Difficult) Find a presentation of
  $\Sym_n$ by generators and relations. We already know a nice
  generating set: namely the set of special transpositions
  $t_1=(1,2)$, $t_2=(2,3)$, $t_3=(3,4)$, $\dots,$ $t_n=(n-1, n)$.
  These special transpositions swap consecutive elements of the set
  $\mathbf{n} = \{1, \dots, n\}$. Find a set of relations between
  these generators $t_1, \dots, t_n$ that determines the symmetric
  group $\Sym_n$.

\item (Difficult) It has been previously observed that $\Sym_n$ is
  generated by any transposition along with any $n$-cycle. Put $t =
  (1,2)$ and $c = (1,2,3,\dots, n)$ in the cycle notation; then
  $\Sym_n$ is generated by the set $\{t, c\}$. Find a set of relations
  satisfied by these generators that determines $\Sym_n$.

%%%%%%%%%%%%%%%%%%%%%%%%%%%%%%%%%%%%%%%%%%%%%%%%%%%%%%%%%%%%%
