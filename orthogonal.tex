\documentclass[12pt,oneside]{amsart}
\usepackage{amsthm,amssymb,amscd}
\usepackage[all]{xy}
\usepackage[margin=4cm]{geometry}
\usepackage{graphicx}  

\swapnumbers %to put numbers in the front of proclamations
\newtheorem{thm}[subsection]{Theorem}
\newtheorem{lem}[subsection]{Lemma}
\newtheorem{prop}[subsection]{Proposition}
\newtheorem{cor}[subsection]{Corollary}
\theoremstyle{definition}
\newtheorem{defn}[subsection]{Definition}
\newtheorem*{defn*}{Definition}
\newtheorem{example}[subsection]{Example}
\newtheorem*{example*}{Example}
\newtheorem{examples}[subsection]{Examples}
\newtheorem{exer}[subsection]{Exercise}
\theoremstyle{remark}
\newtheorem{rmk}[subsection]{Remark}
%\numberwithin{equation}{section}

%\allowdisplaybreaks



\newcommand{\N}{{\mathbb N}}
\newcommand{\Z}{{\mathbb Z}}
\newcommand{\Q}{{\mathbb Q}}
\newcommand{\R}{{\mathbb R}}
\newcommand{\C}{{\mathbb C}}
\newcommand{\F}{{\mathbb F}}
\newcommand{\compose}{\circ}
\newcommand{\gen}[1]{\langle #1 \rangle}
\newcommand{\End}{\operatorname{End}}
\newcommand{\GL}{\mathrm{GL}}
\newcommand{\SL}{\mathrm{SL}}
\renewcommand{\O}{\mathrm{O}}
\newcommand{\SO}{\mathrm{SO}}
\newcommand{\U}{\mathrm{U}}
\newcommand{\SU}{\mathrm{SU}}
\newcommand{\g}{\mathfrak{g}}
\newcommand{\transpose}{\mathsf{T}}
\newcommand{\B}{\mathcal{B}}
\newcommand{\Rep}{\operatorname{Rep}}
\newcommand{\Mat}{\operatorname{Mat}}
\newcommand{\inner}[2]{\langle #1, #2 \rangle}


\parskip=2pt
\renewcommand{\thesection}{\Alph{section}}

%\section*{}


\begin{document}
\title{Notes on Orthogonal Groups}
\author{} \maketitle

\noindent
Let $V$ be a vector space over an arbitrary field $F$.  The orthogonal
group $\O(V)$ is the group of isometries of $V$, where by
\emph{isometry} we mean a linear automorphism of $V$ preserving a
given quadratic form.

This pleasing uniform definition works in all characteristics, so long
as one adopts the following nonstandard definition of quadratic form.

\begin{defn*}
  A \emph{quadratic form} on $V$ is a scalar-valued function $q \colon
  V \to F$ satisfying the condition
  \[
    q( a u + b v ) = a^2 q(u) + ab\inner{u}{v} + b^2 q(v),
    \quad\text{for all } u,v \in V,\ a,b \in F
  \]
  where the function $(u,v) \mapsto \inner{u}{v}$ is a bilinear form
  on $V$. This bilinear form is sometimes called the polar bilinear
  form of $q$.
\end{defn*}

This version of the definition typically appears in the literature
only in case $F$ has characteristic $2$, although in the other cases
it is equivalent to the usual, slightly simpler, definition
\cite{Artin, Carter, Grove}.  By taking $b = 0$ in the definition we
obtain the property
\[ \tag{1}
  q(au) = a^2 q(u), \quad \text{all } a \in F, u \in V.
\]
We note that the bilinear form is uniquely determined by the quadratic
form $q$, since by taking $a=b=1$ in the definition we obtain
\[ \tag{2}
  \inner{u}{v} = q(u+v) - q(u) - q(v), \quad\text{all } u,v \in V.
\]
Hence the bilinear form is necessarily symmetric: $\inner{u}{v} =
\inner{v}{u}$ for all $u,v \in V$.  It follows from (2) that if
$\alpha$ is an isometry of $V$ then $\alpha$ must preserve the
bilinear form: $\inner{\alpha(u)}{\alpha(v)} = \inner{u}{v}$ for all
$u,v \in V$. The converse, however, fails in characteristic 2.


In characteristic different from 2, the quadratic form is uniquely
determined by its associated bilinear form; in fact, from (1) and (2)
it follows immediately that $q(u) = \frac{1}{2}\inner{u}{u}$. Thus, in
characteristic different from 2, preserving the quadratic form is
equivalent to preserving the bilinear form. By taking coordinates with
respect to a given (finite) basis, we can represent the bilinear form
by a matrix $Q$ as usual, and thereby obtain the usual isomorphism
$\O(V) \simeq \O_n(F) = \{ A \in \GL_n(F) \colon A^\transpose Q A = Q
\}$. 

In characteristic 2 this equivalence falls apart.  In fact, in this
case it follows from (1) and (2) that we have $\inner{u}{u} = 0$ for
all $u \in V$. This suggests that it is not always possible to recover
$q$ from its bilinear form in characteristic 2.

\begin{example*}
For an instructive example, take $V = F^n$ to be the space of
$n$-dimensional vectors over $F$ where $F$ has characteristic 2, and
take the quadratic form given by $q(x_1, \dots, x_n) = \sum
x_j^2$. Let us compute the values of the associated bilinear form on
pairs of standard basis elements. We denote the standard basis of
$F^n$ by $\{ e_1, \dots, e_n\}$. From the preceding paragraph we have
$\inner{e_i}{e_i} = 0$ for all $i = 1, \dots, n$. Furthermore, for any
$i \ne j$ we have from (2) that $\inner{e_i}{e_j} = 0$. So the
associated bilinear form is the zero form in this case. Clearly it is
not possible to recover the quadratic form from the bilinear form, and
preserving one has quite a different meaning than preserving the
other. In fact the group of linear automorphisms preserving the
bilinear form is just $\GL(V)$.  The orthogonal group $\O(V)$ is quite
different from $\GL(V)$ except in the case $n=2$, $F=\F_2$. We claim
that in fact 
\[
  \O(V) = \{ A \in \GL_n(F) \colon q(A_j) = 1, \text{ all } j =1,
  \dots, n \},
\]
where $A_j$ is the $j$th column of $A$. This is proved by a simple
calculation.
\end{example*}

Most authors impose the additional assumption of nondegeneracy of the
quadratic form. Again, it is possible to do this uniformly in all
characteristics, provided that one adopts the following as the
definition of nondegeneracy.

\begin{defn*}
  A quadratic form $q$ on $V$ is \emph{nondegenerate} provided that $u
  \in V_0$ and $q(u)=0$ implies $u=0$. Here $V_0$ is the radical of
  the polar bilinear form $\inner{\ }{\ }$ associated to $q$.
\end{defn*}

Again, this version of the definition typically appears in the
literature only in characteristic 2. In characteristic different from
2, it is easy to check that $q$ is nondegenerate in the above sense if
and only if $V_0=0$, so our definition is equivalent to the usual one
in all characteristics different from 2. 

Assume from now on that the given quadratic form $q$ is
nondegenerate. In characteristic different from 2, $\O(V)$ contains a
unique subgroup of index 2 which is denoted by $\SO(V)$; this is equal
to $\SL(V) \cap \O(V)$. However, this definition of $\SO(V)$ will not
serve in characteristic 2, since in that case $\O(V) \subseteq
\SL(V)$; i.e., all elements of $\O(V)$ have determinant 1. So in
characteristic 2 an invariant finer than the determinant is needed to
distinguish the special orthogonal group; usually the Dickson
invariant is used and $\SO(V)$ is defined as the set of elements of
Dickson psuedodeterminant zero. However, according to \cite{Dye}, the
subgroup $\SO(V)$ may also be defined in all characteristics as the
subgroup of all elements $\alpha$ of $\O(V)$ satisfying the condition:
$\text{rank }(\alpha - id)$ is even. But Dye considers only the
even dimensional case. (In case $\dim V$ is odd, we have $\SO(V) =
O(V)$ and it is unclear whether Dye's criterion produces this.)



We note that, since $\inner{u}{u}=0$ for all $u \in V$ the bilinear
form $\inner{\ }{\ }$ is alternating as well as symmetric in
characteristic 2, so by \cite[Theorem 2.10]{Grove} the space $V$ has
an orthogonal decomposition of the form
\[
  V = H_1 \oplus \cdots \oplus H_l \oplus V_0
\]
where each $H_j$ ($j = 1, \dots, l$) is a hyperbolic plane and $V_0$
is the radical of $\inner{\ }{\ }$.  Set $d = \dim V_0 = \dim V - 2l$;
this is the \emph{defect} of $q$. If the defect is zero then the
bilinear form is nondegenerate in the usual sense, although it is
customary for authors to use the word ``nondefective'' in this case
instead.  Now, when restricted to the radical $V_0$, we have 
\[
  q(au + bv) = a^2 q(u) + b^2 q(v) .
\]
We note that the orthogonal decomposition of $V$ implies that $V$ has
a basis with respect to which the matrix $Q$ of $\inner{\ }{\ }$ is a
block matrix of the form
\[
  Q = 
  \begin{bmatrix}
    M \\
    & M \\
    & & \ddots \\
    & & & M \\
    & & & & 0
  \end{bmatrix}
\quad \text{ where } M = 
\begin{bmatrix}
  0&1\\
  1&0
\end{bmatrix}
\]
and there are $l$ copies of the $2 \times 2$ matrix $M$ along the
diagonal of $Q$ and a $d \times d$ zero matrix in the lower right hand
corner of $Q$. (So if the defect $d = 0$ then the entry in the lower
right hand corner of the matrix $Q$ displayed above should be
omitted.)

In characteristic 2, an easy exercise \cite[p.~114]{Grove} shows that
if $q$ is nondegenerate and the field $F$ is perfect then $\dim V_0 =
0$ or $1$, depending on whether the dimension of $V$ is even or odd.





\begin{thebibliography}{99}\frenchspacing\raggedright\scriptsize
\bibitem[Artin 1957]{Artin} E. Artin, \emph{Geometric Algebra},
  Interscience Publishers, Inc. 1957.

\bibitem[Carter 1972]{Carter} R. Carter, \emph{Simple Groups of Lie
    Type}, Pure and Applied Mathematics, Vol. 28, John Wiley \& Sons
  1972.

\bibitem[Dye 1977]{Dye} R.H. Dye, A geometric characterization of the
  special orthogonal group and the Dickson invariant, \emph{J. London
    Math. Soc.} (2) 15 (1977), 472--476.

\bibitem[Grove 2002]{Grove} L. Grove, \emph{Classical Groups and
    Geometric Algebra}, Graduate Studies in Mathematics, 39, American
  Mathematical Society, Providence, RI 2002.
\end{thebibliography}

\end{document}
