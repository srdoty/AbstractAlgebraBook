\documentclass[11pt]{article}
\usepackage[nohead,margin=1.50in]{geometry} %set margins
\usepackage{amsmath,amssymb,amsthm,pdiag} %AMS packages for math stuff
\usepackage{multicol} % for use in HW section
\usepackage{enumitem}
  \setlist{topsep=1pt,itemsep=0pt,parsep=1pt}
  \setenumerate[1]{label=(\alph*)}
\usepackage[all]{xy}


\newenvironment{problems}
{
 \begin{enumerate}[topsep=1pt,itemsep=0pt,parsep=2pt,leftmargin=0.6cm,%
 label={\arabic*.}, ref=\arabic*] \small
}
{
 \end{enumerate}
}

%%% Define some theorem and example environments. The starred versions
%%% are un-numbered and the unstarred versions are numbered.
\newtheoremstyle{plain}
  {\topsep}   % ABOVESPACE
  {\topsep}   % BELOWSPACE
  {\slshape}  % BODYFONT
  {0pt}       % INDENT (empty value is the same as 0pt)
  {\bfseries} % HEADFONT
  {.}         % HEADPUNCT
  {5pt plus 1pt minus 1pt} % HEADSPACE
  {}          % CUSTOM-HEAD-SPEC

\swapnumbers
\newtheorem{thm}{Theorem}[section]
\newtheorem{lem}[thm]{Lemma}
\newtheorem{prop}[thm]{Proposition}
\newtheorem{cor}[thm]{Corollary}
\newtheorem*{thm*}{Theorem}
\newtheorem*{lem*}{Lemma}
\newtheorem*{prop*}{Proposition}
\newtheorem*{cor*}{Corollary}

\theoremstyle{definition}
\newtheorem{defn}[thm]{Definition}
\newtheorem{example}[thm]{Example}
\newtheorem{examples}[thm]{Examples}
\newtheorem{rmk}[thm]{Remark}
\newtheorem{rmks}[thm]{Remarks}
\newtheorem{conv}[thm]{Convention}
\newtheorem*{defn*}{Definition}
\newtheorem*{example*}{Example}
\newtheorem*{examples*}{Examples}
\newtheorem*{rmk*}{Remark}
\newtheorem{rmks*}{Remarks}
\newtheorem*{conv*}{Convention}


%%% Define some convenient abbreviations for common mathematical
%%% notations.
\newcommand{\R}{\mathbb{R}} % use \R for the real numbers
\newcommand{\C}{\mathbb{C}} % use \C for the complex numbers
\newcommand{\Z}{\mathbb{Z}} % use \Z for the integers
\newcommand{\Q}{\mathbb{Q}} % use \Q for the rationals
\newcommand{\N}{\mathbb{N}} % use \N for the natural numbers
\newcommand{\F}{{\mathbb F}}
\newcommand{\compose}{\circ} % functional composition
\newcommand{\gen}[1]{\langle #1 \rangle}
\newcommand{\End}{\operatorname{End}}
\newcommand{\GL}{\mathrm{GL}}
\newcommand{\SL}{\mathrm{SL}}
\newcommand{\PSL}{\mathrm{PSL}}
\renewcommand{\O}{\mathrm{O}}
\newcommand{\SO}{\mathrm{SO}}
\newcommand{\U}{\mathrm{U}}
\newcommand{\SU}{\mathrm{SU}}
\newcommand{\g}{\mathfrak{g}}
\newcommand{\transpose}{\mathsf{T}}
\newcommand{\B}{\mathcal{B}}
\newcommand{\Rep}{\operatorname{Rep}}
\newcommand{\Mat}{\operatorname{Mat}}
\newcommand{\inner}[2]{\langle #1, #2 \rangle}
\newcommand{\sgn}{\operatorname{sgn}}
\newcommand{\n}{\underline{\mathbf{n}}}
\newcommand{\Sym}{\mathbb{S}}
\newcommand{\Alt}{\mathbb{A}}
\newcommand{\D}{\mathbb{D}}
\newenvironment{perm}[2]{\left(\begin{smallmatrix}#1 \\ #2}{\end{smallmatrix}\right)}
\newcommand{\lcm}{\operatorname{lcm}}
\newcommand{\res}{\operatorname{res}}
\newcommand{\normal}{\triangleleft\,}%better than \lhd
\newcommand{\morenormal}{\triangleright}
\newcommand{\im}{\operatorname{im}}

\allowdisplaybreaks
\parskip=2pt

%\title{Document Title}
%\author{author's name}

\begin{document}%\maketitle
\setcounter{section}{20}



\section{Quotient groups}\noindent
From now on we use multiplicative notation unless indicated
otherwise. The results apply equally well to additive groups, but the
notation needs to be translated accordingly.


If $H$ is a subgroup of a group $G$, recall that $G/H =\{aH : a \in G
\}$ is the set of all left cosets of $H$ in $G$. The key question for
this section is: when is the set $G/H$ of left cosets a group?  Recall
that we have previously defined the product of two subsets $S,T$ in a
group by $ST = \{xy \mid x \in S, y \in T \}$. In particular, this
means that $aH = \{ax \mid x \in H \}$, $Ha = \{ xa \mid x \in H \}$,
and $(aH)(bH) = \{ axby \mid x,y \in H\}$.  We want the product
$(aH)(bH)$ of two left cosets to always equal another left coset.

\begin{thm}\label{thm:coset-mult}[coset multiplication]
  Suppose that $H$ is a subgroup of a group $G$. Then the following
  are equivalent: 
  \begin{enumerate}
  \item For any $a,b \in G$, there is some $c \in G$ such that
    $(aH)(bH) = cH$.
  \item For any $a,b \in G$, $(aH)(bH)=(ab)H$. 
  \item $Hb = bH$, for all $b \in G$.
  \end{enumerate}
\end{thm}\index{coset~multiplication}

\begin{proof}
(a) $\implies$ (b): Since $ab = a1b1 \in (aH)(bH)$, it follows from
  the given equality $(aH)(bH) = cH$ that $ab \in cH$. Hence $(ab)H =
  cH$.

(b) $\implies$ (c): Let $b \in G$. Then $(bH)(b^{-1}H) = (bb^{-1})H =
  1H = H$; i.e., $bHb^{-1}H = H$. This implies that $b H b^{-1} =
  bHb^{-1} 1 \subset H$, so $bH \subset Hb$. Similarly, $(b^{-1}H)(bH)
  = (b^{-1}b)H = H$ implies that $b^{-1}Hb \subset H$, so $Hb \subset
  bH$. We have shown that $bH \subset Hb$ and $Hb \subset bH$, so $Hb
  = bH$.

(c) $\implies$ (a): Suppose that $Hb = bH$. Then by associativity we
  have $(aH)(bH) = a(Hb)H = a(bH)H = (ab)(HH) = (ab)H$.
\end{proof}

So multiplication of left cosets always produces another left coset
precisely when condition (c) of the previous theorem holds. This leads
to the next definition.

\begin{defn}\index{normal~subgroup}
Let $G$ be a group and $H$ a subgroup of $G$. We say that $H$ is a
{\em normal subgroup} of $G$ if $Ha = aH$ for every $a\in G$. We will
write $H \normal G$ (or $G \morenormal H$) to indicate that $H$ is a
normal subgroup of $G$.
\end{defn}

The trivial subgroup $\{1\}$ and the entire group $G$ are always
normal subgroups of a group $G$.  The definition says that $H$ is a
normal subgroup if and only if every left coset is also a right coset.
Thus, \emph{every} subgroup of an abelian group is normal.

%The next result provides several equivalent ways to verify that a
%given subgroup is normal.

\begin{thm}\label{thm:normality-conds} 
Suppose that $H$ is a subgroup of a group $G$. The following are
equivalent:
\begin{enumerate}
\item $H \normal G$.
\item $aHa^{-1} = H$, for all $a\in G$.
\item $aHa^{-1} \subset H$, for all $a\in G$.
\end{enumerate}
\end{thm}

\begin{proof}
(a) $\implies$ (b): If $H$ is normal in $G$ then by definition $aH=Ha$
for all $a\in G$. Multiplying this equality by $a^{-1}$ on the right 
gives the equality $aHa^{-1} = Haa^{-1}$.  But $Haa^{-1}=H 1 = H$.

(b) $\implies$ (c): This is clear.

(c) $\implies$ (a): Suppose $aHa^{-1} \subset H$, for all $a\in G$.
Right multiply by $a$ to get $aH \subset Ha$. Replacing $a$ by its
inverse in the inclusion $aHa^{-1} \subset H$, we get $a^{-1}Ha
\subset H$, so by left multiplication by $a$ we get $Ha \subset aH$.
We have shown that both $aH \subset Ha$ and $Ha \subset aH$, so $aH =
Ha$. This proves (a). 
\end{proof}

\begin{rmks}
  1. Note that if $H$ is a subgroup of a group $G$ then $aHa^{-1}$ is
  also a subgroup of $G$, for any $a \in G$. This is called a
  \emph{conjugate} subgroup of $H$. Note also that $|H| = |aHa^{-1}|$,
  since the map $x \mapsto axa^{-1}$ defines a bijection from $H$ onto
  $aHa^{-1}$. In general, the conjugate subgroup $aHa^{-1}$ can be
  different from $H$. Then $H \normal G$ if and only if all the
  conjugate subgroups of $H$ are equal to $H$. (We can say that $H$ is
  \emph{stable under conjugation} when $H \normal G$.)

  2. The relation $\normal$ is not transitive. That is, if $H \normal
  K$ and $K \normal G$ then it is not always true that $H \normal G$.

  3. By replacing $a$ by $a^{-1}$ in parts (b), (c) of the preceding
  theorem, we see that $H \normal G$ is also equivalent to:
  \begin{enumerate}
  \item [(b')]  $a^{-1}Ha = H$, for all $a\in G$.
  \item [(c')]  $a^{-1}Ha \subset H$, for all $a\in G$.
  \end{enumerate}
  Condition (c) in the theorem is a closure condition. It says that a
  subgroup $H$ is normal if and only if it is closed under
  conjugation. Elements of the form $axa^{-1}$ are called
  \emph{conjugates} of $x$.
\end{rmks}

The following simple observation can often be used to find a normal
subgroup of a finite group whose order is an even number. It can also
be applied more generally to infinite groups.

\begin{prop}\index{index~2~implies~normal}
  If $H$ is a subgroup of a group $G$ of index $2$ then $H$ must be a
  normal subgroup of $G$. (In other words, $[G:H]=2$ implies $H
  \normal G$).
\end{prop}

\begin{proof}
Let $a\in G$. Either $a \in H$ or $a \notin H$, and we consider the
two cases separately. Case 1: If $a\in H$ then $aH=H$ and $Ha=H$ by
the fact that $H$ is closed under products, so $aH=Ha$. Case 2: If
$a\notin H$ then $aH = G - H = \{ g \in G \mid g \notin H \}$ since
there are just two left cosets, and they are disjoint.  Similarly, $Ha
= G - H$ for exactly the same reason. Thus $aH=Ha$. Cases 1 and 2
taken together show that $aH=Ha$ for all $a \in G$, so $H \normal G$.
\end{proof}
 

\begin{examples}
1. Since $[\Sym_n : \Alt_n] = 2$, it follows that the alternating
group $\Alt_n$ is a normal subgroup of the symmetric group $\Sym_n$,
for any $n$.

2. Since $[\O(2) : \SO(2)] = 2$, it follows that $\SO(2) \normal
\O(2)$.
\end{examples}

The following is a fundamental result in group theory.  It turns out
that when $N \normal G$, coset multiplication makes $G/N$ into a
group.

\begin{thm}\index{quotient~group}[quotient group]
  Let $N$ be a normal subgroup of a group $G$. Then the coset
  multiplication rule $(aN)(bN) = (ab)N$ (for any $a,b \in G$) is a
  well-defined binary operation on the set $G/N$ of left cosets. With
  respect to this multiplication, $G/N$ is a group, with identity
  element $1N = N$. The inverse of $aN$ is $a^{-1}N$.
\end{thm}

\begin{proof}
To show that coset multiplication is well-defined we must show that if
$aN=cN$ and $bN=dN$ then $(ab)N = (cd)N$. Clearly $(aN)(bN) =
(cN)(dN)$, and by Theorem \ref{thm:coset-mult}, it follows that $(ab)N
= (cd)N$. So coset multiplication is well-defined.

Since $(aN bN)cN = (ab)N cN = (ab)c N = a(bc) N = aN (bc)N =
aN(bNcN)$, the associative law (G1) holds. Since $(1N)(aN) = aN =
(aN)(1N)$ it follows that (G2) holds with $N=1N$ serving as the
identity. Finally, $(aN)(a^{-1}N) = 1N = (a^{-1}N)(aN)$ shows that
(G3) holds and $(aN)^{-1} = a^{-1}N$.
\end{proof}

\begin{defn}
  When $N \normal G$, the group $G/N$ is called the {\em quotient
    group} of $G$ by $N$. Quotient groups are also called \emph{factor
    groups}.
\end{defn}


\begin{rmk}
If $G$ is a finite group then by Lagrange's theorem $|G/N| = |G|/|N|$
(the order of the quotient group $G/N$ equals the cardinality of $G$
divided by the cardinality of $N$). Of course, by the definition of
the index $[G\colon N]$ we also have $|G/N| = [G \colon N]$.
\end{rmk}


\section*{Exercises}
\begin{problems}

\item Show that the trivial subgroup is always a normal subgroup of
  any group $G$. Also, show that $G \normal G$. 

\item Show that \emph{every} subgroup of a group $G$ is normal if:
  \begin{enumerate}
  \item $G$ is abelian.
  \item $G$ is cyclic.
  \end{enumerate}

\item Prove that the center\index{center} $Z(G)$ of a group $G$ is a
  normal subgroup of $G$.

\item Show that if $H$ is a subgroup of order $n$ in a group $G$ and
  $H$ is the only subgroup of order $n$, then $H \normal G$.

\item Show that a subgroup $H$ of a group $G$ is normal if and only if
  it satisfies the condition: $ab \in H \iff ba \in H$, for all $a,b
  \in G$.

\item Prove that the intersection of any number of normal subgroups of
  a group $G$ is a normal subgroup of $G$.

\item Let $\D_4$ be the symmetry group of the square, and let $r$ be
  the basic rotation. Is the subgroup $H = \{ 1, r^2 \}$ a normal
  subgroup of $\mathrm{D}_4$?  Prove your answer.

\item Let $\D_n$ be the dihedral group\index{dihedral~group} on $n$
  vertices and let $R$ be its rotation subgroup. Prove that
  $R \normal \mathrm{D}_n$.

\item Let $\D_n$ be the dihedral group on $n$ vertices and let $d \in
  \D_n$ be a reflection. Then $H = \gen{d} = \{1, d\}$ is a cyclic
  subgroup of $\D_n$. Find necessary and sufficient conditions for
  this subgroup to be normal in $\D_n$, and prove your answer.

\item Prove that $\Alt_n$\index{alternating~group} is a subgroup of
  $\Sym_n$ of index 2 by first proving that there is a bijection from
  $\Alt_n$ onto $\alpha \Alt_n$, where $\alpha$ is any odd
  permutation.

\item Find all normal subgroups of the quaternion
  group\index{quaternion~group} $Q$. Justify your answer.

\item Show that if $H$ is a subgroup of index 2 in a group $G$ then
  $G/H$ is isomorphic to the additive group $(\Z_2,+)$.

\item Show that $\SO(n) \normal \O(n)$ and identify a group that is
  isomorphic to $\O(n)/\SO(n)$.

\item Show that $\SL(n) \normal \GL(n)$ and $\GL(n)/\SL(n)$ is
  isomorphic to the multiplicative group $\R^\times$ of the field of
  real numbers.

\item Show that the set $G$ of all real $2 \times 2$ matrices of the
  form $\left(
  \begin{matrix}
    a&b\\0&c
  \end{matrix}
  \right)$ is a subgroup of $\GL(2)$. Let $N$ be the set of all
  matrices of the form  $\left(
  \begin{matrix}
    1&b\\0&1
  \end{matrix}
  \right)$. Prove that $N \normal G$. (Note that you need to show it
  is a subgroup as well as prove that it is normal.)

\item Show that $n\Z = \{nk \mid k \in \Z\}$ is a normal subgroup of
  the additive group $\Z$ of integers, and that $\Z/n\Z \cong \Z_n$.

\item Prove that if $H,K$ are subgroups of $G$ and one of them is
  normal in $G$, then their product $HK$ is a normal subgroup of $G$.

\item Show that if $H, K$ are subgroups of a group $G$ such that $K
  \normal G$ then $H \cap K \normal H$. 

\item\index{normality~is~not~transitive} Give an example to show that
  there are groups $K < H < G$ such that $K \normal H$ and
  $H \normal G$ but $K$ is not a normal subgroup of $G$. In other
  words, normality is not transitive.

\item Is it possible to find a group $G$ in which every subgroup is
  normal but $G$ is not abelian? If so, exhibit one, otherwise prove
  that such a group is abelian.
\end{problems}








\newpage
\section{Homomorphisms}\noindent
This is a central concept in group theory that ties everything
together. To define it, we momentarily revert to generic notation.

\begin{defn}\index{homomorphism}
  Suppose that $(G,*)$ and $(H,\#)$ are groups. A \emph{homomorphism}
  of groups is a function $f: G \to H$ such that $f(a*b) = f(a)\#
  f(b)$ for all $a,b \in G$.
\end{defn}

In particular, group isomorphisms \index{isomorphism} are bijections
with this property, so every isomorphism is a homomorphism. But not
conversely, because we do not require that a homomorphism be a
bijection.

\begin{rmk}
If $G$ and $H$ are additive groups then $f: G \to H$ is a homomorphism
if and only if $f(a+b) = f(a)+f(b)$, for all $a,b \in G$. If $G$ is
additive and $H$ is multiplicative then $f: G \to H$ is a homomorphism
if and only if $f(a+b) = f(a) \cdot f(b)$, for all $a,b \in G$. If $G$
is multiplicative and $H$ is additive then $f: G \to H$ is a
homomorphism if and only if $f(a\cdot b) = f(a) + f(b)$, for all $a,b
\in G$. Finally, if both $G$ and $H$ are multiplicative then $f: G \to
H$ is a homomorphism if and only if $f(a \cdot b) = f(a) \cdot f(b)$,
for all $a,b \in G$.
\end{rmk}

The first two parts of the next result states that a group
homomorphism must match up the identity and inverses in the two
groups.

\begin{thm}[properties of homomorphisms]\label{homprop}
  Let $f:G \to H$ be a homomorphism from a group $(G,*)$ to a group
  $(H,\#)$. Then:
\begin{enumerate}
\item $f(e_G) = e_H$; i.e., $f$ maps the identity of $G$ onto
  the identity of $H$.

\item The inverse of $f(a)$ in $H$ is the image of the inverse of $a$;
  i.e., if $a'$ is the inverse of $a$ in $G$ then the inverse of
  $f(a)$ in $H$ is $f(a)' = f(a')$.

\item[(c)] If $f$ is a bijection, then $f^{-1}: H \to G$ must also be a
  homomorphism.
\end{enumerate}
\end{thm}

\begin{proof}
(a) Let $y = f(e_G)$ be the image of the identity of $G$. We must show
that $y=e_H$, the identity of $H$. We have $e_H\#f(x) = f(x) =
f(e_G*x) = f(e_G)\#f(x) = y\#f(x)$. Thus $e_H\# f(x) = y \#f(x)$ and
by right cancellation we conclude that $e_H = y$.

(b) Let $y = f(b)$ where $b$ is the inverse of $a$ in $G$. Let $z$ be
the inverse of $f(a)$ in $H$. We must show that $y=z$. But $f(a)\#z =
e_H$ by definition of inverse, and $f(a)\#y = f(a)\#f(b) = f(a*b) =
f(e_G) = e_H$, by part (a) and the defining property of
homomorphism. Thus $f(a)\#z = f(a)\#y$. By left cancellation, this
implies $z=y$.

(c) If $f$ matches up products and is a bijective correspondence, then
its inverse must match up products as well.
\end{proof}



\begin{rmk}
By part (a) of the preceding result, a homomorphism between additive
groups must take $0$ to $0$, while a homomorphism between
multiplicative groups must take $1$ to $1$. A homomorphism from an
additive group to a multiplicative group must send $0$ to $1$, while a
homomorphism from an multiplicative group to an additive group must
send $1$ to $0$,

Similarly for inverses in part (b). If $f$ is a homomorphism between
additive groups then $-f(a) = f(-a)$, whereas if $f$ is a homomorphism
between multiplicative groups then $f(a)^{-1} = f(a^{-1})$. If $f$ is
a homomorphism from an additive group to a multiplicative group then
$f(a)^{-1} = f(-a)$. Finally, if $f$ is a homomorphism from an
multiplicative group to an additive group then $-f(a) = f(a^{-1})$.
\end{rmk}


It quickly becomes tiresome to distinguish between these
possibilities. From now on in the general theory, we will always use
multiplicative notation for both groups connected by a homomorphism,
unless stated otherwise, trusting the reader to make the necessary
adjustments in other situations.


\begin{defn}
  Let $f: G \to G'$ be a group homomorphism from a group $G$ into a
  group $G'$. The \emph{kernel} of $f$ (written as $\ker f$) is the
  subset of the domain group $G$ defined by $\ker f = \{x \in G \mid
  f(x) = 1 \}$. 
\end{defn}

Note that if $G'$ is an additive group, then $\ker f = \{ x \in G \mid
f(x) = 0 \}$.

\begin{thm}[kernels are normal subgroups]\index{kernel}\label{KAN} 
The kernel of a homomorphism $f: G \to G'$ of groups is a normal
subgroup of the domain group $G$.
\end{thm}

The proof is an easy exercise.

\begin{defn}\index{canonical~homomorphism}\label{canhom}
Let $G$ be a group and $K$ a normal subgroup of $G$. Then we have a
homomorphism $\pi: G \to G/K$ defined by the rule $\pi(a) = aK$. It is
surjective. This particular homomorphism is called the {\em canonical
  homomorphism} or the \emph{canonical quotient map}.
\end{defn}

One can check (exercise) that the canonical homomorphism actually is a
homomorphism.

\begin{thm}[first isomorphism theorem] \label{thm:first-iso}
\index{first~isomorphism~theorem}\index{isomorphism~theorem!first}
  Let $G$, $G'$ be groups and $f:G \to G'$ a group homomorphism. Set
  $K=\ker f$ and $I=\im f = f(G)$.  Then
  $f = \iota \compose \overline{f} \compose \pi$, where
  $\iota: I \to G'$ is the inclusion map (coming from the inclusion
  $I \subset G'$), $\pi: G \to G/K$ is the canonical homomorphism, and
  $\overline{f}: G/K \to I$ is an isomorphism, given by the rule
  $\overline{f}(aK) = f(a)$ for $a\in G$.  (In particular,
  $G/K \cong I$.)
\end{thm}

This theorem is also known as the {\em fundamental theorem of
  homomorphisms}\index{fundamental~theorem!of~homomorphisms}. The
factorization of the homomorphism $f$ in the theorem may be pictured
by the following commutative diagram:
\[
\xymatrix{ 
G \ar[rr]^f \ar[d]_\pi &&  G'  \\
G/K \ar[rr]^{\overline{f}} && I \ar[u]_\iota 
}
\] 
and the diagram is a very convenient way to remember the theorem.  In
the diagram, there are two routes from $G$ to $G'$. The theorem says
these two routes are the same: $f = \iota \compose \overline{f}
\compose \pi$, and that the induced map $\overline{f}$ is an
isomorphism. (The latter fact is most often used in practice.)

\begin{proof}
The map $\iota: I \to G'$ is defined by the rule $\iota(y)=y$ for any
$y\in I$. The map $\pi$ has previously been defined, and the
definition of $\overline{f}$ is given in the statement of the theorem.
We must show that  $f = \iota \compose \overline{f} \compose \pi$; in
other words, we must show that 
$$ f(x) = (\iota \compose \overline{f} \compose \pi)(x) =
  \iota(\overline{f}(\pi(x)))
$$
for all $x$ in $G$. Well, let $x\in G$. Then by the definitions of the
maps we have $\iota(\overline{f}(\pi(x))) = \iota(\overline{f}(xK)) =
\iota(f(x)) = f(x)$. This proves the desired equality. 

Let us check that the map $\overline{f}$ is a {\em well-defined}
function; i.e., the value $\overline{f}(aK)$ does not depend on the
choice of coset representative. Suppose that $aK=bK$ for two elements
$a,b$ of $G$.  We have to show that $\overline{f}(aK) =
\overline{f}(bK)$. Well, by definition of the map $\overline{f}$ we
have $\overline{f}(aK) = f(a)$ and $\overline{f}(bK) = f(b)$, so we
have to prove that $f(a)=f(b)$. This follows from the equality $aK =
bK$, which implies that $a^{-1}b \in K$, so there exists some $x \in
K$ such that $a^{-1}b = x$, so $b=ax$, so $f(b) = f(ax) = f(a)f(x) =
f(a) 1 = f(a)$ as desired. This proves that the map $\overline{f}$ is
well-defined.

It remains to prove that the induced map $\overline{f}: G/K \to I$ is
an isomorphism. First, note that $\overline{f}$ is a homomorphism,
since $\overline{f}(aK\cdot bK) = \overline{f}(abK) = f(ab) = f(a)f(b)
= \overline{f}(aK) \overline{f}(bK)$, for any $a,b \in G$. It is
injective since if $aK$ lies in the kernel of $\overline{f}$, then
$\overline{f}(aK) = f(a)$ is the identity element of $G'$, so $a \in
K$, so $aK=K$. This proves that the kernel of $\overline{f}$ is just
the identity coset $K$ of $G/K$, so $\overline{f}$ is injective as
claimed.  Finally, $\overline{f}$ is surjective onto $I$ by its
definition.  We have shown that $\overline{f}$ is a bijective
homomorphism. Thus it is an isomorphism. The proof is complete.
\end{proof}


\begin{examples}\label{examples:iso-1}
1. For $n \ge 2$, there is a homomorphism from the symmetric group
$\Sym_n$ to the multiplicative group $\{1,-1\} = \Z^\times$ given by
the rule: $\alpha \mapsto \sgn(\alpha)$. This homomorphism is
surjective (i.e., the image is $\{1,-1\}$). Its kernel is the set
$\Alt_n$ of even permutations, since we know that $\sgn(\alpha) = 1$
if and only if $\alpha$ is even. By the theorem, it follows that
$\Sym_n/\Alt_n \cong \Z^\times$ is cyclic of order 2. 

2. Let $n$ be a positive integer. Recall that we write $\Z_n$ for the
additive group of integer residues modulo $n$. Consider the
homomorphism (of additive groups) $\Z \to \Z_n$ given by $a \to
\overline{a}$, where $\overline{a}$ is the residue of $a$ mod
$n$. This is a homomorphism since $\overline{a+c} = \overline{a} +
\overline{c}$ for all $a,c \in \Z$. The kernel of this homomorphism is
the subgroup $n\Z$ of all multiples of $n$.  It is clear that the
homomorphism is surjective onto $\Z_n$. Thus we have a group
isomorphism $\Z/(n\Z) \cong \Z_n$. This shows that the additive group
$\Z_n$ is, in fact, a quotient group. (Note: To be sure, $\Z$ and
$\Z_n$ are actually rings when we consider the two operations of
addition and multiplication together, but in the isomorphism above we
are for the moment ignoring the multiplication and just paying
attention to the additive group structure.)
\end{examples}



The next result states that every normal subgroup arises as the kernel
of some homomorphism.


\begin{thm}[normal subgroups are kernels] \label{thm:NAK} 
If $K$ is a given normal subgroup of a group $G$, then there exists a
homomorphism $f: G \to G'$ (for some group $G'$) such that $K= \ker
f$.
\end{thm}

\begin{proof}
We set $G' = G/K$ and we let $f$ be the canonical homomorphism. Then
its kernel is $K$, so the proof is complete. 
\end{proof}

The following two results are known as the second and third
isomorphism theorems for groups. Along with the first isomorphism
theorem (Theorem \ref{thm:first-iso}) these results describe 
fundamental properties of quotient groups and homomorphisms.


\begin{thm}[second isomorphism theorem]\label{thm:iso2}
\index{second~isomorphism~theorem}\index{isomorphism~theorem!second}%
  If $K \normal G$ and $H < G$ then:
\begin{enumerate}
  \item $H \cap K \normal H$.
  \item The product set $HK = \{xy \colon x \in H, y \in K\}$ is a
    subgroup of $G$.
  \item $H/(H \cap K) \cong HK/K$.
\end{enumerate}
\end{thm}

This follows from the first isomorphism theorem. The proof is an
exercise.

\begin{thm}[third isomorphism theorem]\label{thm:iso3}% 
\index{third~isomorphism~theorem}\index{isomorphism~theorem!third}%
If $H \normal G$, $K \normal G$, and $H \subset K$ then:
\begin{enumerate}
  \item  $(K/H) \normal (G/H)$.
  \item $(G/H)/(K/H) \cong G/K$.
\end{enumerate}
\end{thm}

This also follows from the first isomorphism theorem. The proof is an
exercise.


We finish our discussion of quotients with an easy application of the
first isomorphism theorem, to classify all cyclic groups.

\begin{thm}[classification of cyclic groups]
\index{classification!of~cyclic~groups}
Let $G$ be a cyclic group. Then either $G$ is isomorphic to the
additive group $\Z$ of integers or $G$ is isomorphic to the additive
group $\Z_n$ of residues modulo $n$, for some $n$.
\end{thm}

\begin{proof}
Let $G = \gen{a}$.  The integers $\Z$ form a group under addition, and
the (surjective) map $f:\Z \to G$ defined by the rule $k \to a^k$ is a
group homomorphism. 

If the order of $a$ is infinite, then the kernel of $f$ is the trivial
subgroup $\{0\}$, and so $f$ is injective and hence an isomorphism. So
$\Z \cong G$ in case $|a|$ is infinite.

If the order of $a$ is finite, say the order is $n$, then the kernel
of $f$ is the set $n\Z$ of all multiples of $n$, and by the first
isomorphism theorem, $\Z/n\Z \cong G$. But $\Z/n\Z$ is $\Z_n$, by
definition. So in this case we have $\Z_n \cong G$. 
\end{proof}




\section*{Exercises}
\begin{problems}

\item Prove that kernels are normal subgroups (see \ref{KAN}).

\item Prove that an isomorphism is a bijective homomorphism. 

\item\index{injectivity~test} (Injectivity test for homomorphisms)
  Prove that if $f: G \to H$ is a group homomorphism with kernel $K$
  then $f$ is injective if and only if $K$ is the trivial subgroup.

\item Prove that the canonical homomorphism (see \ref{canhom}) is
  really a homomorphism; i.e., show that $\pi(ab) = \pi(a)\pi(b)$ for
  all $a,b$ in $G$.

\item Prove that if $f: G \to H$ and $g: H \to K$ are group
  homomorphisms then their composite $g \circ f$ is a group
  homomorphism.

\item (Homomorphic images of subgroups are subgroups) Show that if $f:
  G \to G'$ is a group homomorphism and $H < G$ then $f(H) < G'$,
  where $f(H) = \{f(x) \mid x \in H\}$.

\item (Homomorphic preimages of subgroups are subgroups) Show that if
  $f: G \to G'$ is a group homomorphism and $H' < G'$ then $f^{-1}(H')
  < G$, where $f^{-1}(H') = \{x \in G \mid f(x) \in H'
  \}$. Furthermore, show that $\ker f < f^{-1}(H')$.


\item Define a map $f \colon \R \to \C^\times$ by $f(x) = e^{2\pi i
  x}$ for any $x \in \R$. (Here $\R$ is the additive group of real
  numbers.)
  \begin{enumerate}
  \item Prove that $f$ is a group homomorphism and compute its kernel.
  \item Use the first isomorphism theorem to show that the quotient
    group $\R/\Z$ is isomorphic to the circle group $\{e^{2\pi i x}
    \mid x \in \R  \}$.
  \end{enumerate}
  
\item Prove that $\R/\Z$ in the previous problem is isomorphic to the
  group $\SO(2)$ of rotations of the plane.

\item Let $G$ be an abelian group. Let $H = \{ x^2 \mid x \in G \}$ be
  the set of squares in $G$, and let $K = \{ x \in G \mid x^2 = 1 \}$
  be the set of elements of order one or two. Prove that:
  \begin{enumerate}
  \item The function $f: G \to G$ defined by $f(x) = x^2$ is a
    homomorphism. 
  \item Identify the kernel of $f$ and justify your answer.
  \item Show that $G/K \cong H$.
  \end{enumerate}

\item Prove that $\SL(n) \normal \GL(n)$ and $\GL(n)/\SL(n) \cong
  \R^\times$ by applying the first isomorphism theorem.

\item Prove that $\SO(n) \normal \O(n)$ and $\O(n)/\SO(n) \cong
  \Z^\times$ by applying the first isomorphism theorem.

\item Prove the {\em second isomorphism theorem}: If $K \normal G$ and
  $H < G$ then $H \cap K \normal H$, $HK$ is a subgroup of $G$, and
  $H/(H\cap K) \cong HK/K$. (See \ref{thm:iso2}.)

\item Prove the {\em third isomorphism theorem}: If $H \normal G$, $K
  \normal G$, and $H \subset K$ then $(K/H) \normal (G/H)$ and
  $(G/H)/(K/H) \cong G/K$.  (See \ref{thm:iso3}.)

\item If $G$ is a group, show that the function $f: G \to G$ defined
  by $f(x) = x^2$ is a homomorphism if and only if $G$ is abelian.


\item Use the third isomorphism theorem (see \ref{thm:iso3}) to prove
  that if $m = nk$ for positive integers $m,n,k$ then the additive
  group $\Z_m$ has a subgroup $\Z'_n$ isomorphic to $\Z_n$, and
  $\Z_m/\Z'_n \cong \Z_k$. [Hint: Start by setting $G = \Z$, $H =
    m\Z$, and $K=k\Z$.]

\item Show directly that any quotient group of a cyclic group is
  cyclic. Why does this give a different proof of the result in the
  preceding problem?

\end{problems}
\end{document}
