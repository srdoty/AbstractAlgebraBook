\documentclass[11pt,oneside]{article}
\usepackage[nohead,margin=1.50in]{geometry} %set margins
\usepackage{amsmath,amssymb,amsthm,pdiag} %AMS packages for math stuff
\usepackage{multicol} % for use in HW section
\usepackage{enumitem}
  \setlist{topsep=1pt,itemsep=0pt,parsep=1pt}
  \setenumerate[1]{label=(\alph*)}

\newenvironment{problems}
{
 \begin{enumerate}[topsep=1pt,itemsep=0pt,parsep=2pt,leftmargin=0.6cm,%
 label={\arabic*.}, ref=\arabic*] \small
}
{
 \end{enumerate}
}

%%% Define some theorem and example environments. The starred versions
%%% are un-numbered and the unstarred versions are numbered.
\newtheoremstyle{plain}
  {\topsep}   % ABOVESPACE
  {\topsep}   % BELOWSPACE
  {\slshape}  % BODYFONT
  {0pt}       % INDENT (empty value is the same as 0pt)
  {\bfseries} % HEADFONT
  {.}         % HEADPUNCT
  {5pt plus 1pt minus 1pt} % HEADSPACE
  {}          % CUSTOM-HEAD-SPEC

\swapnumbers
\newtheorem{thm}{Theorem}[section]
\newtheorem{lem}[thm]{Lemma}
\newtheorem{prop}[thm]{Proposition}
\newtheorem{cor}[thm]{Corollary}
\newtheorem*{thm*}{Theorem}
\newtheorem*{lem*}{Lemma}
\newtheorem*{prop*}{Proposition}
\newtheorem*{cor*}{Corollary}

\theoremstyle{definition}
\newtheorem{defn}[thm]{Definition}
\newtheorem{example}[thm]{Example}
\newtheorem{examples}[thm]{Examples}
\newtheorem{rmk}[thm]{Remark}
\newtheorem{rmks}[thm]{Remarks}
\newtheorem{conv}[thm]{Convention}
\newtheorem*{defn*}{Definition}
\newtheorem*{example*}{Example}
\newtheorem*{examples*}{Examples}
\newtheorem*{rmk*}{Remark}
\newtheorem{rmks*}{Remarks}
\newtheorem*{conv*}{Convention}


%%% Define some convenient abbreviations for common mathematical
%%% notations.
\newcommand{\R}{\mathbb{R}} % use \R for the real numbers
\newcommand{\C}{\mathbb{C}} % use \C for the complex numbers
\newcommand{\Z}{\mathbb{Z}} % use \Z for the integers
\newcommand{\Q}{\mathbb{Q}} % use \Q for the rationals
\newcommand{\N}{\mathbb{N}} % use \N for the natural numbers
\newcommand{\F}{{\mathbb F}}
\newcommand{\compose}{\circ} % functional composition
\newcommand{\gen}[1]{\langle #1 \rangle}
\newcommand{\End}{\operatorname{End}}
\newcommand{\GL}{\mathrm{GL}}
\newcommand{\SL}{\mathrm{SL}}
\renewcommand{\O}{\mathrm{O}}
\newcommand{\SO}{\mathrm{SO}}
\newcommand{\U}{\mathrm{U}}
\newcommand{\SU}{\mathrm{SU}}
\newcommand{\g}{\mathfrak{g}}
\newcommand{\transpose}{\mathsf{T}}
\newcommand{\B}{\mathcal{B}}
\newcommand{\Rep}{\operatorname{Rep}}
\newcommand{\Mat}{\operatorname{Mat}}
\newcommand{\inner}[2]{\langle #1, #2 \rangle}
\newcommand{\sgn}{\operatorname{sgn}}
\newcommand{\n}{\underline{\mathbf{n}}}
\newcommand{\Sym}{\mathbb{S}}
\newcommand{\Alt}{\mathbb{A}}
\newcommand{\D}{\mathbb{D}}
\newenvironment{perm}[2]{\left(\begin{smallmatrix}#1 \\ #2}{\end{smallmatrix}\right)}
\newcommand{\lcm}{\operatorname{lcm}}
\newcommand{\res}{\operatorname{res}}
\newcommand{\im}{\operatorname{im}}

\allowdisplaybreaks
\parskip=2pt

%\title{Document Title}
%\author{author's name}

\begin{document}%\maketitle
\setcounter{section}{19}

\section{Cosets}\noindent
We now introduce cosets, which will be used to prove Lagrange's
theorem and to construct quotient groups. Cosets are a fundamental
concept in group theory.  


\begin{defn}\index{coset}\label{def:cosets} 
If $H$ is a subgroup of a group $(G,*)$ and $a \in G$ then we write
$a*H = \{a*x : x \in H\}$ and $H*a = \{x*a : x \in H\}$.  These sets
are called \emph{left} and \emph{right cosets} of $H$ in $G$,
respectively.

If $(G, \cdot)$ is a multiplicative group then we write $aH$ and $Ha$
for the left and right cosets of $H$, whereas if $(G, +)$ is a
additive group then we write them as $a+H$ and $H+a$ instead.
\end{defn}


\begin{examples}
1. Let $H = \{ 1, r, \cdots, r^{n-1} \}$ be the rotation subgroup of
the dihedral group $\D_n$. Then $dH = \{d, dr, \dots, dr^{n-1} \}$,
where $d \in \D_n$ is any reflection, and $Hd = \{d, rd, \dots,
r^{n-1}d \}$. So $dH = Hd$. Furthermore, if $a \in H$ is any rotation,
then $aH = H$ and $Ha = H$.

2. Let $H = 2\Z = \{ 2k : k \in \Z\}$ be the subgroup of even integers
in the additive group $(\Z,+)$. Then $1+H = 1+2\Z = \{ 2k+1 : k \in \Z
\} = H+1$ is the set of all odd integers. Furthermore, $m + H = 1+H$
for any odd integer $m$, and $m+H = 0+H = H$ for any even integer $m$.

3. Let $H = n\Z = \{ nk : k \in \Z \}$ be the subgroup of multiples of
$n$ in the additive group $(\Z,+)$. Then $a+H = a + n\Z = a + \{a+nk :
k \in \Z \}$ is the set of all integers which are congruent to $a$
modulo $n$. Note that $a+H = b+H$ if and only if $a \equiv b
\pmod{n}$.

4. Let $H = \{[0], [2], [4]\}$ in the additive group $G=\Z_6$. It is
easy to check that $H$ is a subgroup of $G$. Then $H = [0]+H = [2]+H =
[4]+H$. Also, $[1]+H = [3]+H = [5]+H = \{[1],[3], [5] \}$.

5. Consider $G=\Z_9^\times = \{[1], [2], [4], [5], [7], [8] \}$, the
multiplicative group of units in the ring $\Z_9$. Let $H =
\{[1],[8]\}$ in Then $H$ is a subgroup of $G$, and $H = [1]H = [8]H$,
$[2]H = [7[H] = \{[2],[7]\}$, and $[4]H = [5]H = \{[4],[5]\}$. These
  are all the left cosets.
\end{examples}

It is annoying to always have to distinguish between the
multiplicative and additive notation, so from now on we adopt the
convention that all groups will be multiplicative groups, unless
stated otherwise. We leave it to reader to make the necessary
adjustments in notation for additive groups.


\begin{defn}[equivalence relations induced by a subgroup]
\label{def:H-equiv-rels}
Let $H$ be a subgroup of a given group $G$. Define a relation $\sim_L$
on $G$ by: $a \sim_L b$ whenever $a^{-1}b \in H$.  Define another
relation $\sim_R$ on $G$ by: $a \sim_R b$ whenever $ba^{-1} \in H$.
The relations $\sim_L, \sim_R$ are called \emph{left and right
  equivalence}.
\end{defn}

Note that the relations $\sim_L$ and $\sim_R$ depend on both the
group $G$ and the chosen subgroup $H$.

\begin{lem}
Both relations $\sim_L$ and $\sim_R$ are equivalence relations on $G$.
The equivalence classes for $\sim_L$ are the left cosets of $H$ and
the equivalence classes for $\sim_R$ are the right cosets of $H$.
\end{lem}

\begin{proof}
The proof that $\sim_L$ and $\sim_R$ are equivalence relations is an
easy exercise. 

We prove the claim about left cosets.  Let $a,b \in G$. We have $a
\sim_L b$ if and only if $a^{-1}b \in H$ if and only if $b \in
aH$. Thus $[a] = \{b \in G: a \sim_L b \} = aH$. This proves that the
equivalence class of $a$ is equal to the left coset $aH$.  The proof
of the claim about right cosets is similar.
\end{proof}


In general cosets (left or right) are just subsets of the group $G$,
and are not necessarily subgroups.  From now on we choose to work only
with \emph{left} cosets for the sake of having a definite choice, but
it should be understood that everything we prove about left cosets
applies equally well to right cosets.


\begin{lem}[properties of left cosets] \label{cosetlemma} 
Let $G$ be a group and $H$ a subgroup of $G$. Let $a,b \in G$. Then:
\begin{enumerate}
\item $a \in aH$.
\item $aH=H$ if and only if $a\in H$.
\item $aH=bH$ if and only if $a \sim_L b$.
\item $aH=bH$ if and only if $b=ax$ for some $x\in H$.
\item Any pair of left cosets of $H$ are either disjoint or
  coincide.
\end{enumerate}
\end{lem}

\begin{proof} 
  Exercise.
\end{proof}

\begin{defn}\index{quotient~set}
Let $G$ be a group and $H$ any subgroup of $G$. We write $G/H = \{ aH
\mid a \in G \}$ for the quotient set $G/\!\!\sim_L$ of all left
cosets of $H$. This is called the \emph{quotient} of $G$ by $H$. We
read the notation $G/H$ as ``$G$ mod $H$.''
\end{defn}

By definition, $G/H$ is a set of sets. The elements of $G/H$ are the
left cosets of $H$, which by definition are certain subsets of $G$.
Since $\sim_L$ is an equivalence relation on the set $G$, it follows
from the fundamental theorem of equivalence relations that $G$
\emph{can be expressed as the disjoint union of its distinct left
  cosets}. Those distinct left cosets are the elements of the quotient
set $G/H$.


\begin{defn}\index{index}\index{GH@$[G:H]$}
The number of distinct elements of the set $G/H$ (i.e., its
cardinality as a set), which is the same as the number of distinct
left cosets, is denoted by either $|G/H|$ or $[G:H]$, and is called
the {\em index} of $H$ in $G$. It can be infinite, but it must be a
finite number if $G$ is a finite group.
\end{defn}

Note that as $a$ varies over $G$, there will in general be a lot of
repetition in the left cosets $aH$. When computing the index, you
count just the distinct cosets. 


\begin{examples}
1. Let $G = \Sym_n$ and let $H = \Alt_n$. Then there are just two left
cosets: $G/H = \Sym_n/\Alt_n = \{ \Alt_n, \alpha \Alt_n \}$, where
$\alpha$ is any odd permutation. This is just the splitting of all
permutations into the even ones ($\Alt_n$) and the odd ones ($\alpha
\Alt_n$). So $[\Sym_n : \Alt_n] = 2$.

2. Take $G = \Z$ (under addition) and $H = n\Z = \{nk: k \in \Z
\}$. The left cosets have the form $a + H = a+n\Z = \{ a+nk : k\in \Z
\}$ for various $a\in \Z$. Moreover, $a+H=b+H$ if and only if $-a+b
\in H$, i.e., if and only if $a \equiv b \pmod{n}$. The distinct
cosets are the $a+H$ where $0\le a \le n-1$. Note that $a+n\Z = [a]$,
the congruence class determined by $a$. So the set $G/H$ of left
cosets is 
\[
  G/H = \Z/n\Z = \{n\Z, 1+n\Z, \dots, n-1 +n\Z \} = \{ 
  [0], [1], \dots, [n-1] \} = \Z_n.
\]
So the index is $[\Z: n\Z] = n$.  Note that we have reconstructed
$\Z_n$ in terms of cosets. To say it another way, the coset
construction is a vast generalization of the construction of $\Z_n$
given previously.

3. Let $G=\O(n)$ for $n \ge 2$ and let $H = \SO(n)$.  Recall that the
determinant of any orthogonal matrix is $\pm 1$. By definition,
elements of $\SO(n)$ are proper orthogonal matrices (of determinant
1).  So we have $G/H = \O(n)/\SO(n) = \{ \SO(n), A\cdot \SO(n) \}$,
where $A$ is any improper orthogonal matrix.  This reflects the fact
that the improper orthogonal matrices are those of determinant $-1$.
So the index is $[\O(n):\SO(n)]=2$.

4. Consider $G=\Z_9^\times = \{[1], [2], [4], [5], [7], [8] \}$, the
multiplicative group of units in the ring $\Z_9$. Then $H =
\{[1],[8]\}$ is a subgroup of $G$. The left cosets of $H$ in $G$ are
$H = [1]H$, $[2]H = \{[2],[7]\}$, and $[4]H = \{[4],[5]\}$. So the
index is $[G:H] = 3$. Notice that in this case $|G|/|H| = 6/2 = 3$.
\end{examples}



The next result is a fundamental result about finite groups, with
numerous applications. In particular, it is heavily used in the design
of public-key cryptosystems.

\begin{thm}[Lagrange's Theorem]\index{Lagrange's~theorem} 
\label{thm:Lagrange}%
Let $G$ be a finite group and $H$ a subgroup of $G$. Then $|G| = [G:H]
\cdot |H|$. In words, the order of $G$ is the order of the subgroup
$H$ times its index in $G$.
\end{thm}

\begin{proof}
Since $G$ is a finite set the left cosets must be finite sets as
well. Moreover, {\em all the left cosets must have the same
  cardinality}. This is because $H$ is in bijective correspondence
with $aH$, for any $a \in G$. The correspondence is given by the map
$x \mapsto ax$ ($x\in H$). So all the left cosets have the same
cardinality as the subgroup $H$ and therefore the same cardinality as
one another. (Note that $H = 1H$ is also a left coset.)  There are
precisely $[G:H]$ distinct left cosets. And $G$ is the disjoint union
of the left cosets, since cosets are equivalence classes. So if $m =
[G:H]$ then we have the disjoint union
\[
  G = a_1H \cup a_2H \cup \cdots \cup a_mH 
\]
where these are the {\em distinct} left cosets. Thus $|G| = m
|H|$, as required.
\end{proof}

If $a$ is an element of a group $G$, recall that its order $|a|$ is
the least positive integer $k$ such that $a^k = 1$. Infinite groups
can have elements of infinite order, but if $G$ is a finite group then
every element has finite order. Lagrange's theorem tells us for
example that if $G$ is finite then the order of all its subgroups and
all its elements must divide the group order $|G|$.

\begin{cor}
Suppose $G$ is a finite group. 
\begin{enumerate}
\item[(a)] If $H$ is a subgroup of $G$ then $|H|$ must divide $|G|$.
\item[(b)] If $a\in G$ then its order $|a|$ must divide $|G|$.
\item[(c)] Any group of order $p$, where $p$ is prime, must be cyclic.
\item[(d)] If $H$ is a subgroup of $G$ then $[G:H]=|G|/|H|$. 
\end{enumerate}
\end{cor}

\begin{proof} (a) and (d) are obvious consequences of Lagrange's Theorem. 

(b) Let $H = \gen{a}$. Then $|H|=|a|$ (the cardinality of $H$ is equal
  to the order of $a$).  The statement in part (b) now follows from
  part (a).

(c) Let $G$ be a group of order $p$, where $p$ is prime. Since $p>1$
we know that $G$ must contain at least one element $x$ which is
different from the identity. Then $|x|$ must divide $p$ by part (b).
Moreover, $|x|>1$ since the only element of order $1$ in $G$ is the
identity element. The only divisors of $p$ are $p$ and $1$, so it
follows that $|x| = p$. Hence $G = \gen{x}$ and $G$ is cyclic.
\end{proof}

Lagrange's theorem can also be applied to number theory, to give easy
proofs of both Euler's theorem and Fermat's little theorem. That both
of these famous number theoretic results follow so easily from
Lagrange's theorem illustrates the power of the abstract approach.

Recall that Euler's phi-function\index{Euler's~phi-function}
$\varphi(n)$ is defined to be the number of integers $a$ such that
$1 \le a \le n-1$ and $\gcd(a,n) = 1$. Thus the multiplicative group
$\Z_n^\times$ of units has order $\varphi(n)$. Recall also that the
ring $\Z_n = \{ [a] : a \in \Z \} = \{ [0], [1], \dots, [n-1] \}$,
where $[a]$ is the congruence class of $a$, given by
$[a] = \{ b \in \Z : a \equiv b \pmod{n} \}$.

\begin{cor}\index{Euler's~theorem}
  Let $a$ be any integer.
  \begin{enumerate}
  \item (Euler's theorem) For any positive integer $n$ such that $a,
    n$ are relatively prime, we have $a^{\varphi(n)} \equiv 1
    \pmod{n}$.
  \item\index{Fermat's~little~theorem} (Fermat's little theorem) For
    any prime $p$ such that $p$ does not divide $a$, we have
    $a^{p-1} \equiv 1 \pmod{p}.$
  \end{enumerate}
\end{cor}

\begin{proof}
(a) Let $G = \Z_n^\times$ be the multiplicative group of units in the
  finite ring $\Z_n$. We have $|G| = \varphi(n)$, by definition of
  $\varphi(n)$. If $a, n$ are relatively prime then $\gcd(a,n)=1$ and
  thus $[a] \in \Z_n^\times$. By the first corollary to Lagrange's
  theorem, the order $|[a]|$ must divide $|G|=\varphi(n)$. Suppose
  $|[a]|=k$. Then $[a]^k = [1]$ holds in $G=\Z_n^\times$, where $k$
  divides $\varphi(n)$. There is some $m \in \Z$ such that $\varphi(n)
  = km$. Now the equality $[a]^k = [1]$ implies $([a]^k)^m = [1]^m =
  [1]$. In other words, $[a]^{km} = [a]^{\varphi(n)} = [1]$ holds in
  $G = \Z_n^\times$. This implies (by definition of congruence class
  multiplication) that the equality $[a^{\varphi(n)}] = [1]$ holds in
  $G = \Z_n^\times$.  To finish, recall that this equality in $\Z_n$
  is equivalent to the desired congruence, so the proof is complete.

(b) We can repeat the argument with $p$ in place of $n$, noting that
  $\varphi(p) = p-1$. Note that $p$ does not divide $a$ if and only if
  $\gcd(a,p) = 1$. Alternatively, we can just note that the result in
  (b) is just a special case of that in (a).
\end{proof}

\begin{defn}\index{coset~representatives}
Let $G$ be a group and $H$ a subgroup of $G$. Any set $S$ of elements
of $G$ such that:
\begin{enumerate}
\item $G/H = \{ aH : a \in S \}$, 
\item for all $a,b \in S$, $a \ne b$ implies $aH \ne bH$
\end{enumerate}
is called a set of left coset \emph{representatives} of the quotient
set $G/H$.
\end{defn}


Picking a set of coset representatives is the same as choosing an
element from each distinct coset. If $S$ is such a set, then we can
write $G$ as a \emph{disjoint} union: $G = \bigsqcup_{a \in S} aH$.
If the index $[G:H] = n$ is finite, then we can write this disjoint
union as: $G = a_1H \sqcup a_2H \sqcup \cdots \sqcup a_nH$, where $S =
\{a_1, a_2, \dots, a_n \}$.

One of the left cosets (say $a_1H$) is always the same as the subgroup
$H$ itself, and we may choose the identity element $1$ as its coset
representative. If we have made that choice, then the above becomes $G
= H \sqcup a_2H \sqcup \cdots \sqcup a_mH$, where $S = \{1, a_2,
\dots, a_n \}$. 

In general coset representatives are far from unique. Thus, whenever
we define functions in terms of a set of coset representatives, then
we must pause to verify that our function is well-defined (independent
of the choice of coset representatives). We will see examples later.




\section*{Exercises}
\begin{problems}

\item Prove that the relation $\sim_L$ defined by a subgroup $H$ (see
  \ref{def:cosets}) is an equivalence relation on the group $G$.

\item Prove Lemma \ref{cosetlemma}. 

\item Let $G$ be the additive group $\Z_{2n}$. Let $H = \{ [2k] \in
  \Z_{2n} \colon 0 \le k < n \}$. Show that $H < G$, and compute the
  index $[G \colon H]$.

\item Compute the index $[G:H]$ for the following cases:
  \begin{enumerate}
  \item $H = \{[0],[3]\}$ in $G=(\Z_6,+)$. 
  \item $H = \{[0], [10]\}$ in $G = (\Z_{90},+)$.
  \item $H = \{[1],[4],[13],[16]\}$ in $G = (\Z_{17}^\times, \cdot)$.
  \item $H = \{[1],[7]\}$ in $G = (\Z_{48}^\times, \cdot)$.
  \end{enumerate}

\item List the left cosets in $G/H$ in part (a) and (c) of the
  preceding problem.

\item Let $G=C_n = \{1, x, x^2, \dots, x^{n-1} \}$ be the abstract
  cyclic group\index{cyclic~group} of order $n$, generated by an
  element $x$ of order $n$. Suppose that $k$ divides $n$ and let
  $H = \gen{x^k}$ be the subgroup generated by $x^k$. Describe the
  left cosets in $G/H$ and compute the index $[G:H]$.

\item Describe the distinct left cosets of the additive subgroup
  $(\Z,+)$ in the additive group $(\R,+)$. In other words, describe
  the quotient set $\R/\Z$.

\item Describe the distinct left cosets of the additive subgroup
  $(\R,+)$ in the additive group $(\C,+)$. In other words, describe
  the quotient set $\C/\R$. What is $[\C: \R]$?

\item Let $\R^+$ be the set of positive real numbers.  Describe the
  distinct left cosets of the multiplicative subgroup $(\R^+,\cdot)$
  in the multiplicative group $(\R^\times,\cdot)$. In other words,
  describe the quotient set $\R^\times/\R^+$. What is $[\R^\times:
    \R^+]$?

\item Describe the distinct left cosets of $\SL(2)$ in $\GL(2)$.

\item If $p$ is a prime, describe the distinct left cosets of
  $\SL_2(\F_p)$ in $\GL_2(\F_p)$, and compute the index $[\GL_2(\F_p):
  \SL_2(\F_p)]$.

\item Suppose that $H$ is a subgroup of a group $G$. Show that if $aH
  = Ha$ and $bH=Hb$ then $abH = Hab$.

\item Suppose that $H$ is a subgroup of a group $G$. Show that if $aH
  = Ha$ then $a^{-1} H = H a^{-1} $.

\item Explain why a group $G$ of order 20 has no subgroups or elements
  of order 3, 7, or 9.

\item Show that if $G$ is a group of order $n$ then $x^n = 1$ for
  every $x \in G$. 
  

\item Prove that a group of prime order has no subgroups other than
  itself and the trivial subgroup.

\item Prove that every element except the identity has order $p$ in a
  group of prime order $p$.


\item In the multiplicative group $\F_{11}^\times$ we have $[2]^2 =
  [4]$, $[2]^4=[4]^2 =[5]$, and $[2]^5 = [2]^4 \cdot [2] = [5] \cdot
  [2] = [10]$. Use one of the corollaries to Lagrange's theorem to
  explain why this immediately implies that $[2]$ must have order 10
  in the group.

\item Suppose that $H$, $K$ are finite subgroups of a group $G$, and
  let $m = |H|$, $n = |K|$.
  \begin{enumerate}
  \item Show that the order $|H \cap K|$ of $H \cap K$ must be a
    common divisor of $m,n$.
  \item Show that if $m,n$ are relatively prime then $H \cap K =
    \{1\}$ is the trivial subgroup.
  \end{enumerate}
  
\item Suppose that $G$ is a group of order $pq$ where $p,q$ are
  distinct primes. Show that if $G$ is not cyclic then every element
  in $G$ except the identity must have order $p$ or $q$.

\end{problems}
\end{document}


\item Show that the multiplicative group $\F_7^\times$ is a cyclic
  group, by finding a generator. Then do the same for
  $\F_{13}^\times$. 

\item Show that the additive group $\Z/m\Z$ is cyclic, by finding a
generator.

\item Is the additive group $\Z$ cyclic? Why or why not?

\item Prove that if $G$ is a group and if $a$ is an element of $G$ of
  order $r$ then $(a^k)^{-1} = a^{r-k}$ for all $0\le k < r$.

\item Prove that $(\Z_n,+)$ is generated by the congruence class $[a]$
  if and only if $\gcd(a,n) = 1$.

\item Compute the index $[\GL_n(\F_p): \SL_n(\F_p)]$.  Justify your
  answer.
